\documentclass{article}
\usepackage{scribe}
\usepackage{epsfig}
\renewcommand{\Pr}[1]{\textrm{\textup{Pr}}\left( #1 \right)}

\title{Approximation Algorithms (continued)}
\date{October 20, 2004}
\author{Lecturer: David Karger\\ Scribe: Sergiy Sidenko}

\begin{document}
\maketitle

\section{Relative Approximation Algorithms}

Since absolute approximation algorithms are known to exist for so few optimization problems, a
better class of approximation algorithms to consider are relative approximation algorithms.
Because they are so commonplace, we will refer to them simply as approximation algorithms.

\begin{definition}
  \textbf{An $\alpha$-approximation algorithm} finds a solution of value at most $\alpha \cdot
  OPT(I)$ for a minimization problem and at least $OPT(I) / \alpha$ for a maximization problem ($\alpha \ge 1$).
\end{definition}

Note that although $\alpha$ can vary with the size of the input, we will only consider those cases
in which it is a constant. To illustrate the design and analysis of $\alpha$-approximation
algorithm, let us consider the Parallel Machine Scheduling problem, a generic form of load
balancing.

\begin{description}
\item[Parallel Machine Scheduling:]

Given $m$ machines $m_i$ and $n$ jobs with processing times $p_j$, assign the jobs to the machines
to minimize the load

$$
\max\limits_{i} \sum\limits_{j \in i} p_j,
$$

the time required for all machines to complete their assigned jobs. In scheduling notation, this
problem is described as $\mathrm {P \parallel C_{\max}}$.

\end{description}

A natural way to solve this problem is to use \textit{greedy algorithm} called \textbf{list
scheduling}.

\begin{definition}
A \textbf{list scheduling} algorithm assigns jobs to machines by assigning each job to the least
loaded machine.
\end{definition}

Note that the order in which the jobs are processed is not specified.

\subsubsection{Analysis}

To analyze the performance of list scheduling, we must somehow compare its solution for each
instance $I$ (call this solution $A(I)$) to the optimum $OPT(I)$. But we do not know how to obtain
an analytical expression for $OPT(I)$. Nonetheless, if we can find a meaningful lower bound
$LB(I)$ for $OPT(I)$ and can prove that $A(I) \le \alpha \cdot LB(I)$ for some $\alpha$, we then
have

$$
\begin{array}{lcl}
A(I)    & \le & \alpha \cdot LB(I) \\
        & \le & \alpha \cdot OPT(I).
\end{array}
$$

Using the idea of lower-bounding $OPT(I)$, we can now determine the performance of list
scheduling.

{\claim{List scheduling is a $(2-1/m)$-approximation algorithm for Parallel Machine Scheduling.}}

\begin{proof}

Consider the following two lower bounds for the optimum load $OPT(I)$:

\begin{itemize}

\item the maximum processing time $p = \max_j p_j,$
\item the average load $L = \sum_j p_j/m.$

\end{itemize}

The maximum processing time $p$ is clearly lower bound, as the machine to which the corresponding
job is assigned requires at least time $p$ to complete its tasks. To see that the average load is
a lower bound, note that if all of the machines could complete their assigned tasks in less than
time $L$, the maximum load would be less than the average, which is a contradiction. Now suppose
machine $m_i$ has the maximum runtime $L = c_{\max}$, and let job $j$ be the last job that was
assigned to $m_i$. At the time job $j$ was assigned, $m_i$ must have had the minimum load (call it
$L_i$), since list scheduling assigns each job to the least loaded machine. Thus,

$$
\begin{array}{lcl}
\sum\limits_{\mbox{machine i}} p_i  & \ge & m L_i + p_j \\
                                    & \ge & m (L - p_j) + p_j
\end{array}
$$

Therefore,

$$
\begin{array} {lcl}
OPT(I)  & \ge & \frac{1}{m} \left( m (L - p_j) + p_j \right) \\
        & = & L - (1-1/m) p_j, \\
\mbox{then}        \\
L       & \le & OPT(I) + (1 - 1/m) p_j \\
        & \le & OPT(I) + (1 - 1/m) OPT(I) \\
        & \le & (2 - 1/m) OPT(I).
\end{array}
$$

The solution returned by list scheduling is $c_{\max}$, and thus list scheduling is a
$(2-1/m)$-approximation algorithm for Parallel Machine Scheduling.

The example with $m(m-1)$ jobs of size $1$ and one job of size $m$ for $m$ machines shows that we
cannot do better than $(2-1/m)OPT(I)$.

\end{proof}

\section{Polynomial Approximation Schemes}

The obvious question to now ask is how good an $\alpha$ we can obtain.

\begin{definition}
A \textbf{polynomial approximation scheme (PAS)} is a set of algorithms $\{ A_\varepsilon \}$ for
which each $A_\varepsilon$ is a polynomial-time $(1+\varepsilon)$-approximation algorithm.
\end{definition}

Thus, given any $\varepsilon >0$, a PAS provides an algorithm that achieves a
$(1+\varepsilon)$-approximation. In order to devise a PAS we can use the method called
$k$-enumeration.

\begin{definition}
An approximation algorithm using \textbf{$k$-enumeration} finds an optimal solution for the $k$
most important elements in the problem and then uses an approximate polynomial-time method to
solve the reminder of the problem.
\end{definition}

\subsection{A Polynomial Approximation Scheme for Parallel Machine Scheduling}

We can do the following:

\begin{itemize}

\item Enumerate all possible assignments of the $k$ largest jobs.

\item For each of these partial assignments, list schedule the remaining jobs.

\item Return as the solution the assignment with the minimum load.

\end{itemize}

Note that in enumerating all possible assignments of the $k$ largest jobs, the algorithm will
always find the optimal assignment for these jobs. The following claim demonstrates that this
algorithm provides us with a PAS.

{\claim{For any fixed $m$, $k$-enumeration yields a polynomial approximation scheme for Parallel
Machine Scheduling.}}

\begin{proof}

Let us consider the machine $m_i$ with maximum runtime $c_{\max}$ and the last job that $m_i$ was
assigned.

If this job is among the $k$ largest, then it is scheduled optimally, and $c_{\max}$ equals
$OPT(I)$.

If this job is not among the $k$ largest, without loss of generality we may assume that it is the
$(k+1)$th largest job with processing time $p_{k+1}$. Therefore,
$$
%\begin{array} {lcl}
A(I) \le OPT(I) + p_{k+1}.
$$

However, $OPT(I)$ can be bound from below in the following way:
$$
OPT(I) \ge \frac{k p_k}{m},
$$

because $\frac{k p_k}{m}$ is the minimum average load when the largest $k$ jobs have been
scheduled.

Now we have:
$$
\begin{array} {lcl}
A(I)    & \le & OPT(I) + p_{k+1} \\
        & \le & OPT(I) + OPT(I) \frac{m}{k} \\
        & = & OPT(I) \left( 1+\frac{m}{k} \right).
\end{array}
$$

Given $\varepsilon > 0$, if we let $k$ equal $m/\varepsilon$, we will get
$$
c_{\max} \le (1+\varepsilon) OPT(I).
$$

Finally, to determine the running time of the algorithm, note that because each of the $k$ largest
jobs can be assigned to any of the $m$ machines, there are $m^k = m^{m/\varepsilon}$ possible
assignments of these jobs. Since the list scheduling performed for each of these assignments takes
$O(n)$ time, the total running time is $O(nm^{m/\epsilon})$, which is polynomial because $m$ is
fixed. Thus, given an $\varepsilon >0$, the algorithm is a $(1+\varepsilon)$-approximation, and so
we have a polynomial approximation scheme.

\end{proof}

\section{Fully Polynomial Approximation Schemes}

Consider the PAS in the previous section for ${\rm P \parallel C_{\max}}$. The running time for
the algorithm is prohibitive even for moderate values of $\varepsilon$. The next level of
improvement, therefore, would be approximation algorithms that run in time polynomial in $1/
\varepsilon$, leading to the definition below.

\begin{definition}
\textbf{Fully Polynomial Approximation Scheme (FPAS)} is a set of approximation algorithms such
that each algorithm $A(\varepsilon)$ in this set runs in time that is polynomial in the size of
the input, as well as in $1/ \varepsilon$.
\end{definition}

There are few $NP$-complete problems that allow for FPAS. Below we discuss FPAS for the Knapsack
problem.

\subsection{A Fully Polynomial Approximation Scheme for the Knapsack Problem}

The Knapsack problem receives as input an instance $I$ of $n$ items with profits $p_i$, sizes
$s_i$ and knapsack size (or capacity) $B$. The output of the Knapsack problem is the subset $S$ of
items of total size at most $B$, and that has profit:

$$
\max \sum\limits_{i \in S} p_i.
$$

Suppose now that the profits are integers; then we can write a $DP$ algorithm based on the minimum
size subset with profit $p$ for items $1, \, 2, \, \ldots, \, r$ as follows:

$$
M(r,p) = \min \left\{ M(r-1,p), M(r-1, p-p_r) + s_r \right\}.
$$

The corresponding table of values can be filled in $O\left(n \sum_i p_i \right)$ (note that this
is not FPAS in itself).

Now, we consider the general case where the profits are not assumed to be integers. Once again, we
use a rounding technique but one that can be considered a generic approach for developing FPAS for
other $NP$-complete problems that allows for FPAS. Suppose we multiplied all profits $p_i$ by
$n/\varepsilon \cdot OPT$; then the new optimal objective value is apparently $n/\varepsilon$.
Now, we can round the profits down to the nearest integer, and hence the optimal objective value
decreases at most by $n$; expressed differently, the decrease in objective value is at most
$\varepsilon \cdot OPT$. Using the $DP$ algorithm above, we can therefore find the optimal
solution to the rounded problem in $O(n^2/ \varepsilon)$, thus providing us with FPAS in
$1/\varepsilon$.

\end{document}
