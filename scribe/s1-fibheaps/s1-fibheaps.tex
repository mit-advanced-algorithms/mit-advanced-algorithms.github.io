\documentclass{article}

\title{Fibonacci Heaps}
\date{September 7, 2005}
\author{Lecturer: David Karger\\ Scribes: David G. Andersen,  Ioana Dumitriu, John Dunagan, Akshay Patil (2003)}

\begin{document}

%%%%%%%%%%%%%%%%%%%%%%%%%%%%%%%%%%%%%%%%%%%%%%%%%%%%%%%%%%%%%%%%%%%%%%
% Your notes start here!
%%%%%%%%%%%%%%%%%%%%%%%%%%%%%%%%%%%%%%%%%%%%%%%%%%%%%%%%%%%%%%%%%%%%%%
%
% For theorems, lemmas, definitions, remarks, etc. use commands
% {\theorem{...}}, {\lemma{...}}, {\definition{...}}, etc.
% For proofs, use \begin{proof} ... \end{proof}
%
% For postscript figures (.ps) use the following block:
%
% \begin{figure}[h]
% \begin{center}
% \mbox{\psfig{figure=notes-nn-fig-mm.ps}}
% \caption{A very nice picture.}
% \label{fig:picture}
% \end{center}
% \end{figure}
%

% For encapsulated postscript figures (.eps) use the following block:
%  (also change documentstyle line )
% \begin{figure}[h]
% \begin{center}
% \mbox{\epsfbox{notes-nn-fig-mm.eps}}
% \caption{A very nice picture.}
% \label{fig:picture}
% \end{center}
% \end{figure}
%


%%%%%%%%%%%%%%%%%%%%%%%%%%%%%%%%%%%%%%%%%%%%%%%%%%%%%%%%%%%%%%%%%%%%%%

\section{Motivation and Background}

Priority queues are a classic topic in theoretical computer
science. As we shall see, Fibonacci Heaps provide a fast and  elegant
solution. The
search for a fast priority queue implementation is motivated
primarily by two network optimization algorithms: Shortest Path and
Minimum Spanning Tree (MST).

\subsection{Shortest Path and Minimum Spanning Trees}

Given a graph $G (V,E)$ with vertices $V$ and edges $E$ and a length
function $l: E \rightarrow \Re^+$. We define the Shortest Path and MST
problems to be, respectively:
\begin{description}
\item[shortest path.]  For a fixed source $s \in V$, find the shortest
  path to all vertices $v \in V$
\item[minimum spanning tree (MST).]  Find the minimum length set of
  edges $F \subset E$ such that $F$ connects all of $V$.
\end{description}

Note that the MST problem is the same as the Shortest Path problem,
except that the source is not fixed. Unsurprisingly, these two problems
are solved by very similar algorithms, Prim's for MST and Djikstra's for
Shortest Path. The algorithm is:

\begin{enumerate}
\item Maintain a priority queue on the vertices
\item Put $s$ in the queue, where $s$ is the start vertex (Shortest
Path) or any vertex (MST). Give $s$ a key of 0.
\item Repeatedly delete the minimum-key vertex $v$ from the queue and
mark it ``scanned''

   For each neighbor $w$ of $v$:

        If $w$ is not in the queue and not scanned, add it with key:
        \begin{itemize}
                \item Shortest Path:  $key(v) + length(v \rightarrow w)$
                \item MST:  $length(v \rightarrow w)$
        \end{itemize}

        If, on the other hand, $w$ is in the queue already, then
        decrease its key to the minimum of the value calculated above
        and $w$'s current key.

\end{enumerate}

\subsection{Heaps}

The classical answer to the problem of maintaining a priority queue on
the vertices is to use a binary heap, often just called a heap.
Heaps are commonly used because
they have good bounds on the time required for the following
operations:

\begin{tabular}{|l|l|}
\hline
insert & $O(\log n$) \\
delete-min & $O(\log n$) \\
decrease-key & $O(\log n$) \\
\hline
\end{tabular}

If a graph has $n$ vertices and $m$ edges, then running either Prim's or
Djikstra's algorithms will require $O(n \log n)$ time for inserts and
deletes.  However, in the worst case,
we will also perform $m$ decrease-keys, because we may have to perform
a key update every time we come across a new edge.
This will take $O(m \log n)$
time.  Since the graph is connected, $m \geq n$, and the overall time
bound is given by $O(m \log n)$.

Since $m \geq n$, it would be nice to have cheaper key decreases.
A simple way to do this is to use \emph{d-heaps}.

\subsection{d-Heaps}

d-heaps make key reductions cheaper at the expense of more costly
deletions. This trade off is accomplished by replacing the binary heap
with a d-ary heap---the branching factor (the maximum number of
children for any node) is changed from 2 to $d$.  The depth of the
tree then becomes $\log _{d}(n)$.  However, delete-min operations must
now traverse all of the children in a node, so their cost goes up to
$d \log _{d}(n)$.  Thus, the running time of the algorithm becomes
$O(n d \log _{d}(n) + m \log _{d}(n))$. Choosing the optimal $d=m/n$
to balance the two terms, we obtain a total running time of $O(m
\log_{m/n}n)$.

When $m = n^2$, this is $O(m)$, and when $m=n$, this is $O(n \log
n)$. This seems pretty good, but it turns out we can do much better.

\subsection{Amortized Analysis}

Amortized analysis is a technique for bounding the running time of an
algorithm. Often we analyse an algorithm by analyzing the individual
operations that the algorithm performs and then multiplying the total
number of operations by the time required to perform an operation.
However, it is often the case that an algorithm will on occasion
perform a very expensive operation, but most of the time the operations
are cheap. Amortized analysis is the name given to the technique of
analyzing not just the worst case running time of an operation but the
average case running time of an operation. This will allow us to balance
the expensive-but-rare operations against their cheap-and-frequent peers.

There are several methods for performing amortized analysis;  for a
good treatment, see \emph{Introduction to Algorithms} by Cormen,
Leiserson, and Rivest.  The method of amortized analysis used
to analyze Fibonacci heaps is the potential method:

\begin{itemize}
 \item Measure some aspect of the data structure using a potential
function. Often this aspect of the data structure corresponds to what we
intuitively think of as the complexity of the data structure or the
amount by which it is out of kilter or in a bad arrangement.
 \item If operations are only expensive when the data structure is
complicated, and expensive operations can also clean up
(``uncomplexify'') the data structure, and it takes many cheap
operations to noticeably increase
the complexity of the data structure, then we can \emph{amortize} the cost of
the
expensive operations over the cost of the many cheap operations to obtain a low
average cost.
\end{itemize}

Therefore, to design an efficient algorithm, we
want to force the user to perform many operations to make the
data structure complicated, so that the work doing the expensive
operation and cleaning up the
data structure is amortized over those many operations.

We compute the \emph{potential} of
the data structure  by using a potential function $\Phi$
that maps the data structure ($DS$) to a real number $\Phi(DS)$.  Once
we have defined $\Phi$, we calculate the cost of the $i^{th}$ operation by:

$$cost_{\it amortized}(operation_i) = cost_{\it actual}(operation_i) +
\Phi(DS_{i}) - \Phi(DS_{i-1})$$

where $DS_{i}$ refers to the state of the data structure after the
$i^{th}$ operation. The sum of the amortized costs is then

$$\sum cost_{\it actual}(operation_i) + \Phi_{\it final} - \Phi_{\it initial}$$.

If we can prove that $\Phi_{final} \geq \Phi_{initial}$, then we've
shown that the amortized costs bound the real costs, that is,
$\sum{cost_{amortized}} \geq \sum{cost_{actual}}$. Then we can just analyze
the amortized costs and show that this isn't too much, knowing that our
analysis is useful. Most of the time it is obvious that
$\Phi_{final} \geq \Phi_{initial}$ and the real work is in coming up with a
good potential function.

\section{Fibonacci Heaps}

The Fibonacci heap data structure invented by Fredman and Tarjan in 1984
gives a
very efficient implementation of the priority queues. Since the goal is
to find a way to minimize the number of operations needed to compute the
MST or SP, the kind of operations that we are interested in are
\emph{insert}, \emph{decrease-key}, \emph{merge}, and
\emph{delete-min}. (We haven't covered why \emph{merge} is a useful operation
yet, but it will become clear.) The method to achieve this minimization
goal is laziness -- \textbf{``do work
only when you must, and then use it to simplify the structure as much as
possible so that your future work is easy''}. This way, the user is
forced to do many cheap operations in order to make the data structure
complicated.

Fibonacci heaps make use of heap-ordered trees. A heap-ordered tree is
one that maintains the \emph{heap property}, that is, where
$key(parent) \leq key(child)$ for all nodes in the tree.

\vspace{.25cm}

\noindent A Fibonacci heap \emph{H} is a collection of heap-ordered trees that
have
the following properties:
\begin{enumerate}
\item The roots of these trees are kept in a doubly-linked list (the
``root list'' of $H$);
\item The root of each tree contains the minimum element in that tree
(this follows from being a heap-ordered tree);
\item We access the heap by a pointer to the tree root with the overall
minimum key;
\item For each node $x$, we keep track of the \emph{rank} (also known as
the \emph{order} or \emph{degree}) of $x$, which is just the number of
children $x$ has; we also keep track of the \emph{mark} of $x$, which is
a Boolean value whose role will be explained later.
\end{enumerate}

%insert figure of a Fib. heap here

\vspace{.25cm}

\noindent For each node, we have at most four pointers that respectively
point to the node's parent, to one of its children, and to two of its
siblings. The sibling pointers are arranged in a doubly-linked list (the
``child list'' of the parent node).
Of course, we haven't described how the operations on
Fibonacci heaps are implemented, and their implementation will add some
additional properties to $H$.
Here are some elementary operations used in maintaining Fibonacci heaps.

\subsection{Inserting, merging, cutting, and marking.}

\textbf{Inserting a node $x$.} We create a new tree containing only $x$
and insert it into the root list of $H$; this is clearly an $O(1)$
operation.

\vspace{.25cm}

\noindent \textbf{Merging two trees.} Let $x$ and $y$ be the roots of
the two trees we want to merge; then if the key in $x$ is no less than
the key in $y$, we make $x$ the child of $y$; otherwise, we make $y$ the
child of $x$. We update the appropriate node's rank and the appropriate
child list; this takes $O(1)$
operations.  

\vspace{.25cm}

\noindent \textbf{Cutting a node.} If $x$ is a root in $H$, we are
done. If $x$ is not a root in $H$, we remove $x$ from the child list
of its parent, and insert it into the root list of $H$, updating the
appropriate variables (the rank of the parent of $x$ is decremented,
etc.). Again, this takes $O(1)$ operations. (We assume that when we
want to find a node, we have a pointer hanging around that accesses it
directly, so actually finding the node takes $O(1)$ time.)

\vspace{.25cm}

\noindent \textbf{Marking.} We say that $x$ is marked if its mark is set
to ``true'', and that it is unmarked if its mark is set to ``false''. A
root is always unmarked. We mark $x$ if it is not a root and it loses a
child (i.e., one of its children is cut and put into the root-list). We
unmark $x$ whenever it becomes a root. We
will make sure later that no marked node loses another child before it
itself is cut (and reverted thereby to unmarked status).

\subsection{Decreasing keys and Deleting mins}

\noindent At first, \emph{decrease-key} does not appear to be any
different than \emph{merge} or \emph{insert}; just find the node and cut
it off from its parent, then insert the node into the root list with a
new key. This requires removing it from its parent's
child list, adding it to the root list, updating the parent's rank,
and (if necessary) the pointer to the root of smallest key. This takes
$O(1)$ operations.

\vspace{.25cm}

\noindent The \emph{delete-min} operation works in the same way as
\emph{decrease-key}: Our pointer into the Fibonacci heap is a pointer to
the minimum keyed node, so we can find it in one step.
We remove this root of smallest key, add
its children to the root-list, and scan through the linked list of all
the root nodes to find the new root of minimum
key. Therefore, the cost of a \emph{delete-min} operation is $O(\mbox{\# of
children })$ of the root of minimum key plus $O(\mbox{\# of root
nodes})$; in order to make this sum as small as
possible, we have to add a few bells and whistles to the data structure.

\subsection{Population Control for Roots}

We want to make sure that every node has a small number of
children. This can be done by
ensuring that the total number
of descendants of any node is exponential in
the number of its children. In the absence of any ``cutting'' operations
on the nodes,
one way to do this is by \textbf{only}
merging trees that have the same number of children (i.e, the same
rank). In the next section we'll see exactly when we want to merge trees.
It is relatively easy to see that if we only ever merge trees
that have the same rank and never cut any nodes, 
the total number of descendants (counting
onself as a descendant) is always
($2^{ \mbox{\# of children}}$). The resulting structure is called a binomial
tree
because the number of descendants at distance $k$ from the root in a
tree of size $n$ is
exactly $n \choose k$. Binomial heaps preceded Fibonacci heaps and were
part of the inspiration for them. We now present Fibonacci heaps in full
detail.

\subsection{Actual Algorithm for Fibonacci Heaps}

\begin{itemize}
\item Maintain a list of heap-ordered trees.
\item \emph{insert}: add a degree 0 tree to the list.
\item \emph{delete-min}: We can find the node we wish to delete
immediately since our handle to the entire data structure is a pointer
to the root with minimum key. Remove the smallest root, and add its children
to the list of roots. Scan the roots to find the next minimum. Then
consolidate all the trees (merging trees of equal rank) until there is
$\leq 1$ of each rank. (Assuming that we have achieved the property that
the number of descendants is exponential in the number of children
for any node, as we did in the
binomial trees, no node has rank
$> c \log n$ for some constant $c$. Thus consolidation leaves us with
$O(\log n)$ roots.) The consolidation is performed by allocating buckets
of sizes up to the maximum possible rank for any root node, which we
just showed to be $O(\log n)$. We put each node into the appropriate
bucket, at cost $O(\log n)$ + $O(\mbox{\# of roots})$. Then we march
through the buckets, starting at the smallest one, and consolidate
everything possible. This again incures cost $O(\log n)$ + $O(\mbox{\#
of roots})$. This is the only time we do consolidation.
\item \emph{decrease-key}: cut the node, change its key, and insert it into
the root list as before, Additionally, if the parent of the node was
unmarked, mark it. If the parent of the node was marked, cut it off
also. Recursively do this until we get up to an unmarked node. Mark it.

\end{itemize}

\subsection{Actual Analysis for Fibonacci Heaps}

Define $\Phi(DS) = (k \cdot$ \# of roots in $DS$ + 2 $\cdot$ \# marked
bits in $DS$). Note that \emph{insert} and \emph{delete-min} do not ever
cause nodes to be marked - we can analyze their behaviour without
reference to marked and unmarked bits. The parameter $k$
is a constant that we will conveniently specify later. We now analyze
the costs of the operations in terms of their amortized costs (defined
to be the real costs plus the changes in the potential function).

\begin{itemize}
\item \emph{insert}: the amortized cost is O(1). O(1) actual work plus $k$ *
O(1) change in potential for adding a new root. O(1) + $k$O(1) = O(1)
total amortized cost.
\item \emph{delete-min}: for every node that we put into the root list (the
children of the node we have deleted), plus every node that is already
in the root list, we do constant work putting that node into a bucket
corresponding to its rank and
constant work whenever we merge the node. 
Our real costs are putting the roots into buckets (O($\#roots$)), walking
through the buckets (O($\log n$)), and doing the consolidating
tree merges (O($\#roots$)).  On the other hand, our change in potential is
$k*(\log n-\#roots)$ (since there are at most $\log n$ roots after consolidation).
Thus, total amortized cost is O($\#roots$) + O($\log n$) + $k*(\log n-\#roots)$ =
O($\log n$).
\item \emph{decrease-key}: The real cost is O(1) for the cut, key decrease
and re-insertion. This also increases the potential function by O(1)
since we are adding a root to the root list, and maybe by another 2
since we may mark a node. The only problematic issue
is the possibility of a ``cascading cut'' - a cascading cut is the name
we give to a cut that causes the node above it to cut because it was
already marked, which causes the
ndoe above it be cut since it too was alrady marked, etc. This
can increase the actual cost of the operation to (\# of nodes already
marked). Luckily, we can pay for this with the potential function! Every
cost we incur from having to update pointers due to a marked node that
was cut is offset by the decrease in the potential function when that
previously marked node is now left unmarked in the root list. Thus the
amortized cost for this operation is just O(1).

\end{itemize}

The only thing left to prove is that for every node in every tree in our
Fibonacci heap, the number of descendants of that node is exponential in
the number of children of that node, and that this is true even in the
presence of the ``weird'' cut rule for marked bits.  We must prove this
in order to substantiate our earlier assertion that all nodes
have degree $\leq \log n$.

\subsection{The trees are big}

Consider the children of some node $x$ in the order in which they were
added to $x$.

$Lemma:$ The $i^{th}$ child to be added to $x$ has rank at
least $i-2$.

$Proof:$ Let $y$ be the $i^{th}$ child to be added to $x$. When it was added,
$y$ had at least $i-1$ children. This is true because we can currently see
$i-1$ children that were added earlier, so they were there at the time
of the $y$'s addition. This means that $y$ had
at least $i-1$ children at the time of it's merger, because we only merge
equal ranked nodes. Since a node could
not lose more than one child without being cut itself, it must be
that $y$ has at least $i-2$ children ($i-1$ from when it was added, and 
no more than a potential $1$ subsequently lost). 
\qed

Note that if we had been working with a binomial tree, the appropriate lemma
would
have been $rank = i-1$ not $\geq i-2$.

Let $S_k$ be the minimum number of descendants of a node with $k$
children. We have $S_0 = 1,  S_1 = 2 $ and, $$S_k \geq
\sum_{i=0}^{k-2}S_i$$
This recurrence is solved by $S_k \geq F_{k+2}$, the $(k+2)^{th}$ Fibonacci
number. Ask anyone on the street and that person will tell you that the
Fibonacci numbers grow exponentially; we have proved $S_k \geq 1.5^k$,
completing our analysis of Fibonacci heaps.

\subsection{Utility}

Only recently have problem sizes increased to the point where Fibonacci
heaps are beginning to appear in practice.
Further study of this issue might make an
interesting term project; see David Karger if you're curious.

Fibonacci Heaps allow us to improve the running time in Prim's
and Djikstra's algorithms. A more thorough analysis of this will be
presented in the next class.



%%%%%%%%%%%%%%%%%%%%%%%%%%%%%%%%%%%%%%%%%%%%%%%%%%%%%%%%%%%%%%%%%%%%%%
% Your notes end here!
%%%%%%%%%%%%%%%%%%%%%%%%%%%%%%%%%%%%%%%%%%%%%%%%%%%%%%%%%%%%%%%%%%%%%%

\end{document}


