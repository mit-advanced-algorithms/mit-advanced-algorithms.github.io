\documentclass{article}
\usepackage{scribe}
\usepackage{epsfig}
\renewcommand{\Pr}[1]{\textrm{\textup{Pr}}\left( #1 \right)}
\begin{document}


%%%%%%%%%%%%%%%%%%%%%%%%%%%%%%%%%%%%%%%%%%%%%%%%%%%%%%%%%%%%%%%%%%%%%%
% Lecture command takes 4 arguments:
%               ordinal number of the lecture
%               date of the lecture
%               lecturer
%               scribe---that is you.
%
% Fill out the next line with this information !!!!
\lecture{17}{11/8/2004}{David Karger}{Matt Rasmussen}

%%%%%%%%%%%%%%%%%%%%%%%%%%%%%%%%%%%%%%%%%%%%%%%%%%%%%%%%%%%%%%%%%%%%%

%%%%%%%%%%%%%%%%%%%%%%%%%%%%%%%%%%%%%%%%%%%%%%%%%%%%%%%%%%%%%%%%%%%%%%
% Your notes start here!
%%%%%%%%%%%%%%%%%%%%%%%%%%%%%%%%%%%%%%%%%%%%%%%%%%%%%%%%%%%%%%%%%%%%%%
%
% For theorems, lemmas, definitions, remarks, etc. use commands
% {\theorem{...}}, {\lemma{...}}, {\definition{...}}, etc.
% For proofs, use \begin{proof} ... \end{proof}
%
% For postscript figures (.ps) use the following block:
%
% \begin{figure}[h]
% \begin{center}
% \mbox{\psfig{figure=notes-nn-fig-mm.ps}}
% \caption{A very nice picture.}
% \label{fig:picture}
% \end{center}
% \end{figure}
%

% For encapsulated postscript figures (.eps) use the following block:
%  (also change documentstyle line )
% \begin{figure}[h]
% \begin{center}
% \mbox{\epsfbox{notes-nn-fig-mm.eps}}
% \caption{A very nice picture.}
% \label{fig:picture}
% \end{center}
% \end{figure}
%


%%%%%%%%%%%%%%%%%%%%%%%%%%%%%%%%%%%%%%%%%%%%%%%%%%%%%%%%%%%%%%%%%%%%%%


\begin{center}
{\large \bf Sweep Line}
\end{center}

\begin{definition}
  \textbf{Sweep Line Technique:} Given some planar problem, sweep a line
through the plane dealing with events that occur on the line and leaving behind a
solved portion of the problem.
\end{definition}


\section{Convex Hull}

The Convex Hull problem is to find the smallest enclosing convex polygon of a
set of given points in the plane.

\subsection{Algorithm}

One method for solving the convex hull problem is to use a sweep line technique
to find the upper envelope of the hull.  The lower envelope of the convex hull
can be found by rerunning the following algorithm with only slight
modifications.   

Use a vertical sweep line that sweeps from negative infinity to positive
infinity on the x-axis.  As the line sweeps, we will maintain a partial convex
hull for the points left of the sweep line.  When the sweep line crosses a new
point we will need to update the partial convex hull to include the new point.
After the sweep line has gone past all the points, we will have a complete upper
envelope of the convex hull.

To determine the order in which the sweep line will cross points we can use a
priority queue such as min-heap that will order points by their x-coordinates. 
Therefore, the only remaining issue is how to insert a new point into an
existing convex hull. When inserting a new point, two cases arise: (1) either
the existing convex hull can be extended to include the new point or (2) the
new point requires the existing convex hull to be modified.  Informally, we can
distinguish the two cases by noticing that in case (1) the new point causes a
``right turn'' in the hull's boundary, whereas in case (2) the new point causes
a ``left turn''. This can be determined mathematically by finding the angle
between the right most line segment of envelope and the line segment of the
right most envelope point and the new point.  If the angle is less than
180  degrees then we have a case (1), otherwise we have a case (2).

{\bf case 1: } 
Since the new point can be safely added to the existing envelope without
violating its convexity, the new point is directly added to the list of
envelope points.

{\bf case 2: } In this case, the new point cannot be safely added to the
existing envelope without violating its convexity. Therefore, the existing
envelope must be modified to accommodate the new point.  This can be done by the
following procedure.  At the new point the envelope.  Perform a convexity check
on the predecessor to the new point.  If the convexity check fails, delete the
point from the envelope.  Now the new point has a new predecessor.  Repeat the
convexity check and deletion on the predecessor of the new point until the
convexity check is satisfied, at which point we will have a valid upper envelope
again.

\subsection{Analysis}

Let there be $n$ points in the problem.  Creating the min-heap and performing 
$n$ extract-mins will have a $O(n\log{n})$ runtime.  Convexity checks take a
constant amount of algebra and therefore take $O(1)$.  Case (1) simply 
extends a linked list in $O(1)$ and may occur a maximum of $n$ times, which
implies a $O(n)$ runtime.  To find the runtime contribution of case (2), we
bound the total number of deletions that may be made from the envelope over the
course of the algorithm.  This bound is $n$ since each point can be deleted at
most once.  Therefore, the amortized cost of case (2) is also $O(n)$. This
gives a total runtime of $O(n\log{n})$.


\section{Segment Intersections}

In the Segment Intersections problem, we are given $n$ line segments and must
output the coordinates of all pair-wise intersections.

\subsection{Algorithm}

In this algorithm, we will use a horizontal sweep line that sweeps from positive
infinity to negative infinity along the y-axis.  The idea, will be to 
output each intersection as we cross it.  Also we will strive to make the
algorithm {\em output sensitive}.  That is, although there may be $O(n^2)$
intersections, which would require an $O(n^2)$-time algorithm, we will run in
much less time if the number of intersections $k$ is much less than $n$.

Let a segment be considered ``active'' if it crosses the current sweep line.  
As the sweep line sweeps, we will encounter three types of events: 

\begin{enumerate}
    \item new segment becomes active
    \item old segment becomes inactive
    \item two active segments cross
\end{enumerate}

Events (1) and (2) can be handled with a min-heap containing segment endpoints
that are ordered by their y-coordinates.  In dealing with the third case, the
key idea is that only ``neighboring'' segments on the sweep line can cross.  To
provide a quick lookup of neighboring segments we can use a Binary Search Tree
(BST) to store the segments by the x-coordinate of their intersection with the
sweep line.  After inserting a new line into the BST, determine if it will
eventually cross with its neighbors.  If so, insert a crossing event into the
event queue.  Later, when a crossing event occurs, we output an intersection, 
swap the order of the lines segments in the BST, and find their two new 
neighbors and possible future crossings.

\subsection{Analysis}

Inserting and extracting line segments activations and deactivations from the 
event queue will take a total of $O(n \log{n})$ time.  The total time inserting 
into the BST will take $O(n \log{n})$ and the total deletes from the BST will 
take $O(n)$.  The number of crossing events is $k$, therefore the total time
needed for inserting crossing events into the event queue is $O(k \log{n})$. 
Therefore, the total runtime is $O((n+k) \log{n})$.

\end{document}
