\documentclass{article}
\usepackage{scribe}
\usepackage{amsmath}
\usepackage{epsfig}
\renewcommand{\Pr}[1]{\textrm{\textup{Pr}}\left( #1 \right)}
\newcommand{\E}{\mbox{$\mathsf E$}}
\begin{document}


%%%%%%%%%%%%%%%%%%%%%%%%%%%%%%%%%%%%%%%%%%%%%%%%%%%%%%%%%%%%%%%%%%%%%%
% Lecture command takes 4 arguments:
%               ordinal number of the lecture
%               date of the lecture
%               lecturer
%               scribe---that is you.
%
% Fill out the next line with this information !!!!
\lecture{25}{11/13/2006}{David Karger}{Chris Wilkens}

%%%%%%%%%%%%%%%%%%%%%%%%%%%%%%%%%%%%%%%%%%%%%%%%%%%%%%%%%%%%%%%%%%%%%

%%%%%%%%%%%%%%%%%%%%%%%%%%%%%%%%%%%%%%%%%%%%%%%%%%%%%%%%%%%%%%%%%%%%%%
% Your notes start here!
%%%%%%%%%%%%%%%%%%%%%%%%%%%%%%%%%%%%%%%%%%%%%%%%%%%%%%%%%%%%%%%%%%%%%%
%
% For theorems, lemmas, definitions, remarks, etc. use commands
% {\theorem{...}}, {\lemma{...}}, {\definition{...}}, etc.
% For proofs, use \begin{proof} ... \end{proof}
%
% For postscript figures (.ps) use the following block:
%
% \begin{figure}[h]
% \begin{center}
% \mbox{\psfig{figure=notes-nn-fig-mm.ps}}
% \caption{A very nice picture.}
% \label{fig:picture}
% \end{center}
% \end{figure}
%

% For encapsulated postscript figures (.eps) use the following block:
%  (also change documentstyle line )
% \begin{figure}[h]
% \begin{center}
% \mbox{\epsfbox{notes-nn-fig-mm.eps}}
% \caption{A very nice picture.}
% \label{fig:picture}
% \end{center}
% \end{figure}
%


%%%%%%%%%%%%%%%%%%%%%%%%%%%%%%%%%%%%%%%%%%%%%%%%%%%%%%%%%%%%%%%%%%%%%%


\begin{center}
{\large \bf Randomized Rounding}
\end{center}

Last time, we began discussing an approximation of MAX-SAT. Today, we continue the derivation and discuss other randomized algorithms. Note that the scribe notes start at the beginning of the derivation and are essentially taken from a wonderful set of notes written in 2004 by Paul Valiant.

\section{MAX-SAT}

Consider a Boolean formula in conjunctive normal form (CNF), namely a formula expressed as the AND of a set of clauses, each of which is the OR of a set of literals, each of which is either a variable or the negation of a variable.

The satisfiability problem (SAT) is to determine whether or not a variable assignment exists that makes the formula true.  This is one of the canonical NP-complete problems.

The \emph{maximum satisfiability} problem (MAX-SAT) is to find a variable assignment that maximizes the number of true clauses.

One common variant of satisfiability is $k$SAT, in which each clause is restricted to be the OR of $k$ literals, for fixed $k$.  There is a straightforward polynomial time algorithm for 2SAT, but 3SAT is NP-complete.

We may similarly define MAX-$k$SAT to be MAX-SAT restricted to those formulas containing only clauses of size $k$.  It turns out, however, that even MAX-2SAT is NP complete.  We consider here, the approximation algorithm for MAX-3SAT consisting of making each variable true or false at random.

We have 
\begin{align}
\E[\textrm{number of clauses satisfied}]&=&\sum_{\textrm{clause\enspace} c}\Pr{c\textrm{\enspace is satisfied}}\\
&=&7/8 (\textrm{number of clauses}),
\end{align}
implying that we have an $8/7$-approximation algorithm.

Note that a slight generalization of the above result implies a $\frac{1}{1-2^{-k}}$-approximation algorithm for MAX-$k$SAT, which implies a 2-approximation for MAX-SAT.

We also note that it has been shown that the $8/7$ approximation ratio for MAX-3SAT is optimal unless P=NP.

\section{LP relaxation of MAX-SAT}

Consider the following integer linear program (ILP).  Let binary variable $z_j$ be 1 iff clause $c_j$ is satisfied.  Let binary variable $y_i$ be 1 iff variable $i$ is true.  Let $c_j^+$ denote the set of variables appearing non-negated in clause $c_j$, and let $c_j^-$ denote those variables that are negated.  The problem is as follows:
$$\max\sum_j z_j \textrm{\enspace such that\enspace} \forall j, \sum_{i\in c_j^+}y_i+\sum_{i\in c_j^-}(1-y_i)\geq z_j, 0\leq z_j\leq 1, 0\leq y_i\leq 1.$$

`It is fairly clear that this expresses the MAX-SAT problem if the variables are restricted to be integral.  We consider the LP that results from relaxing this restriction.

We note that the solution we get in the relaxed problem may have fractional values for the variables $y_i$.  Since these variables are meant to represent whether the variables $x_i$ in the Boolean formula are true or not, we may have difficulty interpreting the LP solution as a variable assignment.

We introduce the technique of \emph{randomized rounding}, a technique applicable to ILPs in general.

Given possibly fractional solutions $y_i$, randomized rounding sets $x_i=1$ with probability $y_i$, and $x_i=0$ with probability $1-y_i$.

We note that a trivial consequence of this is that $$\E[x_i]=y_i.$$

We now work to estimate the probability that a given clause is satisfied.  From the constraints on the LP, we know that the expected number of satisfied variables in a clause $c_j$ is at least $z_j$.  Define a sequence $\beta_k$ by $$\beta_k=1-(1-\frac{1}{k})^k=(1,\frac{3}{4},\sim .704,...\rightarrow 1-\frac{1}{e}\approx .632).$$

\begin{claim}
Given a clause $c_j$ of $k$ variables, the probability that it is satisfied is at least $\beta_kz_j$.
\end{claim}

\begin{proof}
To simplify the argument, we assume that all literals are positive (non-negated).  We note that the probability that $c_j$ is satisfied is the complement of the probability that none of its variables is true, which is $$\prod_{i\in c_j}(1-y_i).$$

We note that the $y_i$s are constrained by $$\sum_{i\in c_j} y_i\geq z_j.$$  Thus appealing to inequality tricks (Jensen's inequality and the fact that the logarithm function is convex) we conclude that this probability is minimized when $$y_i=z_j/k.$$
Thus we can lower bound the probability of $c_j$ being satisfied by $$1-(1-\frac{z_j}{k})^k\geq\beta_kz_j,$$ as desired.
\end{proof}

We now observe that this claim implies that this algorithm is a $\frac{1}{1-1/e}=\frac{e}{e-1}$-approximation: since the sequence $\beta_k$ is always at least $1-\frac{1}{e}$, the expected number of satisfied clauses is at least $(1-\frac{1}{e})\sum z_j$, which is thus within $(1-\frac{1}{e})$ factor of optimal since the solution to the relaxed problem $\sum z_j$ is at least as good as the solution to the ILP.

\section{Combined algorithm}
We introduce a third algorithm to approximate MAX-SAT that consists of running the previous two algorithms and picking the solution that maximizes the number of satisfied clauses.  While randomized rounding provides a solution at least half as good as OPT, and LP relaxation provides a solution within $1-\frac{1}{e}$ factor, the combined algorithm will provide a solution within $3/4$ of optimal.

We note that the above algorithm performs at least as well as the algorithm where we pick randomly between the two choices suggested by randomized rounding and LP relaxation.  Letting $k_j$ denote the number of variables in clause $j$, we have that the expected number of variables satisfied in the random solution is at least $$\frac{1}{2}\left[\sum_j(1-2^{-k_j})z_j+\sum_j(\beta_{k_j})z_j\right]=\sum_j \frac{1}{2}(1+\beta_{k_j}-2^{-k_j})\geq \frac{3}{4}\sum_j z_j.$$
Thus this is a $4/3$-approximation, as desired.

\section{Set Cover}
As before, in the \emph{set cover} problem we are given sets $S_i$ containing elements in $\{1...n\}$ and asked to find the minimum number of sets such that their union ``covers'' all the elements.  We have already seen a greedy algorithm that produces an $\textrm{O}(\log n)$-approximation.  We present an LP relaxation algorithm that also provides an $\textrm{O}(\log n)$-approximation.

Consider the following ILP.  Let $y_i=1$ iff $S_i$ is in the cover.  Find
\begin{align}
\min\sum{y_i}, \textrm{\enspace such that}\\
\sum_{S_i\ni j}y_i\geq 1\\
0\leq y_i\leq 1.
\end{align}

Observe that when $y_i$ is restricted to be integral this captures the set cover problem.  Now consider the result of optimizing the LP relaxation.  If we try the above randomized rounding technique of putting set $S_i$ in our cover with probability $y_i$, then we end up with the expected number of sets covering any point $j$ to be 1.  However, the set cover problem requires that \emph{all} sets be covered, which is a considerably stronger statement.  Thus regular randomized rounding might fail to give a solution to set cover.

We remedy this problem by modifying the randomized rounding scheme as follows: for some fixed $\alpha\geq 1$, choose set $S_i$ with probability $\alpha y_i$.  Note that the expected number of sets chosen is now $\alpha\sum y_i\leq \alpha |\textrm{OPT}|$, so that we will have an $\alpha$-approximation if the sets actually do cover $1...n$.

In order to show that the chosen sets do provide a cover with high probability, we introduce the \emph{Chernoff bound}, another fundamental technique in the probabilistic method.

\begin{theorem}[Chernoff Bound]
Let $X=X_1+...+X_n$ be a sum of indicator (0-1) variables that are independent.  Denote $\E[X]$ by $\mu$.  Then $$\Pr{X\leq(1-\epsilon)\mu}\leq e^{-\epsilon^2\mu/2}.$$
And if $\epsilon\leq 2e-1$, then $$\Pr{X\geq(1+\epsilon)\mu}\leq e^{-\epsilon^2\mu/4},$$
while if $\epsilon>2e-1$, then $$\Pr{X\geq(1+\epsilon)\mu}\leq e^{-(1+\epsilon)\mu}.$$
\end{theorem}
To apply this theorem, we note that the number of sets covering an element $j$ is the sum of independently random indicator variables, with mean $\alpha$.  We need to bound the probability that this random variable is 0.  We apply the Chernoff bound with $\epsilon=1$, which states that the probability that no set covers $j$ is at most $$e^{-\epsilon^2\mu/2}=e^{-\alpha/2}.$$
Applying the union bound, we have that the probability that \emph{any} element is uncovered is at most $$ne^{-\alpha/2}.$$  Setting $\alpha=3\log_2 n$, this probability becomes at most $1/\sqrt{n}\ll 1$.  Thus we have a Monte Carlo $\textrm{O}(\log n)$ approximation, as desired.

\begin{center}
{\large \bf Online Algorithms}
\end{center}

This section taken mostly from John Provine's 2004 scribe notes. Parts of his notes were  adapted from Kevin Zatloukal's scribe notes from 2003.

\section{Introduction to Online Algorithms}

All of the algorithms we have studied so far in this course have
taken their entire input up front and used it to compute an
answer. In this section, we will study algorithms that are given
the input one piece at a time and are forced to make a decision
solely based on the input they have read so far. The goal is for
the quality of those decisions to be close to the quality we would
achieve if we were given the entire input up front. We call
algorithms that get the input one piece at a time \emph{online}
algorithms and those that get the input all at once \emph{offline}
algorithms.

\subsection{Competitive analysis}

In an online problem the input is given as a sequence
$\sigma=\sigma_1\sigma_2\cdots\sigma_k$ of \emph{requests} or
\emph{events}. Each request $\sigma_i$ must be processed
immediately and incurs some cost. Let us denote the total cost of
an algorithm $A$ on input $\sigma$ by $C_A(\sigma)$ and the
optimal cost for this particular input by
$C_{\mathrm{MIN}}(\sigma)$. As mentioned above, our focus will be
on finding online algorithms for which the total cost can be
guaranteed to be relatively close to the optimal cost. The
following definition makes this notion precise.

\begin{definition}\emph{
  An (deterministic) online algorithm $A$ has \emph{competitive ratio} $k$ (or
  $A$ is \emph{$k$-competitive}) if for all inputs $\sigma$,
  $$C_A(\sigma)\leq kC_{\mathrm{MIN}}(\sigma)+O(1).$$ The smallest such $k$ is often referred to as the \emph{regret ratio}.}
\end{definition}

This defines the competitive ratio in a worst-case sense, as is
usual for deterministic algorithms. The constant term allows for
one-time costs related to differences in the starting positions
of the online and optimal algorithms.

For randomized algorithms, we will consider the expected cost of the algorithm.

\begin{definition}\emph{
  An randomized online algorithm $A$ has \emph{competitive ratio} $k$ if for all inputs $\sigma$,
  $$\mathrm{E}[C_A(\sigma)]\leq \alpha
  C_{\mathrm{MIN}}(\sigma).$$}
\end{definition}

The analysis of online algorithms is similar to that of
approximation algorithms in that we are comparing the quality of
our solution to that of the (possibly unknown) optimal algorithm.
As with approximation algorithms, our focus is on the quality of
the solution rather than on running time or space (beyond basic
polynomiality). Furthermore, in both cases, we are comparing our
algorithm against an optimal algorithm that can cheat. In
approximation algorithms, the optimal algorithm may not run in
polynomial time. In online algorithms, the optimal algorithm can
see the future because it is given all of the input up front.

In todays lecture, we examine the canonical toy competitiveness problem,
the \emph{ski rental problem}.

\section{The Ski Rental Problem}

After finals week, suppose that you head to a ski resort. You have
the entire vacation as well as the Independent Activities Period
to ski. Unfortunately, you know from past experience that, at some
point, the fun will come to a premature end when fate steps in and
breaks your leg. On each day until then, you have to make an
important decision: should you rent ski equipment for 1 dollar or
buy your own for $T$ dollars? If you keep renting long enough, you
will eventually find that you have spent more than $T$ dollars, so
it would have been cheaper to buy your own equipment at the
beginning. However, if you buy your own, then you might break your
leg that very day, wasting $T-1$ dollars.

One idea would be to always buy on the first day. However, if you
break your leg that day, then you spent $T$ dollars while the
optimum algorithm would have rented and spent only 1 dollar, so
this algorithm is only $T$-competitive. A better idea is to rent
for $T$ days and then buy on day $T+1$. To analyze this algorithm,
suppose that you break your leg on day $d$. If $d\leq T$, then we
always rented, which was the optimal decision. If $d>T$, then we
will pay $2T$. The optimal decision would have been to buy on the
first day, which would cost $T$ dollars. But we only spent twice
that, so this algorithm is 2-competitive.




\end{document}
