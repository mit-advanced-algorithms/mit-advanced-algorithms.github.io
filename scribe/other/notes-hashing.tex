\documentclass{article}
\usepackage{scribe}
\usepackage{epsfig}
\renewcommand{\Pr}[1]{\textnormal{Pr}\left[ #1 \right]}
\newcommand{\E}[1]{\textnormal{E}\left[ #1 \right]}
\usepackage{amssymb}
\begin{document}

\newcommand{\mod}{\hbox{mod }}

\title{Universal Hashing and Perfect Hashing}
\date{September 22, 2006}
\author{Lecturer: David Karger\\ Scribes: Ali Mohammad(2005) Kuen-Bang Hou(2006 revised)}

\section{Summary}

We all do it, but many of us have never analyzed it.  We have $s$ items
and we want to associate them to integers $1\cdots m$.
We could store them in an array of size $n$ where $m \gg n$, whence we could perform
insert/delete/query operations in constant time (much faster than, say,
a tree data structure) because of indirect addressing.  Hash tables don't
allow you to do predecessor or successor very easily.  The real drawback,
though, is space.  Hashing algorithms really are just about saving space.

\section{Assumption}

We want to map $s$ items (in fact, they are treated as integers) in the range $M = \left\{ 1 \cdots m \right\}$ into an array of size $n$.
Assume that the machine worksize is $\log m$ so that the machine can perform arithmetic operations on items
in O(1).\footnote{$s$ is usually assumed to be $n$ in the discussion about 2-universal Hashing part}

\section{2-universal Hashing}

We will develop 2-universal hashing, first introduced by Carter and Wegman in the 80's.

Goal: Store it in an array of size $O(n)$.

Method: Use a hash function that maps $\{1 \cdots m\}$ to $\{1 \cdots n\}$.
Store item with key $k$ in array position $f(k)$.

Problem: Collisions.

First Fix: Linked list from each array position (buckets).

Problem: We've lost $O(1)$ lookups.

Goal: Guarantee few items per bucket and therefore few collisions.

Problem: There's no \underline{one} kill-all function. We could always feed some bad input to a specific function.

Idea: Instead, use hash family, set of hash functions, such that at least one is good for any input set.

First Trial: A family of all functions. But it's of size $m^n$ and reguires description length of $n \log m$.
It would take lots of time doing calculating positions and lots of space storing function description.

Idea: This is sort of a load balancing problem -- maybe randomization can help with this.
Use a random function $f$ where $f(k)$ is random but well-defined.  Then,

\begin{eqnarray*}
  \E{\hbox{\# items in $k$'s bucket}} & = & \E{\sum c_{ik}}
  \\
  & = & \sum_n \E{ c_{ik} } = \sum \Pr{i\hbox{ collides with } k} \\
  & = & \sum_{i \not= k} 1/n = 1 - 1/n \\
  \Rightarrow \qquad \E{\hbox{time to find } k}  & = & O(1).
\end{eqnarray*}

More generally, if there are $s$ items to $n$ bins, $\E{\hbox{number of items colliding with }i} < \frac{s}{n}$.

But there's a (avoidable) problem that a purely random function takes $O(m)$ space, which brings us back to where we started!

Idea: Use a small family of functions to be chosen randomly.

Solution: Use pseudo-random number generators, which are ``random enough for a given application.''  The key for us is to reduce the collisions of pairs, so we will use a ``pairwise-independent'' hash function.  That is, given any \underline{one} output of the hash function, you have no information about any other (weakening the independence condition).

\begin{definition}[Pairwise Independence]

Random variables $B_1, \cdots, B_n$ are said to be \emph{pairwise independent} if $B_i$ and $B_j$ are independent $\forall\ i, j$.
\end{definition}

Here is an algorithm for generating pairwise-independent random variables in
${\mathbb Z}_p$.  First, pick a random prime $p$ and pick random values $a, b
\in {\mathbb Z}_p$.  Then, set $r_i = ai + b \mod p$.  Observe that
knowing $r_i$ doesn't tell you anything about $r_j$ for $j \not= i$.  However,
knowing any two $r_i$ and $r_j$ will reveal $a$ and $b$, and therefore the
remaining $r_k$.  Here $a$ and $b$ correspond to the seed and the $r_i$s
correspond to a sequence of random variables based on the seed.

\begin{claim}
$\forall i,j < p$, $r_i$ and $r_j$ are pairwise independent and uniformly distributed.
\end{claim}

\begin{proof}
We are arguing that $r_i = ai + b$ and $r_j = aj + b$ are pairwise independent.  Observe that:
\begin{eqnarray*}
  \Pr{ai+b = x,\ aj+b = y} & = &
  \Pr{
  \begin{array}{ccccc}
    a i & + & b & = & x \\
    a j & + & b & = & y 
  \end{array}
  }
  = 1/p^2 = 1/p \cdot 1/p \\
  & = & \Pr{ai+b = x}\Pr{aj+b = y},
\end{eqnarray*}
as desired. \hfill
\end{proof}

Now how do we use this knowledge for hashing?  Pick a prime $p > m$, pick $a,b$ randomly from ${\mathbb Z}_p$, and set $h(k) = [(ak+b)\ \mod p]\ \mod n$.

\begin{claim}
  Items are distributed ``almost'' uniformly.  That is,
$$\Pr{h(k) = x} \approx 1/n.$$
\end{claim}

This is because we know that $(ak+b)\mod p$ is absolutely uniform; when we mod it down to $n$, $p$ will cover the $n$ slots many times over plus some negligible remainder, which we will consequently neglect.

\begin{claim}
  The placements in the array are pairwise independent.
\end{claim}

Clearly the extra $\mod n$ doesn't create dependence.  $r_i$ and $r_j$ are independent implies that $f(r_i)$ and $f(r_j)$ are independent $\forall f$.

\begin{claim}
  ${\cal E}\left[\sum_{i\not=j} c_{ij}\right]$ is unchanged.  $c_{ij}$ depends only on the buckets of $i$ and $j$.
\end{claim}

\underline{Summary:} By choosing random $a,b$ and using $h(k) = [(ak+b)\ \mod p]\ \mod n$, get $1-1/n$ collisions in expectation.  Therefore, insert and lookup cost $O(1)$ in expectation.

We don't know that there will only be two or fewer items in each bucket.  In fact, in the worst case, $\sqrt{n}$ items can be placed in one bucket!  (This is possible if the pairwise-independent family is chosen poorly.)  It is also important that the adversary is unaware of what function was chosen (i.e., what seed was chosen).

The cost of defining our pseudo-random function is now just two integers of size $O(m)$ (a total of two machine words).

\section{2-level Hashing}

2-universal hashing is nice in expectation, but what about the worst-case?

Let's try to define a hash function with \underline{no} collisions!  To simplify things, we are not going to worry about a dynamic scenario where there is insertion and deletion.  Instead, we are going to consider a fixed set of $n$ keys.  Now, it is easy to spread them out, but we will have to work to make lookup fast.  This seems really tricky because even a fully random function has collisions (recall the birthday paradox).

\underline{First Try:} Use more space.

How many items can we map into $n$ spaces with collisions?  The expected number of total collisions is:
$$\E{\mathop{\sum_{i,j}}_{i\not=j} c_{ij}} = \mathop{\sum_{i,j}}_{i\not=j}\E{c_{ij}} = \mathop{\sum_{i,j}}_{i\not=j} {1 \over n} = {s \choose 2}{1 \over n} < {s^2 \over 2n}$$
In particular, if $s < \sqrt{n}$, then the expected number of collisions is less than $1/2$. This is nice, but we want a guarantee of no collisions.

Recall Markov's Inequality (one of the few probabilistic gems we will use in the course):
\begin{lemma}[Markov's Inequality]
If $X$ is nonnegative, then
$\Pr{X \geq a} \leq \E{X}/a.$
\end{lemma}

Consequently, $$\Pr{\sum c_{ij} \geq 1} < \E{\sum c_{ij}}.$$
Observe that we are again using pairwise independence, by the definition of the $c_ij$'s.  Thus, the 2-universal hashing results still apply.

Now, if you fail, try again.  That is, one can amplify the probability of
success by repetition.  The expected number of tries before I succeed is
only 2!  So, in exepcted linear time, I can find a perfect hash function into
quadratic space.  This is nice, but it still isn't perfect.

Maybe by using a more random function (using more independence) we can do better?  Well, no.  The birthday paradox comes into play again.  So we can't do better by using more randomness.

\underline{Another trick:}  First, use a 2-universal hash function to divide $n$ items into $n$ buckets.  Then, build a quadratic space to achieve a perfect hash table in each bucket.  How much space does this use?

\begin{eqnarray*}
\hbox{Total space} & = & \sum_b [\hbox{size of bucket } b] ^2 \\
s_b = \hbox{size of bucket } b & = & \sum_i s_{ib} \qquad\hbox{(event that $i$ hashes to bucket $b$)} \\
s_b^2 & = & \left( \sum_i s_{ib}\right)^2= \sum_{i,j} s_{ib}s_{jb} \\
\sum_b s_b^2 & = & \sum_{i,j,b} s_{ib}s_{jb} = \sum_{i,j}\sum_b s_{ib}s_{jb} = \sum_{i,j} c_{ij} \\
\E{\sum_{i,j}c_{ij}} & = & \sum_{i,n}1/n = n - 1 + n = 2n - 1 = O(n).
\end{eqnarray*}

Thus, the expected space for extra tables is $O(n)$.  If you do this carefully, the space used is $6n$. If you work a lot harder, you can actually trim it down to $n + o(n)$ with this simple technique!

If you want to be certain that you use at most $n$ space, simply use the technique of trying again and again until you succeed.  One small problem is that you need $m$ $a,b$ pairs -- about $m^2$ space.  This might be ok depending on the application.

One more thing: using more sophisticated techniques, one can achieve similar bounds in the dynamic case.
\end{document}
