\documentclass{article}
\usepackage{scribe}
\usepackage{epsfig}
\usepackage{amsfonts}
\renewcommand{\Pr}[1]{\textrm{\textup{Pr}}\left( #1 \right)}

\title{Voronoi Diagrams: Randomized Incremental Construction II}
\date{November 12, 2004}
\author{Lecturer: David Karger\\ Scribe: Charles Han}

\begin{document}

\section{Convex Hull}
The following algorithm provides a randomized incremental construction for convex hull: start with 3 points, 
then process the remaining points in random order, updating the convex hull each time.  Define the set $S_{i}$ 
to the first $i$ points processed, and define $conv(S_{i})$ to be the convex hull of $S_{i}$.  Throughout the algorithm, 
we want to maintain the vertices and edges of $conv(S_{i})$  We take some point $p_{0}$ that lies inside $conv(S_{i})$, 
and notice the following about the remaining points $p$: if the half-line $\overrightarrow{p_{0}p}$ cuts any segment 
of $conv(S_{i})$, then that segment will leave $conv(S_{i})$ if $p$ is added to the hull.  It is also useful to observe that 
for each non-hull point $p$, there is exactly one edge that is cut by $\overrightarrow{p_{0}p}$.  We also know that So, when 
adding some point $p_{i}$, we know that the convex hull changes if $\overrightarrow{p_{0}p}$ cuts a segment before reaching $p$.  
If we know that the hull changes, do the following: start at the cut segment, then walk clockwise, removing all absorbed segments 
until we are done; do the same going counterclockwise.  The total time of this operation is $O(1)$ per removed segment, 
plus $O(1)$ to add the two new segments connecting $p_{i}$ to the convex hull.

To be able to know in $O(1)$ time whether a segment was cut or not, we need to maintain a \emph{conflict list} for each segment 
of the current hull; the conflict list maintains pointers to all the vertices (that have not yet been added to the hull) that 
would cut the given segment if added.  Given a vertex $p$, we can easily tell if it cuts a given segment by looking for it in that 
segment's conflict list (in $O(1)$ time with hashing, for instance).  To build the conflict list for the two new segments after a 
hull update, we need $O(1)$ units of time for each vertex that had a cut segment absorbed by the new segment.

In order for a segment to be deleted, it must have been added at some point, so we can simply charge the deletion cost to the addition, 
for a real cost of at most 2 per point; it follows that the cost of deletion is $O(n)$.  The cost of updating the conflict lists turns out 
to be harder to analyze, since the cost depends on the expected number of points in the deleted conflict lists.  For this analysis, we turn 
to the method of \emph{backwards analysis}.

\section{Backwards Analysis}
In backwards analysis, we simply imagine running our algorithm backwards and count the resulting pointer updates.  In this case, 
we want to know how much work needs to be done whenever we \emph{remove} a point $p_{i}$ from $S_{i}$ (not $conv(S_{i})$!) to 
arrive at $S_{i-1}$.  In particular, we are interested in analyzing the total amount of work done on the conflict lists in our 
convex hull representation.  For the $i$th step of the convex hull [de]construction, consider the following:
\begin{itemize}
\item a given segment of $conv(S_{i})$ will not change unless it is removed from the convex hull.
\item if a segment of is removed from the $conv(S_{i})$, we must remove all of the pointers in its conflict list.
\item a segment will only be removed if one of its endpoints is $p_{i}$.
\item since each of the $i$ remaining vertices is equally to be removed, each segment in $conv(S_{i})$ has a $2/i$ probability of being removed.
\end{itemize}
Knowing these things, we can find the expected number of conflict pointers updated per step:
\begin{eqnarray*}
\mathbb{E}[\#~\mbox{of pointers updated}] &=& 
\sum_{e \in conv(S_{i})}\bigg(\mathbb{E}[\mbox{size of }e\mbox{'s conflict list}]\cdot\Pr{{e~\mbox{is removed}}}\bigg)\\
&=& \Pr{{e~\mbox{is removed}}} \cdot \sum_{e \in conv(S_{i})}\mathbb{E}[\mbox{size of }e\mbox{'s conflict list}]\\
&=& \Pr{{e~\mbox{is removed}}} \cdot ~ O(n)\\
&=& O(n/i)\\
\end{eqnarray*}

Finally, we can find the total expected amount of work spent maintaining conflict lists over the entire course of the algorithm:
$$\mbox{E}[\mbox{total expected work}] = \sum_{i=1}^{n} O(n/i) = O\left(n \sum_{i}^{n} (1/i)\right) = O(n \log{n})$$

\section{Linear Programming}
Now consider the following application of randomized incremental construction to linear 
programming.  Let $d$ represent the number of variables in our $n$-constraint linear program.
Recall that an LP can be represented geometrically as the problem of finding the optimal vertex 
(as measured by the objective function) in the intersection of $d$-dimensional half-spaces 
represented by the constraints.  Say that we build our polyhedron by adding our half-spaces in 
random order, and at each step maintaining the current optimum vertex $OPT$.  In order to accurately 
and efficiently maintain the current $OPT$, we rely on the following obervation: adding a new 
constraint will not improve our current $OPT$, but it may make it infeasible.  To state it a bit 
differently:

\claim{Let $OPT_i$ be the optimum vertex of the polyhedron after inserting $i$ constraints 
(half-spaces).  Now add the next constraint $c_{i+1}$; if $OPT_(i+1) \neq OPT_i$, then the new 
optimum vertex $OPT_{i+1}$ must be tight for the new constraint $c_{i+1}$}

\begin{proof}
We prove the above claim by contraposition.  Suppose that $c_{i+1}$ is not tight for $OPT_{i+1}$.  
Then removing $c_{i+1}$ would have no effect on the optimum, implying that $OPT_i = OPT_{i+1}$.  This 
contradicts the condition that $OPT_i \neq OPT_{i+1}$; therefore, the claim holds.
\end{proof}

We already know the set $\mathcal{C}$ of constraints for which the previous $OPT$ was tight, so 
we can reduce our search for the new $OPT$ to just those contstraints.
We also know that the new $OPT$ is tight for our new constraint, so it sufficient to optimize 
within the hyperplane defined by that constraint.  In this new optimization problem, we intersect
each constraint $\mathcal{C}_j \in \mathcal{C}$ with the new tight constraint, and we also project 
the objective function onto the new constraint.  This gives us a new $(d-1)$-dimension optimization 
problem, which we can recursively solve in the same manner to find the new $OPT$.

In the worst-case scenario, we would need to recurse for every single new constraint added.  
This behavior is modelled by the recurrence:
$$T(n,d) = T(1,d-1)+T(2,d-1)+...+T(n,d-1) \approx nT(n,d-1) \approx n^{d}$$

Since this is a randomized construction, we can hope to do better in the expected case.  We only 
need to recurse in cases where $OPT_i$ is infeasible for constraint $c_{i+1}$ (otherwise, 
$OPT_i = OPT_{i+1}$).  At step $i$ in the construction, consider that only $d$ constraints 
ever specify $OPT_i$.  In a backwards analysis, consider removing a random constraint from 
the polyhedron at step $i$; we will remove a tight constraint only with probability $d/i$, 
knowing this, we get a much better-looking recurrence for our algorithm: 
$$T(n,d) = T(n-1,d) + O(n) + \frac{d}{n}(T(n-1,d-1))$$

This gives an expected running time of $O(d!n)$.  This is a strongly polynomial LP algorithm, which 
is in fact linear in the number of constraints and FPT with respect to $d$.

\end{document}
