\documentclass{article}
\usepackage{scribe}
\usepackage{epsfig}
\usepackage{amsfonts}
\usepackage{hyperref}
\renewcommand{\Pr}[1]{\textrm{\textup{Pr}}\left( #1 \right)}

\title{Treewidth and Online Algorithms\\6.854 Scribe Notes \#25}
\date{October 29, 2004}
\author{Lecturer: David Karger\\ Scribe: John Provine}

\begin{document}

%%%%%%%%%%%%%%%%%%%%%%%%%%%%%%%%%%%%%%%%%%%%%%%%%%%%%%%%%%%%%%%%%%%%%%
% Your notes start here!
%%%%%%%%%%%%%%%%%%%%%%%%%%%%%%%%%%%%%%%%%%%%%%%%%%%%%%%%%%%%%%%%%%%%%%
%
% For theorems, lemmas, definitions, remarks, etc. use commands
% {\theorem{...}}, {\lemma{...}}, {\definition{...}}, etc.
% For proofs, use \begin{proof} ... \end{proof}
%
% For postscript figures (.ps) use the following block:
%
% \begin{figure}[h]
% \begin{center}
% \mbox{\psfig{figure=notes-nn-fig-mm.ps}}
% \caption{A very nice picture.}
% \label{fig:picture}
% \end{center}
% \end{figure}
%

% For encapsulated postscript figures (.eps) use the following block:
%  (also change documentstyle line )
% \begin{figure}[h]
% \begin{center}
% \mbox{\epsfbox{notes-nn-fig-mm.eps}}
% \caption{A very nice picture.}
% \label{fig:picture}
% \end{center}
% \end{figure}
%


%%%%%%%%%%%%%%%%%%%%%%%%%%%%%%%%%%%%%%%%%%%%%%%%%%%%%%%%%%%%%%%%%%%%%%


In this lecture we continue the discussion of treewidth and begin
our study of online algorithms.

\section{Treewidth}

\subsection{Review}

The treewidth of a graph measures roughly ``how close'' a graph is
to a tree. The definition of treewidth follows from the definition
of a tree decomposition.

\textbf{Definition}:
A \emph{tree decomposition} of a graph $G(V,E)$ is a pair
$(T,\chi)$ where $T=(I,F)$ is a tree and $\chi=\{\chi_i|i\in I\}$
is a family of subsets of $V$ such that (a) $\bigcup_{i\in
I}\chi_i=V$, (b) for each edge $e=\{u,v\}\in E$, there exists
$i\in I$ such that both $u$ and $v$ belong to $\chi_i$, and (c)
for all $v\in V$, there is a set of nodes $\{i\in I|v\in\chi_i\}$
that form a connected subtree of $T$. The \emph{width} of a tree
decomposition $(T,\chi)$ is given by $\max_{i\in I}(|\chi_i|-1)$.

The \emph{treewidth} of a graph $G$ is equal to the minimum width over
all of its tree decompositions.

Treewidth is often characterized in terms of elimination
orderings. An \emph{elimination ordering} of a graph is simply
an ordering of its vertices. Given an elimination ordering of a
graph $G$, one can construct a new graph by iteratively removing
the vertices $v$ in order and adding an edge between each of $v$'s
neighbors. The induced treewidth of this elimination ordering is
the maximum neighborhood size in this elimination process. The
treewidth of $G$ is then the minimum induced treewidth over all
possible elimination orderings. For example, the treewidth of a
tree is 1, and the treewidth of a cycle is 2 (each time you remove
a vertex, you connect its two neighbors to form a smaller cycle).

In general, the problem of computing the treewidth of a graph is
NP-complete. Therefore, finding the optimal elimination ordering is NP-hard as well.
However, the problem is fixed-parameter tractable
with respect to treewidth itself.

\subsection{Fixed-parameter tractability of SAT}

In this section, we will show that the problem of
\emph{satisfiability} (SAT) is fixed-parameter tractable with
respect to treewidth. Clearly, we need to define the graph that
contains our treewidth parameter.

\textbf{Definition}:
Given a CNF formula $F$ (i.e. a conjunction of clauses), the
\emph{primal graph} of $F$, $G(F)$, is the graph whose vertices
are the variables of $F$, where two variables are joined by an
edge if they appear in the same clause.

Suppose that the primal graph $G(F)$ has treewidth $k$. The proof
below provides an algorithm that, given an optimal elimination
ordering $x_1,x_2,\ldots,x_n$ of the vertices of $G(F)$, solves
SAT in time $O(nk\cdot 2^k)$. Though we have restricted our
attention to SAT, a similar approach can be used for any
optimization problem that has an associated graph with local
dependency structure.

\textbf{Claim}:
SAT is fixed-parameter tractable with respect to treewidth.

\begin{proof}
Let $F$ be a CNF formula and $G$ its primal graph. As in the tree
decomposition procedure, eliminate the variables
$x_1,x_2,\ldots,x_n$ in that order to construct a new graph $G^*$.
When we eliminate $x_i$, we remove all edges incident to $x_i$ in
$G$, and add any non-existing edges between a pair of neighbors of
$x_i$ in $G^*$. It is clear that $G^*$ is the
primal graph for the new formula.

%Note that we can form a new CNF formula in which we combine all clauses containing $x_i$ into one by taking the disjunction of all the literals in those clauses, except those containing the variable $x_i$. 


Note that removing $x_i$ and adding edges between neighbors of $x_i$ is like combining all the clauses containing $x_i$ into one new clause that contains neighbors of $x_i$. 
For each such new clause, we construct a truth table for it. The truth table is a list of all satisfying assignments of variables in the clause which is a partial
satisfying assignment for all the original clauses. The truth table for the new (combined) clause is constructed as following. Let $C_1$,..,$C_m$ be the clauses that contain variable $x_i$. Let $A_j$ be the set of satisfying assignments for $C_j$, except we remove the assignment for variable $x_i$. Then, take the intersection of the $A_j$'s. The truth table for the new (combined) clause is the list of all assignments in this intersection set, \emph{i.e.}, $\bigcap_{j} A_j$. For instance, consider this example. Suppose $(x\vee y\vee z)$ and $(\neg x\vee\neg y\vee z)$ are the only
two clauses containing $x$, then on eliminating $x$, we form the
new clause $(y\vee z\vee\neg y)$. The truth table for the new
clause is derived from truth tables for the original clauses as
follows:

$$\begin{array}{cccccccccccccccc}
  ( & x & \vee & y & \vee & z & ) && ( & \neg x & \vee & \neg y
  & \vee & z & )\\\hline\\
  &1&&1&&1&&&&1&&1&&1\\
  &0&&1&&1&&&&0&&1&&1\\
  &1&&0&&1&&&&1&&0&&1\\
  &0&&0&&1&&&&0&&0&&1\\
  &1&&0&&0&&&&1&&0&&0\\
  &0&&1&&0&&&&0&&1&&0\\
  &1&&1&&0&&&&0&&0&&0\\
\end{array}\qquad\longrightarrow\qquad
\begin{array}{ccccccc}
  ( & y & \vee & z & \vee & \neg y & )\\\hline\\
  &1&&1&&-\\
  &0&&1&&-\\
  &1&&0&&-\\
  &0&&0&&-\\
\end{array}$$

Repeating this procedure for each variable eliminated, we will
eventually end up with a single truth table representing the
entire formula. If the table has at least one entry, then there is
an assignment which satisfies all the original clauses. In this
case, we return \emph{true}. Otherwise, $F$ is not satisfiable, so
we return \emph{false}.

Since $G$ has treewidth $k$, there are at most $k$ neighbors of
any vertex in an optimal elimination ordering, so we construct a
truth table with at most $k\cdot 2^k$ entries per vertex
eliminated ($k$ columns and $2^k$ rows). Since there are $n$ vertices, this yields a total
running time of $O(nk\cdot 2^k)$.
\end{proof}

\newpage

\section{Online Algorithms\footnote{Parts of this section are adapted from Kevin Zatloukal's scribe notes from 2003.}}

\subsection{Introduction}

All of the algorithms we have studied so far in this course have
taken their entire input up front and used it to compute an
answer. In this section, we will study algorithms that are given
the input one piece at a time and are forced to make a decision
solely based on the input they have read so far. The goal is for
the quality of those decisions to be close to the quality we would
achieve if we were given the entire input up front. We call
algorithms that get the input one piece at a time \emph{online}
algorithms and those that get the input all at once \emph{offline}
algorithms.

\subsubsection{Competitive analysis}

In an online problem the input is given as a sequence
$\sigma=\sigma_1\sigma_2\cdots\sigma_k$ of \emph{requests} or
\emph{events}. Each request $\sigma_i$ must be processed
immediately and incurs some cost. Let us denote the total cost of
an algorithm $A$ on input $\sigma$ by $C_A(\sigma)$ and the
optimal cost for this particular input by
$C_{\mathrm{MIN}}(\sigma)$. As mentioned above, our focus will be
on finding online algorithms for which the total cost can be
guaranteed to be relatively close to the optimal cost. The
following definition makes this notion precise.

\textbf{Definition}:
  A deterministic online algorithm $A$ has \emph{competitive ratio} $k$ (or
  $A$ is \emph{$k$-competitive}) if for all inputs $\sigma$,
  $$C_A(\sigma)\leq k\cdot C_{\mathrm{MIN}}(\sigma)+O(1).$$ The smallest such $k$ is often referred to as the \emph{regret ratio}.

This defines the competitive ratio in a worst-case sense, as is
usual for deterministic algorithms. For randomized algorithms, we
will consider the expected cost of the algorithm.

\textbf{Definition}:
  A randomized online algorithm $A$ has \emph{competitive ratio} $k$ if for all inputs $\sigma$,
  $$\mathbb{E}[C_A(\sigma)]\leq \alpha \cdot
  C_{\mathrm{MIN}}(\sigma).$$

The analysis of online algorithms is similar to that of
approximation algorithms in that we are comparing the quality of
our solution to that of the (possibly unknown) optimal algorithm.
As with approximation algorithms, our focus is on the quality of
the solution rather than on running time or space (beyond basic
polynomiality). Furthermore, in both cases, we are comparing our
algorithm against an optimal algorithm that can cheat. In
approximation algorithms, the optimal algorithm may not run in
polynomial time. In online algorithms, the optimal algorithm can
see the future because it is given all of the input up front.

In todays lecture, we examine two examples of online algorithms:
the \emph{ski rental problem} and \emph{(online) load balancing}.

\subsection{The Ski Rental Problem}

After finals week, suppose that you head to a ski resort. You have
the entire vacation as well as the Independent Activities Period
to ski. Unfortunately, you know from past experience that, at some
point, the fun will come to a premature end when fate steps in and
breaks your leg. On each day until then, you have to make an
important decision: should you rent ski equipment for 1 dollar or
buy your own for $T$ dollars? If you keep renting long enough, you
will eventually find that you have spent more than $T$ dollars, so
it would have been cheaper to buy your own equipment at the
beginning. However, if you buy your own, then you might break your
leg that very day, wasting $T-1$ dollars.

One idea would be to always buy on the first day. However, if you
break your leg that day, then you spent $T$ dollars while the
optimum algorithm would have rented and spent only 1 dollar, so
this algorithm is only $T$-competitive. A better idea is to rent
for $T$ days and then buy on day $T+1$. To analyze this algorithm,
suppose that you break your leg on day $d$. If $d\leq T$, then we
always rented, which was the optimal decision. If $d>T$, then we
will pay $2T$. The optimal decision would have been to buy on the
first day, which would cost $T$ dollars. But we only spent twice
that, so this algorithm is 2-competitive.

\subsection{Online Load Balancing}

In the online load balancing problem, the goal is to assign a
sequence of jobs to $m$ identical machines. The jobs arrive at
different times, and we must immediately choose the machine to
which we will assign it. Each job comes with a load, the amount of
computational resources it will consume. And all jobs are
permanent---they never terminate. As in the offline problem, the
goal is to minimize the maximum load on any machine.

Although we did not present the algorithm in this way originally,
we have already seen one example of an online algorithm for load
balancing: \emph{Graham's algorithm}. In this algorithm, when each
job arrives, assign it to the least loaded machine. Since this
algorithm does not require looking at any of the jobs yet to
arrive, it is an online algorithm. We proved before that, over any
sequence of jobs, the maximum load produced by our algorithm is at
most twice the maximum load of the optimal solution. This means
that the algorithm is 2-competitive.

In fact, the competitve ratio is exactly $2-\frac{1}{m}$, where
$m$ is the number of machines. The fact that greedy is not better
than $2-\frac{1}{m}$ is shown by a simple example which consists of
$m(m-1)$ unit size tasks followed by one task of size $m$.

\subsubsection{Improvements to Graham's algorithm}

There has been a surprisingly large amount of research towards
achieving a better competitive ratio for the online load balancing
problem. In 1992, Fiat et al. found an algorithm with competitive
ratio $2-\epsilon$ for a small constant
$\epsilon\approx\frac{1}{70}$. In 1994, Karger, Phillips and Torng
used an LP solver to achieve a competitive ratio of 1.945. In
1997, Albers found a 1.923-competitive algorithm.

It is worth noting that one may consider the case where the value
of the optimum cost is known. In this case, the lower bound is
provably $4/3$ (the competitive ratio is exactly $4/3$ for two
machines). The best known deterministic algorithm for this
instance, due to Azar and Regev, is 1.625-competitive.

See \cite{azar97line} for an overview of online load balancing.


\begin{thebibliography}{9}
\bibitem{azar97line} Azar, Y., : \href{http://www.cs.tau.ac.il/~azar/survey.ps}{\emph{Online Load Balancing}}
\end{thebibliography}

\end{document}
