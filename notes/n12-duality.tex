\documentclass{article}
\usepackage{me}
\usepackage{amssymb}
\usepackage{amsfonts}
%\input{abbrevs}
\setlength{\parindent}{0pt}

\newcommand\vol{{\mbox{vol}}}

\title{Duality}
\author{David Karger}

\begin{document}

\section{Last Time}

\begin{itemize}
\item Introduced the definition of linear programs (LPs). 
\item Standard form LP: $\min c^Tx$, s.t., $Ax=b, x\geq 0$. 
\item Analyzed the structure of optimal solutions of such LP. 
\item Proved that there is always a basic feasible solution (BFS) and its bit complexity is not too large.

$\Rightarrow$ The decision problem $LP-Dec$: for a given $v$, does there exists $x$ s.t. $c^Tx\leq v, Ax=b, x\geq 0$, is in $NP$.
(If such $x$ exists, then the optimal BFS $x^*$ is a succinct certificate that can be checked in polynomial time.)
\item How about $LP-Dec\in co-NP$? Is there always a succinct certificate of {\em infeasibility} of an LP?
\end{itemize}

\section{Duality}

Checking feasibility of a given solution is easy.
\textbf{Motivating question:} How to certify optimality of a given solution?
\begin{itemize}
\item \textbf{Key notion:} {\em Duality} - strongest result in the theory of LP.
\item Enable one to give a (succinct) proof of optimality.
\item Recovers a lot of classic results: max-flow min-cut theorem, prices for min-cost flow, existence of a Nash equilibrium in every two-player zero-sum game,
  lots other stuff.
\end{itemize}

\textbf{Want:} Find a {\bf lower} bound on $v^*:=\min \set{c^Tx \mid Ax=b, x
  \ge 0}$.
\begin{itemize}
\item Standard approach: try adding up linear combinations of existing equations.

$\Rightarrow$ Try multiplying each constraint $A_i x = b_i$ by some $y_i$ and adding  all of them together.

$\Rightarrow$ We will always have that $yAx=yb$.
\item If find $y$ s.t. $y^TA=c^T$, then $y^Tb=y^TAx=c^Tx$. So, every feasible solution $x$ has the same value $v^*$!
\item But, in general, cannot hope that such $y$ exists - $c$ might not be in the span of $A^T$. 
\item Instead, let us focus on a looser goal:  find $y$ s.t. $y^TA \le c^T$. 

$\Rightarrow$ We can lower bound $v^*$ by noting that $y^T b= y^TAx \le c^Tx$, for {\em any} feasible $x$. (The inequality holds as $x\geq 0$.)

\item How about getting the {\em best} lower bound to be gotten this way?

\item Write an LP for that!

\item {\em Dual} LP: $\max b^Ty$ s.t. $A^Ty\leq c$. (Note that the dual LP is defined in a completely syntactic way - no thinking involved!)

\item Let $w^*:=\max \set{b^Ty \mid A^Ty \le c}$.
\item We just have shown {\em weak duality:} $w^*\leq v^*$. 
\end{itemize}


{\bf Important observation (and a sanity check):} The dual of the dual LP gives the primal LP. 

\begin{itemize}
\item Let the primal LP be $\min c^Tx$ s.t. $Ax=b, x\ge 0$.
\item Its dual LP is $\max b^Ty$ s.t. $A^Ty\leq c$.
\item Let us transform the latter LP into a standard form:
\[
\max b^Ty \text{ s.t. } A^Ty \le c
\]
$\Rightarrow$
\[
- \min -b^Ty \text{ s.t. } A^Ty + I s = c, s\geq 0
\]
$\Rightarrow$
\[
- \min (-b)^T(y^+-y^-) \text{ s.t. } A^Ty^+- A^Ty^-+ I s = c, s, y^+, y^-\geq 0
\]
$\Rightarrow$ In matrix form, this LP dual becomes
\begin{eqnarray*}
-\min
\left(
\begin{array}{ccc}
-b&b&0
\end{array}
\right)
\left(
\begin{array}{c}
y^+\\y^-\\s
\end{array}
\right) \text{ s.t. }
&&\left(\begin{array}{ccc}
A^T&-A^T&I
\end{array}
\right)
\left(\begin{array}{c}
y^+\\
y^-\\
s
\end{array}
\right) = c\\
&&y^+,y^-,s \ge 0\\
\end{eqnarray*}
\item The dual of the above LP is then:
\[
-\max c^T z \text{ s.t. }  \left(\begin{array}{ccc}
A^T&-A^T&I
\end{array}
\right)^T z
\le
\left(
\begin{array}{ccc}
-b&b&0
\end{array}
\right)^T\\
\]
$\Rightarrow$ 
\[
\min -c^T z \text{ s.t. }  A(-z)=b, z\leq 0
\]
$\Rightarrow$ Setting $x=-z$, we get
\[
\min c^T x \text{ s.t. }  Ax=b, x\geq 0,
\]
as desired.
\end{itemize}

{\bf A simple corollary of weak duality and the above observation:}  If $P$ is the primal LP and $D$ is the dual LP then either
\begin{itemize}
\item Both $P$ and $D$ are feasible and bounded. 
\item if $P$ is feasible, $D$ either infeasible or bounded.
\item If $P$ is feasible and unbounded (i.e., $v^*=-\infty$), $D$ infeasible.
\item (Similarly, if $D$ feasible and unbounded (i.e., $w^*=\infty$), $P$ is infeasible.)  
\item In fact, together with strong duality (see below), this implies that we can have only four possibilities: both $P$ and $D$ bounded and feasible, one of them unbounded and one of them infeasible,, both of them infeasible.
\end{itemize}



\section{Strong Duality}

Weak duality described below is just a tip of an iceberg.
{\bf Strong duality (key theorem of LP):} If $P$ and $D$ feasible then $v^*=w^*$. \\

{\bf ``Proof'' by picture:}
\begin{itemize}
\item Consider a ``dual'' problem $\min b^Ty$ s.t. $A^Ty \ge c$, which we obtain by flipping the sign of $y$ in our above dual problem formulation.
\item Suppose $b$ points straight up.
\item  Imagine that $y$ is the position of a ball that falls down subject to gravity and the constraints correspond to ``floors'' that limit how far the ball can fall.

$\Rightarrow$ Minimizing $b^T y$ subject to our constraints corresponds here exactly to minimizing the height of the ball given the limitations posed by the floor. 

$\Rightarrow$ Let $y^*$ be the minimum height position the ball settles in (according to physics). This is also the optimal solution for our dual LP.  

\item Why does the ball stay in position $y^*$? 
\item The forces exerted on the ball by the floors are canceling the ``gravity'' $-b$.

$\Rightarrow$ Force exerted by the floor $i$ has to be normal (i.e., perpendicular) to the surface of that floor. That is, it is equal to $A_i x_i$, for some scaling factor $x_i$, where $A_i$ is the $i$-th column of $A$. 

\item Equilibrium condition: $\sum_i A_i x_i = b$. That is, $Ax=b$. 
\item Also, the force exerted by the floors can be only pushing the ball away. So, $x_i\geq 0$, for each $i$. 

$\Rightarrow$ $x\geq 0$. 
\item In other words, $x$ is feasible (but not necessarily optimal) for the ``primal'' LP $\min c^x$ s.t. $Ax=b, x\geq 0$. 
\item Furthermore, physics tells us that the floors that do not touch the ball, i.e., all floors $i$ with $A_i^T y^{*} >c_i$, do not exert any force. So, $x_i=0$ for them. 

$\Rightarrow$ Compact way of writing that: $(c_i-A_i^Ty^{*})x_i=0$, for all $i$. (This is {\em complementary slackness} condition!)
\item Thus we have that $c^T x = \sum_i (A_i^T y^{*}) x_i = x^T A^T y^{*} = (Ax)^Ty^{*}=b^Ty^{*}$.
\item So, by weak duality, $x$ is indeed optimal and $v^*=c^Tx=b^Ty^*=w^*$. 
\end{itemize}

% Guest lecture, Justin Oct 18 2017 end (approximate)
% Karger Oct 20 2017 start

{\bf Formal proof:}
\begin{itemize}
\item Consider the optimum solution $y^*$ to the $\min b^Ty$ s.t. $A^Ty \ge c$. (The sign of $y$ is still flipped.)
\item Let us choose a subset $S$ of constraints that are tight, i.e., $A_i^Ty^*=c$, and linearly independent. (Recall that $A_i$ is the $i$-th column of $A$.)
\item Note that $|S|$ is at most $m$, where $m$ is the dimension of $y^*$.
\item Let $A_S$ and $c_S$ be the matrix $A$ and the vector $c$ after dropping the rows/entries corresponding to the constraints not in $S$.

$\Rightarrow$ The matrix $A_S^T$ is of full column rank and $A_S^T y^*=c_S$.
\item Also, we still have that $b^Ty^*=w^*=\min b^Ty$ s.t. $A_S^Ty\geq c_S$.
\item Thus, if we prove strong duality wrt to $A_S$, i.e., that there exists $x_S^*$ s.t. $A_Sx_S^*=b$, $x_S^*\geq 0$, and $c_S^T x_S^*=b^T y^*=w^*$,  then we recover strong duality for the original LP by just considering $x^*$ being $x_S^*$ with all the coordinates not in $S$ padded with zeros. Specifically, we will have that $c^Tx^*=c_S^Tx_S^*=w^*$, $Ax^*=A_Sx_S^*=b$ and $x^*\geq 0$, as desired.
\item So, let us drop the reference to $S$ and assume from now on wlog that $A^T$ is of full column rank and $A^Ty^*=c$. 
\end{itemize}

{\bf Claim:} There exists $x^*$ such that $Ax^*=b$.
\begin{itemize}
\item Suppose not?  Then, by the ``duality'' for linear equalities we considered earlier, tells us there exists a vector $z$ with $A^Tz=0$ but $b^Tz\ne 0$. 
\item Wlog assume $b^Tz<0$.  (Else, we just take $-z$.)
\item Consider a solution $y'=y^*+z$. 
\item It is feasible since $A^Ty'=A^Ty^*+A^Tz=A^Ty^*$.
\item Also, we have that $b^Ty'=b^T(y^*+z)=b^Ty^*+b^Tz<b^Ty^*=w^*$. Solution $y'$ is strictly better than the optimum. Contradiction!
\item (This formalizes our idea that if the forces are not in balance, the
  ball will fall further down.)
\end{itemize}

{\bf Claim:} We have that $b^Ty^*=c^Tx^*$.
\begin{itemize}
\item We know that $Ax^*=b$ and $A^Ty^*=c$. 
\item So $b^Ty^*=(Ax^*)^Ty^*=(A^Ty^*)^Tx^*=c^Tx^*$.
\end{itemize}

{\bf Claim:} It is the case that $x^* \ge 0$.

(Formalizes the intuition that barriers have to \emph{push} ball not pull it.)
\begin{itemize}
\item  Suppose not, i.e., some $x_i^* < 0$.
\item Let $c'=c+e_i$, where $e_i$ is a vector with all-zeros except a $1$ at coordinate $i$. (This corresponds to tightening the $i$-th constraint, i.e., moving it ``upwards''.)
\item Consider solution to $A^T y'=c'$. (Such a solution has to exist as $A^T$ has a full column rank.)
\item Clearly, $c' \ge c$, and thus $A^Ty'=c'\geq c$. 

$\Rightarrow$ $y'$ is feasible for the original constraints $A^Ty\ge c$. 
\item The value of the objective is $b^Ty'=(Ax^*)^Ty'=(A^Ty')^Tx^*=(c')^Tx^*$.
\begin{itemize}
\item We assumed $x_i^* < 0$, and \emph{increased} $c_i$.
\item So $b^Ty'=(c')^Tx^* < c^Tx^* =b^Ty^*$.
\item Again, got a feasible solution $y'$ with a better value than optimum. Contradiction!
\end{itemize}
\item {\em Intuition:} $x_i^*$ is telling us how much the value of the optimum will change if we ``tighten'' $i$-th constraint.

$\Rightarrow$ Making constraint tighter should increase opt (as it corresponds to a minimization problem).

$\Rightarrow$ $x_i^*$ should be thus non-negative.
\end{itemize}

{\bf Interesting corollary (of strong duality):} For LPs, finding a solution that is merely feasible is {\em algorithmically equivalent} to finding a solution that is optimal. 
\begin{itemize}
\item Given an LP optimizer, we can find a feasible solution by optimizing with an arbitrary cost vector.
\item Given an LP feasible  solution finding algorithm, can compute the optimal solution by combining primal and dual. Specifically, suffices to find a feasible solution to the following set of LP constraints:
  \[
  \{ Ax=b, x\geq 0, c^Tx=b^Ty,  A^Ty\leq c\},
  \]
  with $x$ and $y$ being the variables. 
\end{itemize}

\section{Rules for Taking Duals}

\begin{itemize}
\item So far we defined a dual LP only when the primal LP was in standard form. 
\item Since every LP can be converted to standard form, this was sufficient.
\item  Still, having to go through the standard form to figure out the dual is a bit painful. \item Thus, below, we provide a general rules for taking a dual of an arbitrary LP. 
\end{itemize}

\begin{itemize}
\item Consider a primal LP to be as follows:
\begin{eqnarray*}
v^* &= &\min c_1x_1+c_2x_2+c_3x_3\\
A_{11}x_1+A_{12}x_2+A_{13}x_3 &= &b_1\\
A_{21}x_1+A_{22}x_2+A_{23}x_3 &\ge &b_2\\
A_{31}x_1+A_{32}x_2+A_{33}x_3 &\le &b_3\\
x_1 &\ge &0\\
x_2 & \le &0\\
x_3 &&UIS,
\end{eqnarray*}
Here, we split the cost vector $c$ and the variables $x$ into the parts corresponding to different sign constraints. (UIS emphasizes the fact that the variables are unrestricted in sign.) Also, every submatrix of $A$ corresponds to a different combination of the sign of the variable and the constraint. 
\item The corresponding dual LP is:
\begin{eqnarray*}
w&=&\max y_1b_1+y_2b_2+y_3b_3 \\
A_{11}^Ty_1+A_{21}^Ty_2+A_{31}^Ty_3 &\le &c_1\\
A_{12}^Ty_1+A_{22}^Ty_2+A_{32}^Ty_3 &\ge &c_2\\
A_{13}^Ty_1+A_{23}^Ty_2+A_{33}^Ty_3 &= &c_3\\
y_1 &&UIS\\
y_2 &\ge &0\\
y_3 &\le &0
\end{eqnarray*}
\item In general, variable corresponds to constraint (and vice versa):\\
\begin{tabular}{|c|c|c|c|}
\hline&&&\\
PRIMAL & minimize & maximize & DUAL\\
\hline&&&\\
&$\ge b_i$ & $\ge 0$ & \\
constraints &$\le b_i$ & $\le 0$ & variables\\
&$= b_i$ & UIS &\\
\hline &&&\\
& $\ge 0$ & $\le c_j$ &\\
variables & $\le 0$ & $\ge c_j$ & constraints\\
& UIS & $= c_j$ &\\
\hline
\end{tabular}
\end{itemize}


{\em Some observations:}
\begin{itemize}
\item Adding primal constraints creates a new dual variable: more dual
  flexibility.
  \item The tighter the primal constraint, the looser the dual variable (and vice versa). In particular, equality primal constraints introduces a dual variable with an unrestricted sign.
\item If constraint is in ``natural'' direction opposing the minimization/maximization, dual variable is positive (and vice versa).
\end{itemize}

\section{Insights on Specific Problems via LP Duality}

\begin{itemize}
\item The derivation of the LP dual is completely automatic/''brain-dead''.
\item Still,  understanding what problem the dual LP captures can bring a lot of non-trivial insight into the problem corresponding to the primal LP. 
\item This is the real power of LP duality!
\item {\bf Nonetheless:} Knowing the actual optimal dual solution may be useless for finding the primal optimum solution! 

$\Rightarrow$ More formally: if your algorithm runs in time $T$ to find primal optimal solution {\em given} an optimal dual solution, it can be adapted to run in time $O(T)$ and find the primal optimal solution {\em without} having to know the optimal dual solution!

$\Rightarrow$ (See the problem set.)

\item Let us take a look at the strong duality ``in action'' on a couple of example problems we studied before.
\end{itemize}


\subsection{Shortest Paths Problem}

\begin{itemize}
\item Let us formulate shortest paths as a ``dual'' (i.e., maximization) LP:
\begin{eqnarray*}
w^*&=&\max d_t-d_s\\
\text{s.t.} && \\
d_j-d_i &\le &c_{e} \quad \quad\quad\quad\quad \text{  $\forall_{e=(i,j)}$}
\end{eqnarray*}
\item The constraint matrix $A$ has one row per each edge and one column per each vertex. 
\item The solution $d$ corresponds to embedding of vertices on the real line. 

$\Rightarrow$ All the constraints and the cost vector are ``shift-invariant'', i.e., shifting each $d_i$ by the same value does not change anything.

$\Rightarrow$ We can assume wlog $d_s=0$ and then $d_i$ is just the distance of vertex $i$ from vertex $s$ in this line embedding. 

\item Each constraints is just a triangle inequality imposed by edge $e$. 

\item Optimal solution sets $d_t$ to be the shortest $s$-$t$ path distance with all the constraints corresponding to the edges on this path being tight.

\item What is the dual LP? 

\item One variable $y_e$ per each edge constraint. As these are upperbounding constraints, the variables $y_e$ are non-negative (see the correspondence table above - note that our dual LP here is ``primal'' from the point of view of the table as our primal LP corresponds to maximization).
\item One constraint per each variable $d_i$.  Since variables were unbounded, the constraint is an equality constraints. (Again, see the table above.)
\item End result:
\begin{eqnarray*}
v^*&=&\min \sum_e c_e y_e\\
\text{ s.t.} & &\\
\sum_{j} \left(y_{ji}-y_{ij}\right) & = & 
{\begin{cases} -1 & \text{if $i=s$}\\
1 & \text{if $i=t$}\\
0 & \text{otherwise}
\end{cases}}\\
y &\ge &0\\
\end{eqnarray*}
\item So, the dual of the shortest path problem is the task of sending one unit of flow from $s$ to $t$ while minimizing the cost (and having no capacity constraints).
\item This makes a lot of sense and might seem not too profound but the key point is: we got this via purely syntactic manipulations!
\end{itemize}

\subsection{Maximum Flow Problem}

\begin{itemize}
\item Given our maximum flow instance, let us change it into a circulation problem by adding an edge from $t$ to $s$ and make it have infinite capacity, i.e., $u_{ts}=\infty$. 
\item Our goal then becomes to find a circulation that preserves all capacity constraints and maximizes the flow on the added ``back edge'' from $t$ to $s$. 
\item Our primal LP becomes:
\begin{eqnarray*}
w^*& = &\max x_{ts}\\
\text{ s.t.} & &\\
\sum_w x_{vw}-x_{wv} &= &0 \quad \quad\quad\quad\quad \text{  $\forall_{v}$}\\
x_{e} &\le &u_{e} \quad \quad\quad\quad\quad \text{$\forall_{e}$}\\\
x &\ge &0,
\end{eqnarray*}
with each variable $x_e$ encoding the flow on edge $e$.
\item In our dual problem we have two types of variables: $z_v$ corresponding to each flow conservation constraint; and $y_e$ corresponding to each capacity constraint.
\item One constraint per each variable $x_e$.  
\item The resulting dual LP is:
\begin{eqnarray*}
v^*&=&\min \sum_{e} u_e y_{e}\\
\text{ s.t.} & &\\
z_v-z_w+y_{vw} &\ge &0 \quad \quad\quad\quad\quad \text{  $\forall_{(v,w)\neq (t,s)}$}\\
z_t-z_s+y_{ts} &\ge &1\\
y &\ge &0
\end{eqnarray*}
\item Think of $y_{e}$ as ``lengths''
\item {\em Note:} Can wlog assume that $y_{ts}=0$ since otherwise dual infinite due to the fact that $u_{ts}=\infty$.
\item  So, we actually need to have that $z_t-z_s \ge 1$.
\item {\em Observe:} Variables $z_v$ can be viewed as ``distances'' and the corresponding constraints are just triangle inequality constraints for them wrt lengths given by $y$s. 
\item Since the occurrences of $z$s are shift-invariant can assume wlog that $z_s=0$ and thus $z_v$ is again the distance of $v$ from $s$ in the line embedding given by $z$. 
\item So, our goal is to choose edge lengths $y$ so as the sink $t$ is at distance of at least $1$ from $s$ wrt lengths $y$ while minimizing the total {\em volume} of the edges. 
\item {\em Side note:} This gives good justification for shortest augmenting path argument and blocking flows. This is {\em not} unusual: many \emph{primal-dual} algorithms leverage dual solution to indicate the best way to optimize the primal solution. 
\item How to solve our optimization task? 
\item One approach: find the minimum $s$-$t$ cut $S^*$, assign lengths $1$ to all the edges leaving it and $0$ to all the remaining ones. 
% Karger Oct 20 2017 end
% Karger Oct 23 2017 begin
$\Rightarrow$ Our volume will be equal to the capacity of this cut $u(S^*)$. So, by weak duality, we get $w^*\leq u(S^*)$, i.e.,  that the maximum flow value $w^*$ is at most the capacity of the minimum $s$-$t$ cut!
\item Would like use strong duality to also get the max flow min cut theorem, but to this end need to argue that $v^*=w^*\geq u(S^*)$, i.e., the dual ``volume'' problem does not have a better solution that the one given by the minimum $s$-$t$ cut. 
\end{itemize}


\section{Complementary Slackness}

Duality leads to another idea: {\em complementary slackness}:
\begin{itemize}
\item Given feasible solutions $x$ and $y$, we define their {\em duality gap} as 
\[
c^Tx-b^Ty.
\]
\item Duality gap measures ``how far off'' our dual objective is from the primal objective. That  is, how far from being optimal the solutions $x$ and $y$ are. 
\item {\em Note:} By weak duality we know that duality gap is always non-negative. Also, by strong duality, we know that when $x$ and $y$ is optimal the gap is zero. 
\item Let us rewrite our dual LP by changing inequalities into equalities by introducing additional variables:
\[
\max b^T y \text{ s.t. } A^Ty+s=c, s\geq 0.
\]
\item The variables $s$ are called {\em slack variables}.
\item Observe that
\[
c^Tx-b^Ty = (A^Ty+s)^Tx-b^Ty = y^TAx+s^Tx-y^Tb=s^Tx.
\]
\item So, $x$ and $y$ are optimal iff $c^Tx-b^Ty=s^Tx=\sum_i s_i x_i$ is equal to $0$.
\item Observe that $s,y\geq 0$. So, $\sum_i s_i x_i=0$ iff $s_ix_i=0$ for {\em each} $i$. 


$\Rightarrow$ {\em Complementary slackness} condition: Feasible primal and dual solutions $x$ and $y$ are optimal iff $x_is_i=0$, for each $i$. 

\item Note that $x_is_i=0$ implies that {\em at most one} of these variables can be non-zero. 

$\Rightarrow$ Either a primal variable $x_i$ is zero or its corresponding dual constraint is tight (i.e., $s_i=0$, which means that $A_i^Ty=c_i$).
\end{itemize}


\subsection{Apply CS to Max-flow Duality}

\begin{itemize}
\item To do that, consider the optimal dual solution $y^*$, $z^*$. 
\item Define a cut $S=\{v \mid z_v^* < 1\}$.

$\Rightarrow$ Need to have that $s\in S$ and $t\notin S$. So, $S$ is an $s$-$t$ cut. 
\item We now use complementary slackness:
\begin{itemize}
\item If $(v,w)$ \emph{leaves} $S$, then $y_{vw}^* \ge z_w^*-z_v^* > 0$

$\Rightarrow$ the corresponding primal constraints $x_{vw}^* \leq u_{vw}$ corresponding to the dual variable $y_{vw}^*$ has to be tight, i.e., $x_{vw}^*=u_{vw}$ for any primal optimal solution $x^*$. 
\item If $(v,w)$ \emph{enters} $S$, then $z_v^*>z_w^*$.  Also, we know $y_{vw}^*\ge 0$
 
 $\Rightarrow$ $z_v^*+y_{vw}^*>z_w^*$ i.e. the dual constraint corresponding to the primal variable $x_{vw}^*$ is {\em not} tight.
 
 $\Rightarrow$ $x_{vw}^*=0$. 
\item So, in other words, in the optimal solution $x^*$ all  edges leaving $S$ are saturated, all edges entering $S$ are empty. 

$\Rightarrow$ The value of the flow $w^*$ is equal to the capacity $u(S)$ of $S$, which is at least the capacity $u(S^*)$. 
\end{itemize}
\item We indeed recovered the max-flow min-cut theorem!
\end{itemize}

\subsection{Minimum Cost Circulation Problem}

\begin{itemize}
\item Can adapt our maximum flow formulation. Just need to add costs on edges (and make it be minimization instead of maximization problem).
\item The resulting primal LP:
\begin{eqnarray*}
v^*&=&\min \sum_e c_{e} x_{e}\\
\text{ s.t.} & &\\
\sum_w x_{vw}-x_{wv} &= &0 \quad \quad\quad\quad\quad \text{  $\forall_{v}$}\\
x_{e} &\le &u_{e} \quad \quad\quad\quad\quad \text{$\forall_{e}$}\\\
x &\ge &0,
\end{eqnarray*}
\item The dual LP almost the same as before, except our right-side constraints changed (as our primal cost vector changed) and changing the primal problem to be a minimization one flipped the sign constraints on variables $y$. 
\item The resulting dual LP:
\begin{eqnarray*}
w^*&=&\max \sum_{e} u_e y_{e}\\
\text{ s.t.} & &\\
z_v-z_w+y_{vw} &\le & c_{vw} \quad \quad\quad\quad\quad \text{  $\forall_{(v,w)}$}\\
y &\le &0
\end{eqnarray*}
\item Changing variables $p_v=-z_v$ and rearranging the constraints gives us:
\begin{eqnarray*}
w^*&=&\max \sum_{e} u_e y_{e}\\
\text{ s.t.} & &\\
y_{vw} &\le & c_{vw} +p_v -p_w \quad \quad\quad\quad\quad \text{  $\forall_{(v,w)}$}\\
y &\le &0
\end{eqnarray*}
\item But $c_{vw} +p_v -p_w $ is just the reduced cost $c^{(p)}_{vw}$ of the edge $(v,w)$!
\item {\em Note:} Optimum $\le 0$ since $y=0$ is a feasible solution. (Put differently: zero circulation is feasible.)
\item {\em Note:} $y\le 0$ says the objective function is the sum of
  the {\bf negative parts} of the reduced costs. (Positive ones get truncated to $0$.)
  \item Let $x^*$ and $y^*$ (and prices $p^*$) be the primal and dual optimal solutions. 
 \item Invoking complementary slackness tells us that:
\begin{itemize}
\item If $x_{vw}^*<u_{vw}$ then the corresponding dual variable $y_{vw}^*$ is equal to $0$. 

$\Rightarrow$ $c_{vw}^{(p^*)}\ge 0$. 
\item If  $c_{vw}^{(p^*)}<0$ then $x_{vw}^*=u_{vw}$.

\item  This means that all edges with negative reduced cost have to be saturated!
\item On the other hand, if $c_{vw}^{(p^*)} > 0$

$\Rightarrow$ the constraint on $y_{vw}^*$ is slack (as $y^*\leq 0$). 

$\Rightarrow$ The corresponding variable $x_{vw}^*=0$. 
\item So, all the edges with positive reduced cost need to carry no flow!
\end{itemize}  
\end{itemize}

% approx 50 minutes into lecture Oct 23 2017

\end{document}
