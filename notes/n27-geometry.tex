\documentclass{article}
\usepackage{me}
\setlength{\parindent}{0pt}

\newcommand{\vol}{\mbox{vol}}

\title{Geometry\\ 6.854 Notes \#27}
\author{David Karger}

\begin{document}

\section{Geometry}

Field:
\begin{itemize}
\item We have been doing geometry---eg linear programming
\item But in computational geometry, historical key difference in focus: \textbf{low dimension} $d$
\item (more recently, innovations in ``high dimensional CG'' 
\item Lots of algorithms that are great for $d$ small, but exponential
  in $d$
\end{itemize}

\newcommand{\conv}{\mathit{conv}}


\subsection{Range Trees for Orthogonal Range Queries}

One key idea in CG: reducing dimension
\begin{itemize}
\item Do some work that reduces problem to smaller dimension
\item Since few dimensions, work doesn't add up much.
\end{itemize}

What points are in this box?
\begin{itemize}
\item goal: $O(n)$ space
\item query time $O(\log n)$ plus number of points
\item (can't beat $\log n$ even for 1d)
\item 1d solution: binary tree.  
\begin{itemize}
\item Find leftmost in range
\item Walk tree till rightmost
\end{itemize}
\end{itemize}


Generalize: Solve in each coordinate ``separately'' %
\begin{itemize}
\item Idea 1: solve each coord, intersecting
\begin{itemize}
\item Too expensive: maybe large solution in each coord, none in
  intersection
\end{itemize}
\item Idea 2: 
\begin{itemize}
\item we know $x$ query will be an interval, 
\item so build a $y$-range structure on each distinct subrange of
  points by $x$
\item Use binary search to locate right $x$ interval
\item Then solve 1d range search on $y$
\item Problem: $n^2$ distinct intervals
\item So $n^3$ space and time to build
\end{itemize}
\end{itemize}

Refining idea 2:
\begin{itemize}
\item Build binary search tree on $x$ coords
\item Each internal node represents a tree subinterval containing some points
\item Our query's $x$ interval can be broken into $O(\log n)$ tree
  subintervals (think of the subtree between search paths of two endpoints of that $x$ interval) 
\item We want to reduce dimension: on each subinterval, range search
  $y$ coords \textbf{ only} among nodes in that $x$ interval
\item Solution: each internal node has a $y$-coord search tree on
  points in its subtree
\item Size: $O(n\log n)$, since each point in $O(\log n)$ internal
  nodes
\item Query time: find $O(\log n)$ nodes, range search in each
  $y$-tree, so $O(\log^2 n)$ (plus output size)
\item more generally, $O(\log^d n)$
\item \textbf{ fractional cascading} improves to $O(\log n)$
\end{itemize}

Dynamic maintenance:
\begin{itemize}
\item Want to insert/delete points
\item Problem to maintain tree balance
\item When insert $x$ coord, may want to re-balance
\item Rotations are obvious choice, but have to rebuild auxiliary
  structures
\item Linear cost to rotate a tree.
\item Remember treaps?  
\begin{itemize}
\item We showed expect 1 rotation
\item Can show expected size of rotated tree is small
\item Then insert $y$ coord in $O(\log n)$ auxiliary structures
\item So, $O(\log^2 n)$ update cost
\end{itemize}
\end{itemize}

\newcommand{\indx}{\mathit{index}}

\marnote{ 2011 Lecture 22 End}
\marnote{2013 Lecture 25 End}

\section{Sweep Algorithms}

Another key idea: 
\begin{itemize}
\item dimension is low 
\item so worth expending lots of energy to reduce dimension
%\item we saw this idea in LP
\item plane sweep is a general-purpose dimension reduction
\item Run a plane/line across space
\item Study only what happens on the frontier
\item Need to keep track of ``events'' that occur as sweep line across
\item simplest case, events occur when line hits a feature
\end{itemize}

\subsection{Convex Hull by Sweep Line}
\begin{itemize}
\item define
\item good for: width, diameter, filtering
\item assume no 3 points on straight line.
\item output:
  \begin{itemize}
  \item points and edges on hull
  \item in counterclockwise order
  \item can leave out edges by hacking implementation
  \end{itemize}
\item $\Omega(n\log n)$ lower bound via sorting
\end{itemize}

Build upper hull:
\begin{itemize}
\item Sort points by $x$ coord
\item Sweep line from left to right
\item maintain upper hull ``so far''
\item as encounter next point, check if hull turns right or left to it
\item if right, fine
\item if left, hull is concave.  Fix by deleting some previous points
  on hull.
\item just work backwards till no left turn.
\item Each point deleted only once, so $O(n)$
\item but $O(n\log n)$ since must sort by $x$ coord.
\end{itemize}

Output sensitive algorithm can achieve $O(n\log k)$ (Chan 1996).

\subsection{Halfspace intersection}

LP in 2 dimensions
\begin{itemize}
\item feasibility problem is deciding whether halfspace intersection
  nonempty
\item and optimization is easy, given intersection
\item just visit all vertices in linear time
\end{itemize}

Duality. 
\begin{itemize}
\item Assume origin inside intersection
\item map $(a,b) \rightarrow L_{ab} = \{(x,y) \mid ax+by+1=0\}$.
\item $(a,b) \in L_{qr}$ implies $qa+rb+1=0$ implies $(q,r) \in
  L_{ab}$
\item so $(q,r) = \bigcap_{(a,b) \in L_{qr}} L_{ab}$ 
\item line through two points becomes point at intersection of 2 lines
\item point at distance $d$ antipodal line at distance $1/d$.
\item intersection of halfspace become convex hull.
\end{itemize}

So, $O(n\log n)$ time.

What if no ``origin in intersection''?
\begin{itemize}
\item ignore verticals
\item divide into plances that ``look up'' and ``look down''
\item i.e. each contains either $(0,\infty)$ or $(0,-\infty)$
\item easy to find feasible ``origin'' in each half
\item then build polygon for each half
\item the interset the two polygons
\end{itemize}

\subsection{Segment intersections}

We saw this one using persistent data structures for a query structure.
\begin{itemize}
\item Maintain balanced search tree of segments ordered by current height.
\item Heap of upcoming ``events'' (line intersections/crossings)
\item pull next event from heap, output, swap lines in balanced tree
\item check swapped lines against neighbors for new intersection
  events
\item lemma: next event always occurs between neighbors, so is in heap
\item \textbf{ note:} next event is always in future (never have to
  backtrack).
\item so sweep approach valid
\item and in fact, heap is monotone!
\end{itemize}




\section{Voronoi Diagram}

Goal: find nearest Athena terminal to query point.

Definitions:
\begin{itemize}
\item point set $p$
\item $V(p_i)$ is space closer to $p_i$ than anything else
\item for two points, $V(P)$ is bisecting line
\item For 3 points, creates a new ``voronoi'' point
\item And for many points, $V(p_i)$ is intersection of halfplanes, so
  a convex polyhedron
\item And nonempty of course.
\item but might be infinite
\item Given VD, can find nearest neighbor via planar point location:
\item   $O(\log n)$ using persistent trees
\end{itemize}

Space complexity:
\begin{itemize}
\item VD is a \textbf{ planar graph}: no two voronoi edges cross (if count
  voronoi points)
\item add one point at infinity to make it a proper graph with ends
\item Euler's formula: $n_v-n_e+n_f=2$
\begin{itemize}
\item By induction.  
\item Removing one edge kills one face, cancels
\item Removing shortcutting degree two vertex removes on vertex and
  one edge, cancels
\item eventually you have a triangle: 3 vertices, 3 edges, 2 faces
  (inside and out).
\end{itemize}
\item ($n_v$ is voronoi points, not original ones)
\item But $n_f = n$
\item Also, every voronoi point has degree at least 3 while every edge
  has two endpoints.
\item Thus, $2n_e \ge 3(n_v+1)$
\item rewrite $2(n+n_v-2) \ge 3(n_v+1)$
\item So $n-2 \ge (n_v+3)/2$, \emph{i.e.}, $n_v \le 2n-7$
\item Gives $n_e \le 3n-6$
\end{itemize}

Summary: $V(P)$ has linear space and $O(\log n)$ query time.

\subsection{Construction}

VD is dual of projection of lower CH of lifting of points to parabola in 3D.

And 3D CH can be done in $O(n\log n)$

Can build each voronoi cell in $O(n\log n)$, so $O(n^2\log n)$.

\subsection{Sweep Line}

Basic idea:
\begin{itemize}
\item Build portion of Voronoi behind sweep line.
\item problem: not fully determined!  may be about to hit a new site.
\item What is determined?  Stuff closer to a point than to line
\item boundary is a parabola
\item boundary of known space is pieces of parabolas: ``beach line''
\item as sweep line descends, parabolas descend too. 
\item We need to maintain beach line as ``events'' change it
\end{itemize}

\textbf{ 2011 lecture 23 end}

Descent of one parabola:
\begin{itemize}
\item sweep line (horizontal) $y$ coord is $t$
\item Equation $(x-x_f)^2+(y-y_f)^2 = (y-t)^2$.
\item Fix $x$, find $dy/dt$
\item $2(y-y_f)dy/dt = 2(y-t)(dy/dt-1)$
\item So $dy/dt=(y-t)/(y_f-t)$
\item Thus, the higher $y_f$ (farther from sweep line) the slower
  parabola descends.
\end{itemize}

Two parabolas
\begin{itemize}
\item when sweep line hits second point, creates new parabola
\item called a ``site event''
\item new parabola starts degenerate---a vertical line
\item we can see where it intersects first parabola
\item as sweep descends, broadens into another parabola
\item splits first parabola in two---now three pieces
\item topologically interesting: intersection of parabolas
\begin{itemize}
\item point at intersection is equidistant from sweep and \emph{both}
  foci
\item which means it is on the bisector
\item as sweep descends, intersection traces out bisector
\item note there are two intersections; they trace out bisector in
  both directions
\item so one intersection is tracing \emph{upwards} as sweep descends
\end{itemize}
\end{itemize}

More parabolas
\begin{itemize}
\item more generally, voronoi edge gets traced by two intersecting parabolas
\item when does edge ``stop''?  When it collides with another edge to
  make a Voronoi point.
\item happens at intersection of 3 parabolas that were all beach line
\item equidistant from 3 sites: ``circle event''
\item after intersection, one parabola vanishes---falls behind the
  beach line
\end{itemize}

Summary:
\begin{itemize}
\item site event adds segment to beach line
\item circle event removes segment from beach line
\end{itemize}

Topology:
\begin{itemize}
\item ``breakpoints'' where two parabola's of beach line meet up
\begin{itemize}
\item these points are equidistant from sweep line \emph{and} both foci
\item which means they are on the bisector of the two points 
\item which is a voronoi edge because no other points are closer
\item since other parabola are ``higher'', so farther, so foci are farther too
\item as sweep line descends, this intersection traces out the voronoi
  edge
\end{itemize}
\item what changes breakpoints?  changes in voronoi edges
\item case 1: a single edge splits in two
\begin{itemize}
\item a new voronoi region is created ``between'' the previous two
\item 
\end{itemize}
\end{itemize}

What changes the beach line?

Site event:
\begin{itemize}
\item Sweep line hits site
\item creates new degenerate parabola (vertical line)
\item widens to normal parabola
\item adds arc piece to beach line.
\end{itemize}

Claim: no other create events.
\begin{itemize}
\item case 1: one parabola passing through one other
\begin{itemize}
\item At crossover, two parabolas are tangent.
\item 
\item then ``inner'' parabola has higher focus then outer
\item so descends slower
\item so outer one stays ahead, no crossover.
\item this case cannot happen
\end{itemize}
\item case 2: new parabola descends through intersection point of two
  previous parabolas.
\begin{itemize}
\item At crossover, all 3 parabolas intersect
\item thus, all 3 foci and sweep line on boundary of circle with
  intersection at center.
\item called \textbf{ circle event}
\item ``appearing'' parabola has highest focus
\item so it is slower: won't cross over
\item In fact, this is how parabola's \textbf{ disappear} from beach line
\item outer parabolas catch up with, cross inner parabola.
\end{itemize}
\end{itemize}

Summary:
\begin{itemize}
\item maintain beach line as pieces of parabolas
\item Note one parabola may appear several times
\item only \textbf{ site events} add to beach line
\item only \textbf{ circle events} remove from beach line.
\item $n$ site events
\item so only $n$ circle events
\item as insert/remove events, only need to check for events in newly
  adjacent parabolas
\item so $O(n\log n)$ time
\end{itemize}



\section{Randomized Incremental Constructions}

\subsection*{BSP}

\begin{itemize}
\item linearity of expectation.  hat check problem
\item Rendering an image
\begin{itemize}
\item render a collection of polygons (lines)
\item painters algorithm: draw from back to front; let front overwrite
\item need to figure out order with respect to user
\end{itemize}
\item define BSP.  
\begin{itemize}
\item BSP is a data structure that makes order determination easy
\item Build in pre-process step, then render fast.
\item Choose any hyperplane (root of tree), split lines onto correct
  side of hyperplane, recurse
\item If user is on side 1 of hyperplane, then nothing on side 2
  blocks side 1, so paint it first.  Recurse.
\item time=BSP size
\end{itemize}
\item sometimes must split to build BSP
\item how limit splits?
\item auto-partitions
\item random auto
\item analysis
  \begin{itemize}
  \item $\indx(u,v)=k$ if $k$ lines block $v$ from $u$
  \item $u \dashv v$ if $v$ cut by $u$ auto
  \item probability $1/(1+\indx(u,v))$.
  \item tree size is (by linearity of $E$)
    \begin{eqnarray*}
      n+\sum 1/\indx(u,v) &\le & \sum_u 2H_n
    \end{eqnarray*}
  \end{itemize}
\item result: \textbf{ exists} size $O(n\log n)$ auto
\item gives randomized construction
\item equally important, gives \textbf{ probabilistic existence proof} of
  a small BSP
\item so might hope to find deterministically.
\end{itemize}


\subsection*{Backwards Analysis---Convex Hulls}

Define.

algorithm (RIC):
\begin{itemize}
\item random order $p_i$
\item insert one at a time (to get $S_i$)
\item update $\conv(S_{i-1})\rightarrow \conv(S_i)$
  \begin{itemize}
  \item new point stretches convex hull
  \item remove new non-hull points
  \item revise hull structure
  \end{itemize}
\item Data structure:
  \begin{itemize}
  \item point $p_0$ inside hull (how find?)
  \item for each $p$, edge of $\conv(S_i)$ hit by $\vec{p_0p}$
  \item say $p$ \emph{cuts} this edge
  \end{itemize}
\item To update $p_i$ in $\conv(S_{i-1})$:
  \begin{itemize}
  \item if $p_i$ inside, discard
  \item delete new non hull vertices and edges
  \item $2$ vertices $v_1,v_2$ of $\conv(S_{i-1})$ become $p_i$-neighbors
  \item other vertices unchanged.
  \end{itemize}
\item To implement:
  \begin{itemize}
  \item detect changes by moving out from edge cut by $\vec{p_0 p}$.
  \item for each hull edge deleted, must update cut-pointers to
    $\vec{p_iv_1}$ or $\vec{p_iv_2}$
  \end{itemize}
\end{itemize}

Runtime analysis
\begin{itemize}
\item deletion cost of edges:
  \begin{itemize}
  \item charge to creation cost
  \item 2 edges created per step
  \item total work $O(n)$
  \end{itemize}
\item pointer update cost
  \begin{itemize}
  \item proportional to number of pointers crossing a deleted cut edge
  \item BACKWARDS analysis
    \begin{itemize}
    \item run backwards
    \item delete random point of $S_i$ (\textbf{ not} $\conv(S_i)$) to get $S_{i-1}$
    \item same number of pointers updated
    \item  expected number $O(n/i)$
      \begin{itemize}
      \item what $\Pr[\mbox{update }  p]$?
      \item $\Pr[\mbox{delete cut edge of } p]$
      \item $\Pr[\mbox{delete endpoint  edge of } p]$
      \item $2/i$
      \end{itemize}
    \item deduce $O(n\log n)$ runtime
    \end{itemize}
  \end{itemize}
\item  3d convex hull using same idea, time $O(n\log n)$,
\end{itemize}

\subsection{Linear Programming}

\begin{itemize}
\item define
\item assumptions:
  \begin{itemize}
  \item nonempty, bounded polyhedron
  \item minimizing $x_1$
  \item unique minimum, at a vertex
  \item exactly $d$ constraints per vertex
  \end{itemize}
\item definitions:
  \begin{itemize}
  \item hyperplanes $H$
  \item \textbf{ basis} $B(H)$
  \item optimum $O(H)$
  \end{itemize}
\item Simplex
  \begin{itemize}
  \item exhaustive polytope search:
  \item walks on vertices
  \item runs in $O(n^{d/2})$ time in theory
  \item often great in practice
  \end{itemize}
\item polytime algorithms exist, but bit-dependent!
\item OPEN: strongly polynomial LP
\item goal today: polynomial algorithms for small $d$
\end{itemize}


Randomized incremental algorithm
\[
T(n) \le T(n-1,d)+\frac{d}{n}(O(dn)+T(n-1,d-1)) = O(d!n)
\]

Combine with other clever ideas:
\[
O(d^2 n) + 2^{O(\sqrt{d \log d})}
\]

\subsection*{Trapezoidal decomposition:}


Motivation: 
\begin{itemize}
\item manipulate/analayze a collection of $n$ segments
\item assume no degeneracy: endpoints distinct
\item (simulate touch by slight crossover)
\item e.g. detect segment intersections
\item e.g., point location data structure
\item Basic idea:
\begin{itemize}
\item Draw verticals at all points and intersects
\item Divides space into slabs
\item binary search on $x$ coordinate for slab
\item binary search on $y$ coordinate inside slab (feasible since
  lines noncrossing)
\item problem: $\Theta(n^2)$ space
\end{itemize}
\end{itemize}

Definition.
\begin{itemize}
\item draw altitudes from each endpoints and intersection till hit a segment.
\item trapezoid graph is \emph{planar} (no crossing edges)
\item each trapezoid is a \emph{face}
\item show a face.  
\item one face may have many vertices (from altitudes that hit the
  \emph{outside} of the face)
\item but max vertex degree is 6 (assuming nondegeneracy)
\item so total space $O(n+k)$ for $k$ intersections.
\item number of faces also $O(n+k)$ (at least one edge/face, at most 2
  face/edge)
\item (or use Euler's theorem: $n_v-n_e+n_f \ge 2$)
\item standard clockwise pointer representation lets you walk around a face
\end{itemize}

Randomized incremental construction:
\begin{itemize}
\item to insert segment, start at left endpoint
\item draw altitudes from left end (splits a trapezoid)
\item traverse segment to right endpoint, adding altitudes whenever
  intersect 
\item traverse again, erasing (half of) altitudes cut by segment
\end{itemize}

Implementation
\begin{itemize}
\item clockwise ordering of neighbors allows traversal of a face in
  time proportional to number of vertices
\item for each face, keep a (bidirectional) pointer to all not-yet-inserted
  left-endpoints in face
\item to insert line, start at face containing left endpoint
\item traverse face to see where leave it
\item create intersection, 
  \begin{itemize}
  \item update face (new altitude splits in half)
  \item update left-end pointers
  \end{itemize}
\item segment cuts some altititudes: destroy half
  \begin{itemize}
  \item removing altitude merges faces
  \item update left-end pointers
  \item (note nonmonotonic growth of data structure)
  \end{itemize}
\end{itemize}
Analysis:
\begin{itemize}
\item Overall, update left-end-pointers in faces neighboring new line
\item time to insert $s$ is 
  \[
  \sum_{f \in F(s)} (n(f)+\ell(f))
  \]
  where 
  \begin{itemize}
  \item $F(s)$ is faces $s$ bounds after insertion
  \item $n(f)$ is number of vertices on face  $f$ boundary
  \item  $\ell(f)$ is number of left-ends inside $f$.
  \end{itemize}
\item So if $S_i$ is first $i$ segments inserted, expected work of
  insertion $i$ is
  \[
  \frac1i \sum_{s \in S_i}\sum_{f \in F(s)} (n(f)+\ell(f))
  \]
\item Note each $f$ appears at most 4 times in sum since at most 4
  lines define each trapezoid.
\item so $O(\frac1i \sum_f (n(f)+\ell(f)))$.
\item Bound endpoint contribution:
  \begin{itemize}
  \item note $\sum_f \ell(f) = n-i$
  \item so contributes $n/i$
  \item so total $O(n\log n)$ (tight to sorting lower bound)
  \end{itemize}
\item Bound intersection contribution
  \begin{itemize}
  \item $\sum n(f)$ is just number of vertices in planar graph
  \item So $O(k_i+i)$ if $k_i$ intersections between segments so far
  \item so cost is $E[k_i]$
  \item intersection present if both segments in first $i$ insertions
  \item so expected cost is $O((i^2/n^2)k)$
  \item so cost contribution $(i/n^2)k$
  \item sum over $i$, get $O(k)$
  \item \textbf{ note:} adding to RIC, assumption that first $i$ items are random.
  \end{itemize}
\item Total: $O(n\log n+k)$
\end{itemize}


\subsection*{Search structure}

Starting idea: 
\begin{itemize}
\item extend all vertical lines infinitely
\item divides space into slabs
\item binary search to find place in slab
\item binary search in slab feasible since lines in slab have total
  order
\item $O(\log n)$ search time
\end{itemize}

Goal: apply binary search in slabs, without $n^2$ space
\begin{itemize}
\item Idea: trapezoidal decom is ``important'' part of vertical lines
\item problem: slab search no longer well defined
\item but we show ok
\end{itemize}

The structure:
\begin{itemize}
\item A kind of search tree
\item ``$x$ nodes'' test against an altitude
\item ``$y$ nodes'' test against a segment
\item leaves are trapezoids
\item each node has two children
\item \textbf{But} may have many parents
\end{itemize}

Inserting an edge contained in a trapezoid
\begin{itemize}
\item update trapezoids
\item build a 4-node subtree to replace leaf
\end{itemize}

Inserting an edge that crosses trapezoids
\begin{itemize}
\item sequence of traps $\Delta_i$
\item Say $\Delta_0$  has left endpoint, replace leaf with $x$-node for
  left endpoint and $y$-node for new segment
\item Same for last $\Delta$
\item middle $\Delta$: 
\begin{itemize}
\item each got a piece cut off
\item cut off piece got merged to adjacent trapezoid
\item Replace each leaf with a $y$ node for new segment
\item two children point to appropriate traps
\item merged trap will have several parents---one from each premerge trap.
\end{itemize}
\end{itemize}

Search time analysis
\begin{itemize}
\item depth increases by one for new trapezoids
\item RIC argument shows depth $O(\log n)$
\begin{itemize}
\item Fix search point $q$, build data structure
\item Length of search path increased on insertion only if trapezoid
  containing $q$ changes
\item Odds of top or bottom edge vanishing (backwards analysis) are $1/i$
\item Left side vanishes iff \textbf{unique} segment defines that side and
  it vanishes
\item So prob. $1/i$
\item Total $O(1/i)$ for $i^{th}$ insert, so $O(\log n)$ overall.
\end{itemize}
\end{itemize}

\end{document}

% Local Variables: 
% mode: latex
% TeX-master: t
% End: 
