\documentclass[12pt]{article}
\usepackage{wide,me}
\parindent0pt

\title{Parallel Algorithms}
\author{David Karger}

\begin{document}

\marnote{2013 Lecture 29 Start}

Two closely related models of parallel computation.

Circuits
\begin{itemize}
\item Logic gates (AND/OR/not) connected by wires
\item important measures
\begin{itemize}
\item number of gates
\item depth (clock cycles in synchronous circuit)
\end{itemize}
\end{itemize}

PRAM
\begin{itemize}
\item $P$ processors, each with a RAM, local registers
\item global memory of $M$ locations
\item each processor can in one step do a RAM op or read/write to one
  global memory location
\item synchronous parallel steps
\item not realistic, but explores ``degree of parallelism''
\end{itemize}

Essentially the same models, but let us focus on different things.

\subsection*{Circuits}

\begin{itemize}
\item Logic gates (AND/OR/not) connected by wires
\item important measures
\begin{itemize}
\item number of gates
\item depth (clock cycles in synchronous circuit)
\end{itemize}
\item bounded vs unbounded fan-in/out
\item $AC(k)$ and $NC(k)$: unbounded and bounded fan-in with depth
  $O(\log^k n)$ for problems of size $n$
\item $AC(k) \subset NC(k) \subset AC(k+1)$ using full binary tree
\item $NC=\cup NC(k)=\cup AC(k)$
\end{itemize}

Addition
\begin{itemize}
\item consider adding $a_i$ and $b_i$ with carry $c_{i-1}$ to produce
  output $s_i$ and next carry $c_i$
\item Ripple carry: $O(n)$ gates, $O(n)$ time
\item Carry lookahead: $O(n)$ gates, $O(\log n)$ time
\item preplan for late arrival of $c_i$.
\item given $a_i$ and $b_i$, three possible cases for $c_i$
\begin{itemize}
\item if $a_i=b_i$, then $c_i=a_i$ determined without $c_{i-1}$:
  \emph{generate} $c_1=1$ or \emph{kill} $c_i=0$
\item otherwise, \emph{propagate} $c_i=c_{i-1}$
\item write $x_i=k,g,p$ accordingly
\end{itemize}
\item consider $3 \times 3$ ``multiplication table'' for effect of two
  adders in a row.  
\item pair propagates previous carry only if both propagate.
\begin{tabular}{rr|c|c|c}
            &&   &$x_i$&\\
            &&$k$&$p$&$g$\\
\hline
         &$k$&$k$&$k$  &$g$\\
$x_{i-1}$&$p$&$k$&$p$  &$g$\\
         &$g$&$k$&$g$  &$g$\\
\end{tabular}
\item Now let $y_0=k$, $y_i = y_{i-1} \times x_i$
\item constraints ``multiplication table'' by induction
\begin{tabular}{rr|c|c|c}
            &&   &$x_i$&\\
            &&$k$&$p$&$g$\\
\hline
         &$k$&$k$&$k$  &$g$\\
$y_{i-1}$&$p$&$k$&never&$g$\\
         &$g$&$k$&$g$  &$g$\\
\end{tabular}
\item conclude: $y_i=k$ means $c_i=0$, $y_i=g$ means $c_i=1$, and
  $y_i=p$ never happens
\item so problem reduced to computing all $y_i$ in parallel
\end{itemize}

Parallel prefix
\begin{itemize}
\item Build full binary tree
\item two gates at each node
\item pass up product of all children
\item pass down product of all $x_i$ preceding leftmost child
\item works for any associative function
\end{itemize}


\subsection*{PRAM}

various conflict resolutions (CREW, EREW, CRCW)
\begin{itemize}
\item $CRCW(k) \subset EREW(k+1)$
\item $NC = \cup CRCW(k)$
\end{itemize}

PRAMs simulate circuits, and vice versa
\begin{itemize}
\item So $NC$ well-defined
\end{itemize}

differences in practice
\begin{itemize}
\item CRCW can OR in $O(1)$ time
\item EREW requires $\log n$ time (info theory lower bound)
\item EREW needs $\log n$ to find max
\item CRCW finds max in constant time with $n^2$ processors
\begin{itemize}
\item Compare every pair
\item If an item loses, write ``not max'' in its entry
\item Check all entries
\item If item is max (not overwritten), write its value in answer
\end{itemize}
\item in $O(\log\log n)$ time with $n$ processors
\begin{itemize}
\item Suppose $k$ items remain
\item Make $k^2/n$ blocks of $n/k$ items 
\item quadratic time max for each: $(k^2/n)(n/k)^2=n$ processors total
\item recurrence: $T(k)=1+T(k^2/n)$
\item $T(n/2^i)=1+T(n/2^{2i})$
\item so $\log\log n$ iters.
\end{itemize}
\end{itemize}

parallel prefix
\begin{itemize}
\item using $n$ processors
\end{itemize}

list ranking EREW
\begin{itemize}
\item next pointers $n(x)$
\item $d(x)+=d(n(x))$; $n(x)=n(n(x))$.
\item by induction, sum of values on path to end doesn't change
\end{itemize}

\section{Work-Efficient Algorithms}

Idea: 
\begin{itemize}
\item We've seen parallel algorithms that are somewhat ``inefficient''
\item do more total work  (processors times time) than sequential
\item Ideal solution: arrange for total work to be
  proportional to best sequential work
\item \emph{Work-Efficient Algorithm}
\item Then a small number of  processors (or even 1) can ``simulate''
  many processors in a fully efficient way
\item Parallel analogue of ``cache oblivious algorithm''---you write
  algorithm once for many processors; lose nothing when it gets
  simulated on fewer.
\end{itemize}


Brent's theorem
\begin{itemize}
\item Different perspective on work: count number of processors actually
working in each time step.
\item If algorithm does $x$ total work and critical path $t$
\item Then $p$ processors take $x/p+t$ time
\item So, if use $p=x/t$ processors, finish in time $t$ with efficient
  algorithm
\item detail: assigning processors to tasks can be tricky if e.g. work
  emerges dynamically during execution
\end{itemize}

Work-efficient parallel prefix
\begin{itemize}
\item linear sequential work
\item going to need $\log n$ time
\item so, aim to get by with $n/\log n$ processors
\item give each processor a block of $\log n$ items to add up
\item reduces problem to $n/\log n$ values
\item use old algorithm
\item each processor fixes up prefixes for its block
\end{itemize}

Work-efficient list ranking
\begin{itemize}
\item harder: can't easily give contiguous ``blocks'' of $\log n$ to each
  processor (requires list ranking)
\item However, assume items in arbitrary order in array of $ n$
  structs, so \emph{can} give $\log n$ distinct items to each processor.
\item use random coin flips to knock out ``alternate'' items
\item shortcut any item that is heads and has tails successor
\item requires at most one shortcut
\item and constant probability every other item is shortcut (and
  independent)
\item so by chernoff, 1/16 of items are shortcut out
\item ``compact'' remaining items into smaller array using parallel
  prefix on \textbf{array} of pointers (ignoring list structure) to
  collect only ``marked'' nodes and update their pointers 
\item let each processor handle $\log n$ (arbitrary) items
\item $O(n/\log n)$ processors, $O(\log n)$ time
\item After $O(\log\log n)$ rounds, number of items reduced to $n/\log
  n$
\item use old algorithm
\item result: $O(\log n \log\log n)$ time, $n/\log n$ processors
\item to improve, use faster ``compaction'' algorithm to collect
  marked nodes: $O(\log\log n)$ time randomized, or $O(\log n/\log\log
  n)$ deterministic.  get optimal alg.
\item How about deterministic algorithm?  Use ``deterministic coin
  tossing''
\item take all local maxima as part of ruling set.
\end{itemize}

Euler tour to reduce to parallel prefix for
\begin{itemize}
\item inorder traversal of tree
\item computing depth in tree
\item numbering leaves
\item work efficient
\end{itemize}

\section*{Expression Tree Evaluation}

plus and times nodes.

Idea
\begin{itemize}
\item merge leaves into parents (good for bushy tree)
\item pointer jump degree-2 nodes (good for tall trees)
\item combine?
\item how ``pointer jump'' an operator?
\end{itemize}

Generalize problem:
\begin{itemize}
\item ``product lookahead''
\item Each tree edge has a label $(a,b)$ 
\item meaning that if subtree below evaluates to $y$ then value
  $(ay+b)$ should be passed up edge
\end{itemize}

Merging a single leaf
\begin{itemize}
\item method for eliminating all left-child leaves
\item root $q$ with right child $p$ (product node) on edge labelled $(a_3,b_3)$
\item $p$ has left child edge $(a_1,b_1)$ leaf $\ell$ with value $v$
\item right child edge to $s$ with label $(a_2,b_2)$
\item fold out $p$ and $\ell$, make $s$ a child of $q$
\item what label of new edge?
\item prepare for $s$ subtree to eval on $y$.
\item choose $a,b$ such that
  $ay+b=a_3\cdot[(a_1v+b_1)\cdot(a_2y+b_2)]+b_3$
\end{itemize}

Parallelize
\begin{itemize}
\item keep tree full, so killing all leaves kills whole tree
\item collapse a leaf and pointer jump parent
\item problem: can't do to both children at once
\item solution: number leaves in-order (Euler Tour)
\item three step process:
\begin{itemize}
\item shunt odd-numbered left-child leaves
\item shunt odd-number right-child leaves
\item divide leaf-numbers by 2
\item guarantees never simultaneously shunt siblings
\end{itemize}
\end{itemize}


\section*{Sorting}

%% Bitonic sort:
%% \begin{itemize}
%% \item Recall sorting bitonic sequences:
%% \begin{itemize}
%% \item Compare/swap opposite pairs
%% \item Recurse on halves
%% \item $O(\log n)$ time with $n$ processors.
%% \end{itemize}
%% \item Bitonic sort can be used to merge two sorted sequences
%% \item So use merge sorts in parallel to make groups of 1,2,4,8 etc.
%% \item $O(\log^2 n)$ time with $n$ processors
%% \end{itemize}


Parallelizing binary search
\begin{itemize}
\item placing one item into a sorted array takes $\log n$ time with 1
  processor
\item but with $k$ processors can improve to $\log_k n$
\item compare to $k$ evenly spaced items
\item use concurrent OR to find out which pair I'm between in $O(1)$ time
\item recurse
\item shows how to gain parallel speed by wasting work
\item in particular, can find in $O(1)$ time with $n$ processors
\item or even $\sqrt{n}$ processors
\end{itemize}


CREW Merge sort:
\begin{itemize}
\item merge two length-$k$ sequences using $k$ processors
\item each element of first seq. uses binary search to find place in
  second
\item so knows how many items smaller
\item so knows rank in merged sequence: go there
\item then do same for second list
\item $O(\log k)$ time with $k$ processors
\item handle all length-$k$ lists in parallel with $n$ processors
\item total time $O(\sum_{i \le \lg n} \log 2^i)=O(\log^2 n)$ with $n$ processors
\end{itemize}

Better with more processors?
\begin{itemize}
\item use $n^2$ processors to do all pairwise comparisons in parallel 
\item For each item, count number of wins in parallel in $O(\log n)$
  time
\item This determines item's rank in output list
\end{itemize}

Use same ideas with fewer processors?
\begin{itemize}
\item Note can sort $\sqrt{n}$ items in $O(\log n)$ time
\item Use insights to improve merge?
\item Idea: nearby items in one list go to nearby places in merged list
\item So don't do whole computation for every item
\item Break $A$ into blocks and figure out roughly where each block
  goes in $B$
\item Then spread each block to exact right places in $B$ (recursively)
\end{itemize}

Details: merge $n$ items of $A$ into $m$ items of $B$ in $O(\log\log n)$
  time using $m+n$ processors
\begin{itemize}
\item choose $\sqrt{n}$ evenly spaced fenceposts
  $\alpha_i$ among $A$
\item use $\sqrt{m}$ processors to find exact location of each in $B$
\item total processors $\sqrt{nm}\le \le \max(n,m) \le n+m$ 
\item split $B$ at locations of $\alpha_i$
\item now know $\sqrt{n}$ items (between $\alpha_i$ and $\alpha_{i+1}$)
  go in each split piece of $B$
\item So recurse: $T(n)=2+T(\sqrt{n})=O(\log\log n)$
\end{itemize}

Use in parallel merge sort: $O(\log n \log\log n)$ with $n$
processors.
\begin{itemize}
\item Cole shows how to ``pipeline'' merges, get optimal $O(\log n)$ time.
\end{itemize}

Qsort idea
\begin{itemize}
\item $\sqrt{n}$ pivots
\item sort all in $O(\log n)$ time using $n$ processors
\item binary search each item to locate position in pivots
\item recurse
\item $T(n) = O(\log n) + T(\sqrt{n}) = O(\log n)$
\item rigorous analysis messy
\item need to deal with variation in sizes of subproblems
\end{itemize}


\section*{Connectivity and connected components}

Linear time sequential trivial.

\subsection*{Directed}

Squaring adjacency matrix
\begin{itemize}
\item $\log n$ time to reduce diameter to 1
\item $mn$ processors for first iter, but adds edges
\item so, $n^3$ processors
\item and space to $n^2$ even if initial graph is sparse
\item improvements to $mn$ processors
\item But ``transitive closure bottleneck'' still bedevils parallel
  algs.
\item Even if just want to find $s$-$t$ path we don't know any better
\item Problem is that output has size $n^2$ and we don't know which
  part of it matters.
\end{itemize}

\subsection*{Undirected}

Basic approach:
\begin{itemize}
\item Sets of connected vertices grouped as stars
\item One root, all others parent-point to it
\item Initially all vertices alone
\item Edge ``live'' if connects two distinct stars
\item Find live edges in constant time by checking roots
\item For live edge with roots $u<v$, connect $u$ as child of $v$
\item May be conflicts, but CRCW resolves
\item Now get stars again
\begin{itemize}
\item Use pointer jumping
\item Note: may have chains of links, so need $\log n$ jumps
\end{itemize}
\item Every live star attached to another
\item So number of stars decreases by 2
\item $m+n$ processors, $\log^2 n$ time.
\end{itemize}

Smarter: interleave hooking and jumping:
\begin{itemize}
\item Maintain set of rooted trees
\item Each node points to parent
\item Hook some trees together to make fewer trees
\item Pointer jump (once) to make shallower trees
\item Eventually, each connected component is one star
\end{itemize}

More details:
\begin{itemize}
\item ``top'' vertex: root or its children
\item detect top vertex in constant time
\item each vertex has label
\item find root label of each top vertex
\item Can detect if am star in constant time:
\begin{itemize}
\item no pointer double reaches root
\end{itemize}
\item for each edge:
\begin{itemize}
\item If ends both on top, different nonstar components, then hook smaller
  index component to larger
\item may be conflicting hooks; assume CRCW resolves
\item indices prevent cycles
\item If star points to non-star, hook it
\item do one pointer jump
\end{itemize}
\end{itemize}

Potential function: height of live stars and tall trees
\begin{itemize}
\item Live stars get hooked to something (star or internal)
\item But never hooked to leaf.  So add 1 to height of target
\item So sum of heights doesn't go up
\item But now, every unit of height is in a tall tree
\item Pointer doubling decreases by at least 1/3
\item Total height divided each time
\item So $\log n$ iterations
\end{itemize}
Summary: $O(m+n)$ processors, $O(\log n)$ time.

Improvements:
\begin{itemize}
\item $O((m+n)\alpha(m,n)/\log n)$ processors, $\log n$ time, CRCW
\item Randomized $O(\log n)$, $O(m/\log n)$ processors, EREW
\end{itemize}


\section{Randomization}


Randomization in parallel:
\begin{itemize}
\item load balancing
\item symmetry breaking
\item isolating solutions
\end{itemize}

Classes:
\begin{itemize}
\item NC: poly processor, polylog steps
\item RNC: with randomization.  polylog runtime, monte carlo
\item ZNC: las vegas NC
\item immune to choice of R/W conflict resolution
\end{itemize}

\section*{Sorting}

Quicksort in parallel:
\begin{itemize}
\item $n$ processors
\item each takes one item, compares to splitter
\item count number of predecessors less than splitter
\item determines location of item in split
\item total time $O(\log n)$
\item combine: $O(\log n)$ per layer with $n$ processors
\item problem: $\Omega(\log^2 n)$ time bound
\item problem: $n\log^2 n$ work
\end{itemize}

Using many processors:
\begin{itemize}
\item do all $n^2$ comparisons
\item use parallel prefix to count number of items less than each item
\item $O(\log n)$ time
\item or $O(n)$ time with $n$ processors
\end{itemize}

Combine with quicksort:
\begin{itemize}
\item Note: single pivot step inefficient: uses $n$ processors and
  $\log n$ time.
\item Better: use $\sqrt{n}$ simultaneous pivots
\item Choose $\sqrt{n}$ random items and sort fully to get $\sqrt{n}$ intervals
\item For all $n$ items, use binary search to find right interval
\item recurse
\item $T(n)=O(\log n)+T(\sqrt{n})=O(\log n + \frac12\log n+\frac14\log
  n + \cdots)=O(\log n)$
\end{itemize}

Formal analysis:
\begin{itemize}
\item consider root-leaf path to any item $x$
\item argue total number of parallel steps on path is $O(\log n)$
\item consider item $x$
\item claim splitter within $\alpha\sqrt{n}$ on each side
\item since prob. not at most $(1-\alpha\sqrt{n}/n)^{\sqrt{n}} \le
  e^{-\alpha}$
\item fix $\gamma, d<1/\gamma$
\item define $\tau_k = d^k$
\item define $\rho_k = n^{(2/3)^k}$ $(\rho_{k+1}=\rho_k^{2/3})$
\item note size $\rho_k$ problem takes $\gamma^k\log n$ time
\item note size $\rho_k$ problem odds of having child of size
  $>\rho_{k+1}$ is less than $e^{-\rho_k^{1/6}}$
\item argue at most $d^k$ size-$\rho_k$ problems whp
\item follows because probability of $d^k$ size-$\rho_k$ problems in a
  row is at most 
\item deduce runtime $\sum d^k\gamma_k = \sum (d\gamma)^{k}\log n =
  O(\log n)$ 
\item note: as problem shrinks, allowing more divergence in quantity
  for whp result
\item minor detail: ``whp'' dies for small problems
\item OK: if problem size $\log n$, finish in $\log n$ time with $\log
  n$ processors
\end{itemize}

\section*{Maximal independent set}

trivial sequential algorithm
\begin{itemize}
\item inherently sequential
\item from node point of view: each thinks can join MIS if others stay
  out
\item randomization breaks this symmetry
\end{itemize}

Randomized idea
\begin{itemize}
\item each node joins with some probability
\item all neighbors excluded
\item many nodes join
\item few phases needed
\end{itemize}


Algorithm:
\begin{itemize}
\item all degree 0 nodes join
\item node $v$ joins with probability $1/2d(v)$
\item if edge $(u,v)$ has both ends marked, unmark lower degree vertex
\item put all marked nodes in IS
\item delete all neighbors
\end{itemize}


Intuition: $d$-regular graph
\begin{itemize}
\item vertex vanishes if it or neighbor gets chosen
\item mark with probability $1/2d$
\item prob (no neighbor marked) is $(1-1/2d)^d$, constant
\item so const prob. of neighbor of $v$ marked---destroys $v$
\item what about unmarking of $v$'s neighbor?
\item prob(unmarking forced) only constant as argued above.  
\item So just changes constants
\item const fraction of nodes vanish: $O(\log n)$ phases
\item Implementing a phase trivial in $O(\log n)$.
\end{itemize}




Prob chosen for IS, given marked, exceeds $1/2$
\begin{itemize}
\item suppose $w$ marked.  only unmarked if higher degree neighbor
  marked
\item higher degree neighbor marked with prob. $\le 1/2d(w)$
\item only $d(w)$ neighbors
\item prob. any superior neighbor marked at most $1/2$.
\end{itemize}

For general case, define good vertices
\begin{itemize}
\item good: at least $1/3$ neighbors have lower degree
\item prob. no neighbor of good marked $\le (1-1/2d(v))^{d(v)/3} \le
  e^{-1/6}$.
\item So some neighbor marked with prob. $1-e^{-1/6}$
\item Stays marked with prob. 1/2
\item deduce prob. good vertex killed exceeds $(1-e^{-1/6})/2$
\item Problem: perhaps only one good vertex?
\end{itemize}

Good edges
\begin{itemize}
\item any edge with a good neighbor
\item has const prob. to vanish
\item show half edges good
\item deduce $O(\log n)$ iterations.
\end{itemize}

Proof
\begin{itemize}

\item Let $V_B$ be bad vertices; we count edges with both ends in $V_B$.
\item direct edges from lower to higher degree $d_i$ is indegree,
  $d_o$ outdegree
\item if $v$ bad, then $d_i(v) \le d(v)/3$
\item deduce 
\[\sum_{V_B} d_i(v) \le \frac13 \sum_{V_B}
  d(v)=\frac13\sum_{V_B}( d_i(v)+d_o(v))
\]
\item so $\sum_{V_B} d_i(v) \le \frac12 \sum_{V_B} d_o(v)$
\item which means indegree can only ``catch'' half of outdegree; other
  half must go to good vertices. 
\item 
more carefully, 
\begin{itemize}
\item $d_o(v)-d_i(v) \ge
  \frac13(d(v))=\frac13(d_o(v)+d_i(v))$.  
\item Let $V_G,V_B$ be good, bad vertices
\item degree of bad vertices is 
\begin{eqnarray*}
2e(V_B,V_B)+e(V_B,V_G)+e(V_G,V_B)
&= & \sum_{v\in V_B} d_o(v)+d_i(v)\\
&\le &3\sum(d_o(v)-d_i(v))\\
&= &3(e(V_B,V_G)-e(V_G,V_B))\\
&\le &3(e(V_B,V_G)+e(V_G,V_B)
\end{eqnarray*}
Deduce $e(V_B,V_B) \le e(V_B,V_G)+e(V_G,V_B)$.  result follows.
\end{itemize}
\end{itemize}

Derandomization:
\begin{itemize}
\item Analysis focuses on edges, 
\item so unsurprisingly, pairwise independence sufficient
\item not immediately obvious, but again consider $d$-uniform case
\item prob vertex marked $1/2d$
\item neighbors $1,\ldots,d$ in increasing degree order
\item Let $E_i$ be event that $i$ is marked. 
\item Let $E'_i$ be $E_i$ but no $E_j$ for $j<i$
\item $A_i$ event no neighbor of $i$ chosen
\item Then prob eliminate $v$ at least 
\begin{eqnarray*}
\sum \Pr[E'_i \cap  A_i] &= &\sum\Pr[E'_i]\Pr[A_i \mid E'_i]\\
&\ge &\sum \Pr[E'_i]\Pr[A_i]
\end{eqnarray*}
\item Wait: show $\Pr[A_i \mid E'_i] \ge \Pr[A_i]$
\begin{itemize}
\item true if independent
\item measure $\Pr[\neg A_i \mid E'_i] \le \sum \Pr[E_w \mid E'_i]$
  (sum over neighbors $w$ of $i$)
\item measure 
\begin{eqnarray*}
\Pr[E_w \mid E'_i] &= &\frac{\Pr[E_w \cap E']}{\Pr[E'_i]}\\
&= &\frac{\Pr[(E_w \cap \neg E_1 \cap \cdots) \cap E_i]}{\Pr[(\neg E_1
  \cap \cdots) \cap E_i]} \\
&= &\frac{\Pr[E_w \cap \neg E_1 \cap \cdots \mid E_i]}{\Pr[\neg E_1
  \cap \cdots \mid E_i]} \\
&\le &\frac{\Pr[E_w \mid E_i]}{1-\sum_{j\le i}\Pr[E_j \mid E_i]}\\
&= &\Theta(\Pr[E_w])
\end{eqnarray*}
(last step assumes $d$-regular so only $d$ neighbors with odds $1/2d$)
\end{itemize}
\item But expected marked neighbors $1/2$, so by Markov $\Pr[A_i]>1/2$
\item so prob eliminate $v$ exceeds $\sum\Pr[E'_i]=\Pr[\cup E_i]$
\item lower bound as $\sum\Pr[E_i]-\sum\Pr[E_i \cap E_j] =
  1/2-d(d-1)/8d^2 > 1/4$
\item so $1/2d$ prob. $v$ marked but no neighbor marked, so $v$ chosen
\item Generate pairwise independent with $O(\log n)$ bits
\item try all polynomial  seeds in parallel
\item one works
\item gives deterministic $NC$ algorithm
\end{itemize}

with care, $O(m)$ processors and $O(\log n)$ time (randomized)

LFMIS P-complete.


\section*{Perfect Matching}

We focus on bipartite; book does general case.


Last time, saw detection algorithm in $\RNC$:
\begin{itemize}
\item Tutte matrix
\item Sumbolic determinant nonzero iff PM
\item assign random values in $1,\ldots,2m$
\item Matrix Mul, Determinant in $\NC$
\end{itemize}

How about finding one?
\begin{itemize}
\item If unique, no problem
\item Since only one nozero term, ok to replace each entry by a 1.
\item Remove each edge, see if still PM in parallel
\item multiplies processors by $m$
\item but still $\NC$
\end{itemize}

Idea: 
\begin{itemize}
\item make unique minimum \textbf{weight} perfect matching
\item find it
\end{itemize}

Isolating lemma: [MVV]
\begin{itemize}
\item Family of distinct sets over $x_1,\ldots,x_m$
\item assign random weights in $1,\ldots,2m$
\item Pr(unique min-weight set)$\ge 1/2$
\item Odd: no dependence on number of sets! 
\item (of course $<2^m$)
\end{itemize}

Proof:
\begin{itemize}
\item Fix item $x_i$
\item $Y$ is min-value sets containing $x_i$
\item $N$ is min-value sets not containing $x_i$
\item true min-sets are either those in $Y$ or in $N$
\item how decide? Value of $x_i$
\item For $x_i=-\infty$, min-sets are $Y$
\item For $x_i=+\infty$, min-sets are $N$
\item As increase from $-\infty$ to $\infty$, single transition value
  when both $X$ and $Y$ are min-weight
\item If only $Y$ min-weight, then $x_i$ in every min-set
\item If only $X$ min-weight, then $x_i$ in no min-set
\item If both min-weight, $x_i$ is \emph{ambiguous}
\item Suppose no $x_i$ ambiguous.  Then min-weight set unique!
\item Exactly one value for $x_i$ makes it ambiguous given remainder
\item So Pr(ambiguous)$=1/2m$
\item So Pr(any ambiguous)$< m/2m=1/2$
\end{itemize}

Usage:
\begin{itemize}
\item Consider tutte matrix $A$
\item Assign random value $2^{w_i}$ to $x_i$, with $w_i \in
  1,\ldots,2m$
\item Weight of matching is $2^{\sum w_i}$
\item Let $W$ be minimum sum
\item Unique w/pr $1/2$
\item If so, determinant is odd multiple of $2^W$
\item Try removing edges one at a time
\item Edge in PM iff new determinant$/2^W$ is even.
\item Big numbers?  No problem: values have poly number of bits
\end{itemize}

$NC$ algorithm open.

For exact matching, $P$ algorithm open.

\newcommand{\Wbar}{\overline W}

\end{document}
% Local Variables: 
% mode: latex
% TeX-master: t
% End: 
