\documentclass{article}
\usepackage{me}
%\input{abbrevs}
\setlength{\parindent}{0pt}

\title{Flow}
\author{David Karger}

\begin{document}

%\marnote{2011 Lecture 6 start}
%\marnote{2013 Lecture 8 start}

Combinatorial Optimization: Finding Optimum Solution
\begin{itemize}
\item Breakdown approach
\item Solution: Understand feasibility
\item Optimum: Verifying that a solution is optimum
\item And understanding how to improve one that is not optimum
\item Finding: actually producing the optimum comes last.
\end{itemize}

\section{Maximum Flow}

\subsection{Definitions}

Tarjan: \emph{Data Structures and Network Algorithms}

Ford and Fulkerson, \emph{Flows in Networks}, 1962 (paper 1956)

Ahuja, Magnanti, Orlin \emph{Network Flows}.  Problem: do min-cost.

Problem: in a graph, find a \emph{flow} that is \emph{feasible} and has
maximum \emph{value}.

Directed graph, edge \emph{capacities} $u(e)$ or $u(v,w)$.  Why not
$c$? reserved for costs, later.

\emph{source} $s$, \emph{sink} $t$

Goal: assign a \emph{flow} value to each edge:
\begin{itemize}
\item \emph{conservation:} for every vertex $v$, $\sum_w g(v,w)-g(w,v)=0$, unless $v=s,t$
\item \emph{capacity:} $0 \le g(e) \le u(e)$ (flow is {\em
    feasible/legal}) (sometimes lower bound $l(e)$)
\item Flow \emph{value} $|f| = \sum_w g(s,w)-g(w,s)$ (in gross model).
\end{itemize}

Alternative: (net flow form)
\begin{itemize}
\item \emph{skew symmetry}: $f(v,w) = -f(w,v)$
\item \emph{conservation:} for every vertex $v$, $\sum_w f(v,w)=0$, unless $v=s,t$
\item \emph{capacity:} $f(e) \le u(e)$ (flow is {\em
    feasible/legal}) (sometimes lower bound $l(e)$)
\item Flow \emph{value} $|f| = \sum_w f(s,w)$
\end{itemize}


Net flow cleaner for math but involves negative flows and makes
lower-bound flows messy.  Raw flow may
fit intuition/reality better.

Water hose intuition.  Also routing commodities, messages under
bandwidth constraints, etc.  Often ``per unit time'' flows/capacities.

Maximum flow problem: find flow of maximum value.


Path decomposition (another perspective):
\begin{itemize}
\item claim: any $s$-$t$ flow can be decomposed into paths/cycles with quantities
\item proof: induction on number of edges with nonzero flow
\item if $s$ has out flow, traverse flow outward from $s$
\item till find an $s$-$t$ path or cycle (why can we?
  conservation) and kill
\item if $t$ has in-flow (won't any more, but no need to prove) work
  backwards to $s$ or cycle and kill
\item if some other vertex has outflow, find a cycle and kill
\item corollary: flow into $t$ equals flow out of $s$ (global
  conservation)
\item Note: every path is a flow (balanced).
\item Note: decomposition of \emph{positive} value
  \begin{itemize}
  \item i.e. never have paths traveling opposite each other
  \end{itemize}
\item Note: at most $m$ paths/cycles (zero out one edge's flow each time)
\end{itemize}

\marnote{Point A. 15 min to here.}

\marnote{2014 Lec 8 start}

Decision question: is there nonzero feasible flow
\begin{itemize}
\item Suppose yes.  consider one.
\item decompose into paths
\item proves there is a flow path
\item Suppose no
\item then no flow path
\item consider vertices reachable from $s$
\item they form a \textbf{cut} $(s\in S, t \in T={\overline S})$
\item no edge crossing cut
\end{itemize}

%\marnote{2012 lecture 8 start}

What about max-flow? 		% 2017 lecture 9 (9/27/2017) start
\begin{itemize}
\item Need some upper bound to decide if we are maximal
\item Consider any flow, and cut $(S,T)$
\item decompose into paths
\item every path leaves the cut (at least) once
\item So, outgoing cut capacity must exceed flow value
\item \textbf{min cut} is upper bound for \textbf{max flow}
\item So if flow equals cut value, we know we have a max flow and a min cut.
\end{itemize}

Suppose have some flow.  How can we send more?
\begin{itemize}
\item Might be no path in original graph
\begin{verbatim}

      /|\
     / | \
    /  |  \
 s /   |   \
   \   |   / t
    \  |  /
     \ | /
      \|/

\end{verbatim}
\item Instead need \textbf{residual graph:}
\begin{itemize}
\item Suppose flow $f(v,w)$
\item set $u_f(v,w) = u(v,w)-f(v,w)$ (decrease in available capacity)
\item which means set $u_f(w,v) = u(w,v)+f(v,w)$ (increase in capacity indicates
  can remove $(v,w)$ flow
\end{itemize}
\item If \textbf{augmenting path} in residual graph, can increase flow
  \begin{itemize}
  \item might use residual arcs that are reverse of original arcs
  \item this represents \emph{removing} flow from that residual arc
  \item so not a real path in original graph
  \item but, augmentation does preserve conservation invariant at each vertex
  \end{itemize}
\end{itemize}

What if no augmenting path?
  \begin{itemize}
\item Implies zero cut $(S,T)$ in residual graph
\item Implies every $S$-$T$ edge is saturated
  \item and every $T$-$S$ edge is empty
\item i.e. cut is saturated
\item consider path decomposition
\item each path crosses the cut exactly once
  \begin{itemize}
  \item must cross once to reach $t$
  \item cannot cross back (and cross again) because $T$-$S$ empty
  \end{itemize}
\item Thus, capacity of crossing paths equals capacity of crossing edges
\item i.e. value of cut equals value of flow
\item but already saw value of any flow cannot exceed
  value of any cut (because all paths cross cut at least once)
\item so equality means flow is maximum
  \end{itemize}

We have proven Max-flow Min-cut \textbf{duality}:
\begin{itemize}
\item Equivalent statements:
  \begin{itemize}
  \item $f$ is max-flow
  \item no augmenting path in $G_f$
  \item $|f| = u(S)$ for some $S$
  \end{itemize}
\end{itemize}

Proof summary:
\begin{itemize}
\item if augmenting path, can increase $f$
\item if no augmenting path, can find empty residual cut
\item flow has same value as cut (so is optimal)
\item conversely, if flow equals cut they must both be optimal
\end{itemize}

Another cute corollary:
\begin{itemize}
\item Net flow across any cut (in minus out) equals flow value
\item True for a path as it must go out once more than in
\item So true for a sum of paths
\end{itemize}

Note that max-flow and min-cut are witnesses to each others'
optimality.

% Added for the 9/16/19 lecture (as we plan to blaze through the intro topics)

\section{Maximum Flow Algorithms}

We understand now the optimality conditions for maximum flow well. But how about computing one?

\textbf{Observe:} The Max-Flow Min-Cut theorem can be easily turned into an actual algorithm. 

\subsection{Computing Max Flow with Augmenting Paths}

\textbf{Natural algorithm:} Repeatedly find augmenting paths in the residual graph. 
\begin{itemize}
	\item By Max-Flow Min-Cut theorem we know that either our flow is already optimal or there is an augmenting path to find.
	\item Each augmenting path computation can be easily done in $O(m)$ time.
	\item \textbf{Key question:} How many times we might need to find an augmenting path before we reach the max flow?
	\item Consider first the case of {\em integer} capacities. (This will be most often the case anyway.)
	\begin{itemize}
		\item If capacities are integral then the bottleneck capacity of each augmenting path we find is integer and positive, so at least $1$.
		
		$\Rightarrow$ We have at most $|f|$ augmenting path computations, giving an $O(m|f|)$ overall bound. 
		
		\item Note that, by Max-Flow Min-Cut Theorem, 
		\[
		|f|=u(S^*)\leq u(\{s\})\leq nU,
		\]
		where $U:=\max_e u_e$ is the largest edge capacity. 
		
		$\Rightarrow$ The overall running time is $O(mnU)$.
                \begin{itemize}
                  \item \textbf{Note:} This running time is polynomial in the {\em values} in our input
                \item but not in the {\em size} of its binary representation.
                \item (Such algorithms are called {\em pseudo-polynomial}.)
                  \item When $U=1$, this is polynomial time algorithm though.
                \end{itemize}
		\item \textbf{Important corollary:} If all capacities are integral then there exists an integral maximum flow too. (Although there could still exist maximum flows that are non-integral.)
	\end{itemize}
      \item When capacities are {\em rational}, we will augment by (multiple of) their least common denominator
        
        $\Rightarrow$ least common denominator never changes
        
	$\Rightarrow$ running time finite, but likely not even pseudo-polynomial. 
	\item When capacities are {\em real} numbers, we might not even have termination.
\end{itemize}

\textbf{Sidenote:} The above algorithm was simple to describe but its dynamics can be quite complex.
\begin{itemize}
	\item Each augmentation constitutes a ``greedy'' improvement step, but these augmentations might undo/overwrite previous updates, adding flow to an edge and then removing it. 
	\item Instructive example to look at: ``diamond'' graph with edges $(s, v)$, $(s,w)$, $(v,w)$, $(v,t)$, and $(w,t)$, and capacities being $1000$ on all the edges except $(v,w)$ which has a capacity of $1$. What will happen if we always choose an augmenting path going through the capacity $1$ edge? How does it compare to making a ``smart'' choice of the augmenting path?
\end{itemize}

\subsection{Maximum Bottleneck Capacity Augmenting Paths}

So far, we assumed that we just find {\em an} augmenting path. But maybe there is a ``smart'' way to choose one?

\textbf{Natural idea:} Always choose an augmenting path $P$ that has maximum bottleneck capacity $u_f(P)$.

\begin{itemize}
	\item How to find one? Actually, quite simple.  Easy $O(m\log n)$-time algorithm: just do binary search over the capacity of ``bottlenecking'' edge and, for each guess $u$, check for $s$-$t$ path in $G_f$ after all the edges with capacity less than $u$ were removed. By being (much) more sophisticated, one can get $O(\min\{ m+n \log n, m \log^*n\})$ time. 
	\item How much progress each such augmenting path guarantees to make?
	\item \textbf{Recall:} The flow decomposition bound tells us that any flow can be decomposed into at most $m$ flow paths. 
	
	$\Rightarrow$ If we apply this observation to the maximum (remaining) flow, i.e., the maximum flow in the {\em residual} graph $G_f$ together with an averaging argument,  we get that at least one of these paths carries at least $1/m$-fraction of the remaining value of the flow. (Note that all such paths have to be augmenting in $G_f$ by definition.)
	
	$\Rightarrow$ Each flow augmentation reduces the remaining value of the flow by a factor of $(1-1/m)$. 
	\item After $m\ln |f| +1$ augmentations, the remaining flow would be less than $1$. So, if capacities are integral, the obtained solution has to be optimal. The overall running time is 
	\[
	O((m\log n)\cdot m \ln |f|)=O(m^2 \log n \log nU).
	\]
	\item \textbf{Note:} This is a ``truly'' polynomial running time, i.e., it is polynomial in the size of the representation of the input numbers. Still, sometimes one can ask for even more: running time that is {\em independent} of the input numbers, i.e., depends only on the size of the graph itself. This kind of algorithms are called {\em strongly polynomial} (and the ones that are only polynomial in the size of the numbers are called {\em weakly polynomial}).
	\item If capacities are rational, the above algorithm still has interesting properties. In particular, if $U$ will be now the largest denominator/enumerator in the capacities, then by multiplying all capacities by the product of all denominators and applying the algorithm above will give us a running time of
	\[
	O((m\log n)\cdot m \ln |f|)=O(m^2 \log n \log nU^m)=O(m^3 \log n \log nU),
	\]
	which is still (weakly) polynomial. (Note: we do not actually need to explicitly multiply all capacities here.)
\end{itemize} 

\textbf{From now on:} Unless stated otherwise, we assume that capacities are integral.  	% 2017 lecture 9 (9/27/2017) end



\end{document}

%%% Local Variables:
%%% mode: latex
%%% TeX-master: t
%%% End:
