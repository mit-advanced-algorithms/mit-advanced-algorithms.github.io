\documentclass{article}
\usepackage{me}
%\input{abbrevs}
\setlength{\parindent}{0pt}
\def\zhat{{\hat z}}

\title{Fixed Parameter Tractability}
\author{David Karger}

\begin{document}

We've talked about approximation algorithms that do well on \textbf{all}
instances.  How about on \emph{some} instances?
\begin{itemize}
\item For any one instance, there is polytime alg.
\item Look for interesting (natural, common) ``easy'' \bf{classes} of
instances
\end{itemize}

Vertex cover
\begin{itemize}
\item Suppose solution has $k$ vertices
\item Try all possibilities?  $O(mn^k)$
\item Polynomial for any fixed $k$, but in a bad way (like PAS, not
  FPAS).
\item Of course, cannot really have FPAS since $k \leq n$ would mean it
  gives exact solution in polynomial time
\end{itemize}

Bounded search tree method
\begin{itemize}
\item Consider search space as a search tree
\item Argue you can cut it short, so exponential in levels isn't too bad
\item Pick any edge
\begin{itemize}
\item one end is in VC
\item try each
\item each ``branch'' of search tree is binary
\item but after $k$ branchings you are done (success or failure clear)
\item So, $2^k$ leaves in search tree
\item $O(m)$ work per path
\item time $O(2^k m)$
\end{itemize}
\end{itemize}

Kernelization
\begin{itemize}
\item In polytime (independent of $k$) find a subproblem of size confined with a function of $k$ such that the
  solution of this subproblem extends to the whole problem
\item Gives runtime of $p(n)+f(k)$, ``better'' than $p(n)\cdot f(k)$
\item eg vertex cover
\begin{itemize}
\item any vertex of degree $>k$ must be in cover
\item mark, remove with incident edges
\item remainder is covered by at most $k$ vertices of degree $<k$
\item so at most only $k^2$ edges
\item find opt in $f(k)$
\item e.g. $2^k \cdot k^2$ using bounded search tree method.
\end{itemize}
\end{itemize}

Equivalence
\begin{itemize}
\item Actually, kernelization is no better
\item these three are equivalent:
\begin{itemize}
\item algorithm with runtime $f(k)\cdot p(n)$
\item algorithm with runtime $f(k)+p(n)$
\item algorithm to reduce to kernel of size $f(k)$
\end{itemize}
\item Proof is nonconstructive, however.
\end{itemize}


% 2011 Lecture 17 start (based on overlap with previous Demaine lecture)

\subsection{Treewidth}
Imagine a canonical recursive solution on a graph
\begin{itemize}
\item Pick a vertex
\item For each possible setting of it, compute optimum solution on
  rest of graph
\item Kind of what we did in bounded search tree for FPT
\item What goes wrong?  Exponential search space.  New problem for
  each setting of variable
\end{itemize}

Perhaps we can ``memoize'' the recursion, turn into dynamic
  programming?
\begin{itemize}
\item In many graph problems, feasibility is determined by ``local''
   constraints on vertices and who they interact with
\begin{itemize}
\item Graph coloring
\item Max independent set
\item vertex cover
\item matching
\end{itemize}
\item edges in graph represent ``interactions'' constraining joint
   behavior of neighboring variables
\item Perhaps only need to memoize one answer for each state of
neighbors, ignoring rest of the graph
\item get something exponential in the degree
\item Intuition: maximum matching in a tree
\begin{itemize}
\item Root tree anywhere
\item Compute, for subtree $T$ with root $v$, max matching in $T$ with $v$
  matched and unmatched
\item Can evaluate for $v$ given tables of children of $v$
\end{itemize}
\item intuition: vertex cover
\end{itemize}

Elimination orderings
\begin{itemize}
\item Represent an ``unraveling'' of the graph one vertex at a time,
  with plans to memoize
\item Problem to be solved: when I eliminate a vertex, I create hidden
  interactions between neighbors of that vertex
\item To represent those interactions, need to add edges between all
  neighbors.
\item If can do so without ever creating large neighbor set, there is
  hope!
\end{itemize}

Treewidth
\begin{itemize}
\item \textbf{ Induced Treewidth:} For a given graph, for any elimination ordering, induced treewidth of that ordering is the size of the largest neighbor set created by
  that elimination ordering
\item \textbf{ Graph Treewidth:} For any graph, its treewidth is the induced treewidth of the best elimination ordering (the one with smallest induced treewidth)
\item Treewidth 1: tree
\item Treewidth 2: series-parallel graphs
\end{itemize}

SAT
\begin{itemize}
\item Treewidth not just for problems on graphs
\item Use graph to reflect interactions between any variables
\item eg, edge between two variables if they share a clause
\item Maintain truth-table for each clause---list of satisfying
  assignments for it
\item Take eliminated vertex/variable $v$
\item Combine its clause's truth tables---combined clause is happy
  with a given assignments if original set are all happy for
  \emph{some} setting of $v$
\item Reduced formula is SAT iff original is
\item Size of new clause: degree of $v$
\item So, if small treewidth, never create large clause
\item In which case, easy to maintain tables
\item Runtime: $n$ eliminations, each involving about $O(2^k)$ work, where k is the treewidth of the graph.
\item When unwinding, always have satisfying assignment that can be
  extended to remove vertex
\end{itemize}





\end{document}

%%% Local Variables: 
%%% mode: latex
%%% TeX-master: t
%%% End: 
