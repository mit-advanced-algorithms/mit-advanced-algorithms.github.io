\documentclass[12pt]{article}
\usepackage{wide,me}
\parindent0pt

\title{Hashing}
\author{David Karger}

\begin{document}


\marnote{This material takes 1:05.}

Everything you need to know about probability
\begin{itemize}
\item Linearity of expectation
\item Indicator variables
\item Independent events
\item Product rule
\item Markov inequality
\end{itemize}

\subsection*{Hashing}

Dictionaries
\begin{itemize} 
   \item Operations.   
      \begin{itemize} 
      \item makeset, insert, delete, find
      \end{itemize}
\end{itemize}

Model
\begin{itemize}
\item keys are integers in $M=\set{1,\ldots,m}$ 
\item (so assume machine word size, or ``unit time,'' is $\log m$)
\item can store in array of size $M$
\item using power: arithmetic, indirect addressing
\item (more than bucket based heaps that use indirection without arithmetic)
\item compare to comparison and pointer based sorting, binary trees
\item problem: space.
\end{itemize}

Hashing:
\begin{itemize}
\item find function $h$ mapping $M$ into table of size $s \ll m$
\item use it to put $n$ items from $M$ in the table 
\item so presumably $s \ge n$
\item Still, some items can get mapped to same place: ``collision''
\item use linked list etc.
\item search, insert cost equals size of linked list
\item goal: keep linked lists small: few collisions
\end{itemize}


Hash families:
\begin{itemize}
\item problem: for any hash function, some bad input (if space $s$,
  then $m/s$ items to same bucket)
\item This true even if hash is e.g. SHA1
\item Solution: build family of functions, choose one that works well
\end{itemize}

\marnote{2013 lecture 7 start}

Set of all functions?
\begin{itemize}
\item Idea: choose ``function'' that stores items in sorted order
  without collisions
\item problem: to evaluate function, must examine data
\item evaluation time $\Omega(\log n)$.
\item ``description size'' $\Omega(n \log m)$,
\item Better goal: choose function without looking at data (except query key)
\end{itemize}

How about a random function?
\begin{itemize}
\item set $N$ of $n$ items
\item If $s=n$,  balls in bins
\begin{itemize}
\item $O((\log n)/(\log\log n))$  collisions w.h.p.
\item And matches that somewhere
\item but we may care more about \emph{average} collisions over many operations
\item $C_{ij}=1$ if $i,j$ collide
\item Time to find $i$ is $1+\sum_j C_{ij}$
\item expected value $1+(n-1)/n \le 1$
\end{itemize}
\item more generally expected search time for item (present or not):
  $O(n/s)=O(1)$ if $s=n$
\end{itemize}

Problem: 
\begin{itemize}
\item $n^m$ functions (specify one of $n$ places for each of $n$
  items)
\begin{itemize}
\item too much space to specify ($m\log n$), 
\item hard to evaluate
\end{itemize}
\item for $O(1)$ search time, want to evaluate function in
  $O(1)$ time.  
\begin{itemize}
\item so function description must fit in $O(1)$ machine words
\item Assuming $\log m$ bit words
\item So, fixed number of cells can only distinguish poly$(m)$
  functions
\end{itemize}
\item This bounds size of hash family we can choose from
\end{itemize}

2-universal family: [Carter-Wegman]
\begin{itemize}
\item Key insight: don't need entirely random function
\item All we care about is which pairs of items collide
\item so: OK if items land \emph{pairwise independent}
\item pick prime (or prime power) $p$ in range $m,\ldots,2m$ (not random)
\item pick random $a,b$
\item map $x$ to $(ax + b \bmod p)\bmod m$
  \begin{itemize}
  \item fix $x$, $y$
  \item mapping pairwise independent, uniform before $\mod m$
  \begin{align*}
     ax+1\cdot b &= s\\
     ay+1\cdot b &= t\\
  \end{align*}
  matrix $\binom{x\quad 1}{y \quad 1}$ determinant $x-y \ne 0$, so unique solution
  \item So pairwise independent, near-uniform after $\mod m$
  \item at most 2 ``uniform buckets'' to same place
  \end{itemize}
\item argument above holds: $O(1)$ expected search time.
\item represent with two $O(\log m)$-bit integers: hash family of poly
  size.
\item \emph{max} load may be large as $\sqrt{n}$, but who cares?
  \begin{itemize}
  \item expected load in a bin is 1
  \item so $O(\sqrt{n})$ with prob. 1-1/n (chebyshev).
  \item this bounds expected max-load
  \item some item may have bad load, but unlikely to be the requested
  one
\item can show the max load is probably achieved for some 2-universal families
  \end{itemize}
\end{itemize}

\marnote{2012 lecture start}

\subsection*{perfect hash families}
Fredman Komlos Szemeredi.

Ideally, would hash with \emph{no} collisions
\begin{itemize}
\item Get \emph{worst case} $O(1)$ lookups
\item Explore case of fixed set of $n$ items (read only)
\item perfect hash function: no collisions
\item Even fully random function of $n$ to $n$ has collisions
\end{itemize}

Alternative try: use more space:
\begin{itemize}
\item How small can $s$ be for random $n$ to $s$ without collisions?
  \begin{itemize}
  \item Expected number of collisions is $E[\sum C_{ij}] =
    \binom{n}{2}(1/s) \approx n^2/2s$
  \item \textbf{Markov Inequality:} $n=\sqrt{s}$ works with prob. 1/2 
  \item Nonzero probability, so, 2-universal hashes can work in
  quadratic space.
  \end{itemize}
\item Is this best possible?
  \begin{itemize}
  \item Birthday problem: $(1-1/s)\cdots(1-n/s) \approx
  e^{-(1/s+2/s+\cdots+n/s)} \approx e^{-n^2/2s}$
  \item So, when $n=\sqrt{s}$ has $\Omega(1)$ chance of collision
  \item 23 for birthdays
  \item even for fully independent
  \end{itemize}
\end{itemize}

Finding one
\begin{itemize}
\item We know one exists---how find it?
\item Try till succeed
\item Each time, succeed with probability 1/2
\item Expected number of tries to succeed is 2
\item so expect $O(n)$ work
\item Probability need $k$ tries is $2^{-k}$
\end{itemize}

Two level hashing for linear space
\begin{itemize}
\item Hash $n$ items into $O(n)$ space 2-universally
\item Build quadratic size hash table on contents of each bucket
\item bound $\sum b_k^2=\sum_k (\sum_i [i \in b_k])^2 =\sum C_i + C_{ij}$
\item expected value $O(n)$.  
\item So try till get (markov)
\item Then build collision-free quadratic tables inside 
\item Try till get 
\item Polynomial time in $n$, Las-vegas algorithm
\item Easy: $6n$ cells
\item Hard: $n+o(n)$ cells (bit fiddling)
\end{itemize}

\marnote{2013 Lecture 7 End}

Define las vegas, compare to monte carlo.

Derandomization
\begin{itemize}
\item Probability $1/2$ top-level function works
\item Only $m^2$ top-level functions
\item Try them all!
\item Polynomial in $m$ (not $n$), deterministic algorithm
\end{itemize}

Followups
\begin{itemize}
\item Dynamic Perfect Hashing
\end{itemize}


\end{document}

% Local Variables: 
% mode: latex
% TeX-master: t
% End: 
