\documentclass{article}
\usepackage{me}
\setlength{\parindent}{0pt}

\title{Persistent Data Structures\\ 6.854 Notes \#2}
\author{David Karger}

\begin{document}

\marnote{This material takes 1 hour.}

\section{Persistent Data Structures}


Sarnak and Tarjan, "Planar Point Location using persistent trees",
Communications of the ACM 29 (1986) 669--679

"Making Data Structures Persistent" by Driscoll, Sarnak, Sleator and Tarjan 
Journal of Computer and System Sciences 38(1) 1989

Idea: be able to query and/or modify past versions of data structure.
\begin{itemize}
\item ephemeral: changes to struct destroy all past info
\item partial persistence: changes to most recent version, query to
  all past versions
\item full persistence: queries and changes to all past versions
  (creates ``multiple worlds'' situtation)
\end{itemize}

Goal: general technique that can be applied to \emph{any} data
structure.



\section{Persistent Trees}

Many data structures
\begin{itemize}
\item consist of fixed-size \emph{nodes}
  conencted by \emph{pointers}
\item accessed from some \emph{root} entry point
\item data structure ops traverse pointers
\item and modify node fields/pointers
\item runtime determined by number of traverse/modify steps
\end{itemize}

Our goal: use \emph{simulation} to transform any ephemeral data
structure into a persistent one
\begin{itemize}
\item measure of success: time and space cost for each \emph{primitive
    traversal} and \emph{primitive modification} step.
\item An exposed data structure operation will do some number of
  primitive steps
\item we will add persistence to primitive steps, so any data
  structure can be made persistent
\item we'll measure cost of simulation in slowdown and storage
\item We'll limit to trees.
\item But ideas generalize
\end{itemize}


Key principles:
\begin{itemize}
\item Lazy: don't work till you must
\item If you must work, use your work to ``simplify''
  data structure too
\item force user (adversary) to spend lots of time to make you work
\item analysis via potential function measuring ``complexity'' of
  structure.  
\begin{itemize}
\item user has to do lots of changes to raise potential, 
\item so you can spread cost of complex ops over many insertions.  
\item potential says how much work \emph{adversary} has done that you
  haven't had to deal with yet.  
\item You can charge your work against that work.
\end{itemize}
\item another perspective: procrastinate.  if you don't do the work,
  might never need to.
%\item Why the name?  Wait and see.
\end{itemize}

Lazy approach:
\begin{itemize}
\item During change, do minimum, i.e. nothing.
\item Indeed, for only change, don't even need to remember change!
\item but we also want lookups, so record changes
\item Then apply relevant changes when a lookup happens
\item So, amortized cost 1.
\end{itemize}

Problem?
\begin{itemize}
\item issue with second and further lookups
\item so, do need to incorporate 
\end{itemize}




Full copy bad.

Fat nodes method:
\begin{itemize}
\item replace each (single-valued) field of data structure by list of
  all values taken, sorted by time.
\item requires $O(1)$ space per data change (\textbf{unavoidable} if keep
  old data)
\item to lookup data field, need to search based on time.
\item store values in binary tree
\item checking/changing a data field takes $O(\log m)$ time after $m$
  updates
\item \textbf{multiplicative} slowdown of $O(\log m)$ in structure
  access.
\item big number from data structures perspective
\end{itemize}

Path copying:
\begin{itemize}
\item can only reach node by traversing pointers starting from root
\item changes to a node only visible to \emph{ancestors} in pointer
  structure
\item when change a node, copy it and ancestors (back to root of
  data structure
\item keep list of roots sorted by update time
\item $O(\log m)$ time to find right root (or const, if time is
  integers) (\textbf{additive} slowdown)
\item same access time as original structure
\item \emph{additive} instead of multiplicative $O(\log m)$.
\item modification time and space usage equals number of ancestors:
  possibly huge!
\item especially if structure not balanced!
\end{itemize}

Combined Solution (trees only):
\begin{itemize}
\item Path copying is too aggressive
\item how can we be lazy about path copying?
\item fat nodes bad, but ``pleasingly plump'' ok.
\item use fat nodes, but when get too fat, make new path copy
\item in each node, store 1 \emph{extra} time-stamped field
\item if full, overrides one of standard fields for any accesses
  later than stamped time.
\item access rule
\begin{itemize}
\item standard access, just check for overrides while following
  pointers
\item constant factor increase in access time.
\end{itemize}
\item update rule: 
\begin{itemize}
\item when need to change/copy pointer, use extra field if available.
\item otherwise, make new copy of node with new info, and recursively
  modify parent.
\item Idea: lazy path copying
\item copy only as much as you must
\end{itemize}
\end{itemize}

Analysis idea: adversary has to do many changes to make copying expensive.
\begin{itemize}
\item analysis: \textbf{potential function} $\phi$ measuring amount of
  ``deferred work''.
\begin{itemize}
\item amortized$_i=$ real$_i+\phi_i-\phi_{i-1}$
\item then $\sum a_i=\sum r_i+\phi_n-\phi_0$
\item upper bounds real cost if $\phi_n \ge \phi_0$.
\item sufficient that $\phi_n \ge 0$ and $\phi_0$ fixed
\end{itemize}
\end{itemize}

Analysis
\begin{itemize}
\item What represents work you need to do later?
  \begin{itemize}
  \item full nodes (that may need copying)
  \item that may still be modified (current values)
  \end{itemize}
\item \emph{live} node: pointed at by \emph{current} root.
\item potential function: number of \emph{full} live nodes.
\item copying a node is free (new copy not full, pays for copy
  space/time)
\item pay for filling an extra pointer (do only once, since can stop
  at that point).
\item amortized space per update: $O(1)$.
\end{itemize}



\section{Application: planar point location.}
\begin{itemize}
\item a computational geometry foreshadow
\begin{itemize}
\item Different model
\item algebraic operations are $O(1)$
\item e.g. computing line intersections or lengths
\item even when these return irrationals!
\item sweep precision under the rug.
\end{itemize}
\item planar subdivision
\begin{itemize}
\item division of plane into polygons
\item query: ``what polygon contains this point''
\item define complexity of input?
\item edges of polygons form $n$ \emph{segments}/edges
\item segments meet only at ends/vertices
\end{itemize}
\item numerous special-purpose solutions
\item One solution: dimension reduction
\begin{itemize}
\item vertical line through each vertex
\item divides into slabs
\item in slab, segments maintain one vertical ordering
\item find query point slab by binary search
\item build binary search tree for slab with ``above-below'' queries
\item $n$ binary search trees, size $O(n^2)$, time $O(n^2\log n)$
\end{itemize}
\item observation: trees all very similar
\item think of $x$ axis as time, slabs as ``epochs''
\item at end of epoch, ``delete'' segments that end, ``insert'' those
  that start.
\item over all time, only $n$ inserts, $n$ deletes.
\item must be able to query over all times
\end{itemize}

Persistent sorted sets:
\begin{itemize}
\item find$(x,s,t)$ find (largest key below) $x$ in set $s$ at time $t$
\item insert$(i,s,t)$ insert $i$ in $s$ at time $t$
\item delete$(i,s,t)$.
\end{itemize}



%Power of twos: Like Fib heaps.  Show binary tree of modifications.

\section{Method: persistent red-black trees}

Red black trees
\begin{itemize}
\item aggressive rebalancers
\item amortized cost $O(1)$ to change a field.
\item store red/black bit in each node
\item use red/black bit to rebalance.
\item depth $O(\log n)$
\end{itemize}

Add persistence
\begin{itemize}
\item updates only in ``present''
\item search: standard binary tree search; $O(1)$ slowdown for persistence
\item update: causes changes in red/black fields on path to item,
  $O(1)$ rotations.
\item result: $(\log n)$ space \emph{per insert/delete}
\item geometry does $O(n)$ changes, so $O(n\log n)$ space.
\item $O(\log n)$ query time.
\end{itemize}

Improvement:
\begin{itemize}
\item red-black bits used only for updates
\item only need current version of red-black bits
\item don't persist old bit versions: just overwrite
\item only persistent updates needed are for $O(1)$ rotations
\item so $O(1)$ space per update
\item so $O(n)$ space overall.
\end{itemize}

Result: $O(n)$ space, $O(\log n)$ query time for planar point
location.

Extensions:
\begin{itemize}
\item method extends to arbitrary pointer-based structures.  
\item $O(1)$ cost per update for any pointer-based structure with any
  constant indegree and constant number of fields.
\item use one extra pointer per field in node
\item and one pointer per indegree
\item full persistence with same bounds.
\end{itemize}

\end{document}
