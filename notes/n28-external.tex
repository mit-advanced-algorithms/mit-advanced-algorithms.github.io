\documentclass{article}
\usepackage{me}
%\input{abbrevs}
\setlength{\parindent}{0pt}

\title{External Memory}
\author{David Karger}

\begin{document}

Memory $M$, block size $B$, problem size $N$

basic operations:
\begin{itemize}
\item Scanning: $O(N/B)$
\item reversing an array: $O(N/B)$
\end{itemize}

matrix operations
\begin{itemize}
\item $N\times N$ matrix
\item Add?
\begin{itemize}
\item need to decide on representation
\item use row-major dense representation
\item add is linear scan
\end{itemize}
\item Multiply?
\item How externalize standard algorithm?
\begin{itemize}
\item Row major
\begin{itemize}
\item scan rows of first and columns of second
\item so need $N$ reads to get each column entry from second matrix
\item and $N^2$ items to read
\item so $N^3$ over all
\end{itemize}
\item Column major
\begin{itemize}
\item What if could somehow transpose second matrix?
\item Now only need $N/B$ reads per output value
\item So $N^3/B$
\item Still wasteful. How?
\item Many output values rely on same column
\item So we read the same rows repeatedly
\item How avoid?
\end{itemize}
\item Block layout
\begin{itemize}
\item Think of matrix as array of smaller matrices
\item Multiply main matrix as sum of products of smaller matrices
\item We've seen addition is linear 
\item Bigger blocks will help
\item But the limit: need block to fit in memory
\item so size $\sqrt{M} \times \sqrt{M}$
\item so need to read $N/\sqrt{M}$ blocks to produce an output block
\item time per block is $M/B$ since block size $M$
\item $N^2/M$ output blocks
\item so total time is product $N^3/B\sqrt{M}$
\end{itemize}
\end{itemize}
\item Can improve this by starting with better sequential matrix
  multiply---e.g Strassen.  
\begin{itemize}
\item get to $N^{2.376}$ sequential
\item can extend to external memory
\end{itemize}
\end{itemize}



Linked list
\begin{itemize}
\item operations insert, delete, traverse
\item insert and delete cost $O(1)$
\item must traversing $k$ items cost $O(k)$?
\item keep segments of list in blocks
\item keep each block half full
\item now can traverse $B/2$ items on one read
\item so $O(K/B)$ to traverse $K$ items
\item on insert: if block overflows, split into two half-full blocks
\item on delete:
  \begin{itemize}
  \item if block less than half full, check next block
  \item if it is more than half full, redistribute items
  \item now both blocks are at least half full
  \item otherwise, merge items from both blocks, drop empty block
  \end{itemize}
\item Note: can also insert and delete while traversing at cost $1/B$
  per operation
\item so, e.g., can hold $O(1)$ ``fingers'' in list and insert/delete
  at fingers.
\end{itemize}

\marnote{2013 Lecture 27 End}

Search trees:
\begin{itemize}
\item binary tree cost $O(\log n)$ memory transfers
\item wastes effort because only getting one useful item per block
  read
\item Instead use $(B+1)$-ary tree; block has $B$ splitters
\item Require all leaves at same depth
\item ensures balanced
\item So depth $O(\log_B N)$, much better
\item Need to keep balanced while insert/delete:
  \begin{itemize}
  \item require every block but root to be at least half full (so degree $\ge
  B/2$)
\item also ensures depth $\log_B N$
  \item on insert, if block is full, split into two blocks and pass up a
    splitter to insert in parent
  \item may overflow parent, forcing recursive splits
  \item on delete, if block half empty, merge as for linked lists
  \item may empty parent, force recursive merges
  \end{itemize}
\item optimal in comparison model:
  \begin{itemize}
  \item Reading a block reveals position of query among $B$ splitters
  \item i.e. $\log B$ bits of information
  \item Need $\log N$ bits.
  \item so $\log N/\log B$ queries needed.
  \end{itemize}
\end{itemize}

\marnote{2011 Lecture 24 end.  2012 Lecture 26 end.}


Sorting:
\begin{itemize}
\item Tree sort?
\begin{itemize}
\item $N$ inserts into $B$-tree, then scan
\item time $O(N\log_B N)$.
\end{itemize}
\item Standard merge sort based on linear scans: $T(N)=2T(N/2) +
  O(N/B)=O((N/B)\log N)$
\item (Better than tree sort: $\log B \rightarrow B$.
\item Can do better by using more memory
\item $M/B$-way merge
\item keep head block of each of $M/B$ lists in memory
\item keep emitting smallest item
\item when emit $B$ items, write a block
\item when block empties, read next block of that list
\item $T(M) = O(N/B)+(M/B)T(N/(M/B)) = O((N/B)\log_{M/B}N/B)$
\end{itemize}

Optimal for comparison sort:
  \begin{itemize}
  \item Assume each block starts and is kept sorted (only
 strengthens lower bound)
  \item loading a block reveals placement of $B$ sorted items among
  $M$ in memory
  \item $\binom{M+B}{B} \approx (eM/B)^B$ possibilities
  \item so $\log() \approx B\log M/B$ bits of info
  \item need $N\log N$ bits of info about sort, 
  \item except in each of $N/B$ blocks $B\log B$ bits are redundant
  (sorted blocks)
  \item total $N\log B$ redundant i.e. $N\log N/B$ needed.
  \item now divide by bits per block load
  \end{itemize}

\subsection*{Buffer Trees}

Mismatch:
\begin{itemize}
\item In memory binary search tree can sort in $O(n\log n)$ (optimal)
using inserts and deletes
\item But sorting with $B$-tree costs $N\log_B N$
\item So inserts are individually optimal, but not ``batch optimal''
\item Basic problem: writing one item costs 1, but writing $B$ items
  together only costs 1, i.e. $1/B$ per item
\item Is there a data structure that gives ``batch optimality''?
\item Yes, but if inserts/queries are to happen in batches, sometimes
  you will have to wait for an answer until your batch is big enough
\item Can achieve optimum throughput,  but must sacrifice latency.
\end{itemize}

Basic idea:
\begin{itemize}
\item Focus on supporting $N$ inserts and then doing inorder traversal
\item Idea: keep buffer in memory, push into $B$-tree when we have a
  block's worth
\item Problem: different items push into different children, no longer
  a block's worth going down
\item Solution: keep a buffer at each internal node
\item Still a problem: writing one item into the child buffer costs 1
  instead of $1/B$
\item Solution: make buffers \textbf{huge}, so most children get whole
  blocks written
\end{itemize}

Details:
\begin{itemize}
\item make buffer ``as big as possible'': size $M$
\item increase tree degree to $M/B$
\item but still B items per leaf node---tree over individual blocks
\item basic operation: pushing overflowing buffer down to children
\begin{itemize}
\item invariant: arriving overflow items are sorted 
\item buffer had $M$
\item sort $M$ items
\item merge with sorted incoming $X$ in time $O(M/B+X/B)$ (write to external)
\item write sorted contiguous elements to proper children
\item check for child overflow, recurse
\item cost is at most 1 partial block per child ($M/B$ children) plus 1 block per $B$
  items ($K/B$) so $M/B+K/B < 2K/B$.
\item since at least $M$ items, account as $1/B$ per item
\end{itemize}
\item on insert, put item in root buffer (in memory, so free)
\item when a buffer fills, flush
\item may fill child buffers.  flush recursively
\item when flush reaches leaves, store items using standard $B$-tree
  ops (split leaf nodes, possibly recursing splits)
\begin{itemize}
\item cost of splits dominated by buffer flushing
\item how handle buffers when split leaves?
\item no problem: they are empty on root-leaf path
  because we have just flushed to leaves.
\end{itemize}
\item cost of flushes:
\begin{itemize}
\item buffer flush costs $1/B$ per item
\item but each flushed item descends a level
\item total levels $\log_{M/B} N/B$
\item So cost per item is $(1/B)\log_{M/B} N/B$
\item So cost to insert $N$ is optimal sort cost
\end{itemize}
\end{itemize}


``Flush'' operation
\begin{itemize}
\item empties \emph{all} buffers so can directly query structure
\item full buffers already accounted
\item unfull buffers cost $M/B$ per internal node
\item but number of internal nodes is $(N/B)/(M/B)$ ($N/B$ leaf blocks
  divided among $M/B$ leaf-blocks per parent)
\item so total cost $N/B$
\item so can sort in $N/B\log_{M/B} N/B$
\end{itemize}


Extensions
\begin{itemize}
\item search: ``insert'' query, look for closest match as flushes down
\item delete: insert ``hole'' that triggers when it catches up to target item
\item delete-min: extra buffer of $M$ minimum elements in memory.
  when empties, flush left path and extract leftmost items to refill
\item range search: insert range, push to \emph{all} matching children
  on flush
\item all ops take $1/B \log_{M/B}N/B$ (plus range output)
\end{itemize}

\section*{Cache Oblivious Algorithms}

In practice, don't want to consider $B$ and $M$
\begin{itemize}
\item hard-coding them makes your algorithm non-portable
\item and, with multi-level memory hierarchies, unclear where bottleneck is,
  i.e. which $B$ and $M$ you use
\item without knowing $B$ and $M$, how can we tune for them?
\item surprisingly often, you can
\item key: divide and conquer algorithms
\item these tend to be sequentially optimal
\item and rapidly shrink subproblems to where they fit on one page
\item regardless of the size of that page
\end{itemize}

Assumption: optimal memory management
\begin{itemize}
\item so we can assume whatever page eviction policy works for us
\item because we know e.g. LRU with double memory is 2-competitive
  against optimal  management 
\item and we are ignoring constant factors
\end{itemize}

Example: scanning contiguous array
\begin{itemize}
\item $N/B$
\item without knowing $B$!
\end{itemize}

\subsection*{Matrix Multiply}

Adding matrices is easy
\begin{itemize}
\item $N \times N$ arrays
\item treat as vectors and scan to add
\item time $N^2/B$
\end{itemize}

Standard sequential algorithm
\begin{itemize}
\item $N \times N$ arrays
\item $N^3$ time
\item row major order
\item so scanning a row is very cheap
\item but scanning a column is super expensive
\item unless $N \times N < M$, evict rows on every column
\item so $N^3$ external memory accesses!
\item can we do better?
\end{itemize}

store second matrix column major?
\begin{itemize}
\item One output now requires $N/B$ reads
\item So $N^3/B$, an improvement
\item problem: what if next have to multiply in other order?
\item need fast transpose operation!
\item and turns out still not optimal---not leveraging large $M$
\end{itemize}

Divide and conquer
\begin{itemize}
\item mentally partition $A$ and $B$ into four blocks
\item solve 8 matrix-multiply subproblems
\item add up the pieces
\item sequentially, $T(N)=8T(N/2) + N^2 = O(N^3)$ (same as standard)
\item external memory: $T(N)=8T(N/2) + N^2/B = O(N^3/B)$ (same as
  column major)
\item because at level $i$ of recurrence do $2^iN^2/B$ fetches
\item wait, at size $N=\sqrt{M}$, whole matrix fits in memory
\item no more recursions.  $T(c\sqrt{M}) = O(M/B)$
\item this is at level $i$ such that $N/2^i = \sqrt{M}$, ie $i=\log
  N/\sqrt{M}$
\item so runtime $2^iN^2/B= N^3/B\sqrt{M}$
\end{itemize}

\subsection*{Binary Search Trees}

$B$-trees are optimal, but you need to know $B$

Divide and conquer
\begin{itemize}
\item We generally search an array by binary search
\item We query one value and cut problem in half
\item bad in external memory, because only halve on one fetch
\item $B$-tree is better, divides by $B$ on one fetch
\item but we don't know $B$
\end{itemize}

Van Emde Boas insight
\begin{itemize}
\item use new degree parameter $d$ to avoid confusion
\item problem of finding right child among $d$ in page is same problem
\item before, we set $d=B$ and ignored it because ``in memory''
\item now we don't know $B$ but can still set a $d$
\item recurrence: $T(N) = T(d) + T(n/d)$
\item Balance: set $d=\sqrt{N}$
\item $T(N) = 2T(\sqrt{N}) = 2^{\log\log N} = \log N$ (still
  sequentially optimal)
\item Wait, when $N=B$ we get the whole array in memory and finish in
  $O(1)$
\item so, stop at $i$ such that $N^{(1/2)^i}=B$, ie $B^{2^i}=N$ so $2^i
  = (\log N)/\log B =\log_B N$
\item so runtime $\log_B N$
\item draw a picture
\item see how you get a chain of $\log B$ items all on
  same page.
\item yields speedup
\end{itemize}

Generalizations:
\begin{itemize}
\item Dynamic
\item Buffer Trees
\end{itemize}

\subsection*{Linked Lists}

Want $O(1)$ insert/delete but $O(1/B$ scan
\begin{itemize}
\item Cache aware, we used dynamic splitting/merging of contiguous ``runs''
to limit fragmentation
\item How without $B$?
\item Need to combat fragmentation without knowing whether it exists
\item idea: defragment while scanning
\item all scanned items become unfragmented
\item use potential function (amount of fragmentation) to pay for
  defrag
\end{itemize}

Solution:
\begin{itemize}
\item on delete, just leave a hole
\item on insert, append to end of external memory
\item on scan, append all scanned items in order to end of memory
\end{itemize}

Analysis
\begin{itemize}
\item potential function: number of contiguous ``runs'' of elements in
  sequence on block
\item equals number of ``jumps'' of pointers from one block to another
\item suppose traverse $K$ items in $r$ runs
\item the cost is at most $K/B+r$
\item scanned items written to end
\item eliminates all but first and last run: 
\item reduces runs by $r-2$
\item so amortized cost $O(K/B)$
\end{itemize}

Controlling size
\begin{itemize}
\item all operations including traversal increase size of data
  structure
\item solution: after $N/B$ work, scan whole list
\item amortized cost $N/B$
\item but now list is sorted/contiguous
\item all previous blocks become free/irrelevant
\end{itemize}

\end{document}
