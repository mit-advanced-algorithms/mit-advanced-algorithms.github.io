\documentclass{article}
\usepackage{me}
\usepackage{amssymb}
\usepackage{amsfonts}
%\input{abbrevs}
\setlength{\parindent}{0pt}

\newcommand\vol{{\mbox{vol}}}

\newcommand{\RR}{\mathbb{R}}

\title{Interior-Point Methods}
\author{David Karger}

\begin{document}

\section{Last Time}
{\em Unconstrained minimization} problem: given a real-valued function $f$ over $\RR^n$, find its minimum $x^*$ (assuming it exists). That is, solve the problem
\[
x^*=\arg\min_{x\in \RR^n} f(x).
\]
\begin{itemize}
\item Gradient descent (strong convexity and smoothness) - condition number and update step (good convergence for condition number)
\item Newton's method. Second-order gradient descent, Update step. Or really gradient descent in local norm. 
\item Works really well when Hessians close, a bit less well otherwise. 
\item Idea - local norm is the best norm makes $f$ locally very well-condition (but not globally) 
\end{itemize}

\section{Solving the LPs via Barrier Method}

\begin{itemize}
\item Pros and cons of Simplex and Ellipsoid
\item New idea: don't deal with constraints explicitly - try to stay away from them - as that's where complexity of simplex comes in
\item How to quantify that? Need a potential that measures how close we are to a given constraint
\item Barrier: bounded on int of P, goes to infinity to boundary, strongly convex (as that's a good way to do it - add later?)
\item How to incorporate barrier trade-off? Just add them together and offset them against our objective function.

\item Turned constraint problem to unconstrained one - win! 
\item logarithmi barrier (figure!) and full objective
\item Need normalization.  Clearly, want to phase out perturbation $LP(\mu)$ and $y^*(\mu)$. 
\item How to do that? Optimize with GD/Newton's method?
\item Key idea: gradually! Keep always in the sphere of rapid convergence. 
\item Dynamical system vs. discretization. 
\item Key questions: when to stop, how many steps needed, how to start? 
\item compute the gradient - optimality condition, dual solution, complementary slackness
\item update step, definition of $\delta$, how many steps based on $\delta$. Condition on $\delta$. 
\item condition on $s$, derivation of Newton's decrement. 
\item analysis for min-circulation problem (note - no capacities but can fix that). 
\item gives fast max flow. 
\item Beyond: on max flow and my result
\item Dikin ellipsoid intuion. John's ellipsoid.
\item How to deal with starting point?
\end{itemize}

\end{document}

%%% Local Variables: 
%%% mode: latex
%%% TeX-master: t
%%% End: 



