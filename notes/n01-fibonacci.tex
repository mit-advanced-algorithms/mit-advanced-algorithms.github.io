\documentclass{article}
\usepackage{me}
\setlength{\parindent}{0pt}

\title{Fibonacci Heaps}
\author{David Karger}

\begin{document}
\marnote{This material takes 1:15}

\section{Heaps}

Shortest path/MST motivation.  Discuss Prim/Dijkstra algorithm.

Note: lots more decrease-key than delete.

Response: \emph{balancing} 
\begin{itemize}
\item trade off costs of operations
\item making different parts equal time.
\end{itemize}

$d$-heaps:
\begin{itemize}
\item  $m\log_d n + nd\log_d n$. 
\item set $d=m/n$
\item $O(m\log_{m/n} n)$
\item already nice, gives $O(n^2)$ (matching trivial alg) on complete graph.
\end{itemize}

\subsection{Fibonacci Heaps}

Fredman-Tarjan, JACM 34(3) 1987. 

{\tt  http://dl.acm.org/citation.cfm?id=28874} 

Key principles:
\begin{itemize}
\item Lazy: don't work till you must
\item If you must work, use your work to ``simplify''
  data structure too
\item force user (adversary) to spend lots of time to make you work
\item analysis via potential function measuring ``complexity'' of
  structure.  
\begin{itemize}
\item user has to do lots of insertions to raise potential, 
\item so you can spread cost of complex ops over many insertions.  
\item potential says how much work \emph{adversary} has done that you
  haven't had to deal with yet.  
\item You can charge your work against that work.
\end{itemize}
\item another perspective: procrastinate.  if you don't do the work,
  might never need to.
\item Why the name?  Wait and see.
\end{itemize}

Lazy approach:
\begin{itemize}
\item During insertion, do minimum, i.e. nothing.
\item Indeed, for only insert, don't even need to remember item!
\item but we also want delete min, so save items
\item For first delete-min, cost is $n$
\item So, amortized cost 1.
\end{itemize}

Problem?
\begin{itemize}
\item issue with second and further delete mins
\item $n$ delete mins cost $n^2$---means amortized $n$
\end{itemize}

Solution: Use your work to simplify
\begin{itemize}
\item As do comparisons, remember outcomes
\item point from loser to winner
\item creates ``heap ordered tree'' (HOT)
\item not full or balanced, but heap ordered
\item in fact, this tree will be a \textbf{caterpillar}: one spine
  with legs
\item but we can consider general trees with this property
\item remembers everything we figured out while finding min
\end{itemize}

Benefit: on delete-min, new min is child of current root
\begin{itemize}
\item find by scanning children
\item but if we build a star, no benefit over brute force
\item can we force few children?
\end{itemize}

Can we be lazy and simultaneously prepare for future laziness?
\begin{itemize}
\item Problem is if we start with min and it beats everything
\item it plays too often
\item how can we ensure nobody plays too often?
\item Tournament
\begin{itemize}
\item pairs compete
\item winners pair to play each other
\item winners of winners pair
\item etc.
\item Still $n-1$ comparisons
\item But now each node plays only $\log n$ times
\end{itemize}
\item What is resulting HOT tree structure?
\begin{itemize}
\item \emph{binomial trees}
\begin{verbatim}
x  x    x  x    x  x    x  x

x       x       x       x
 \       \       \       \
  \       \       \       \
   x       x       x       x


x---             x---     
 \  \             \  \   
  \  -             \  -  
   x  x             x  x 
       \                \
        \                \
        x                 x

x---------             
 \  \     \         
  \  -     ------      
   x  x          x---     
       \          \  \ 
        \          \  - 
        x           x  x 
                        \
                         \
                          x
\end{verbatim}
\item degree $d$ has $2^d$ descendants
\item so max degree $\log n$
\end{itemize}
\end{itemize}

Generalize
\begin{itemize}
\item As we interleave inserts and deletes, can't run a tournament
  from clean start
\item but can achieve same result
\item So max degree $O(\log n)$ for $n$ items
\item how? 
\begin{itemize}
\item record each node's degree
\item i.e., its round in the tournament so far
\item only link HOTs of same degree
\item same ``union by rank'' idea as union find
\end{itemize}
\end{itemize}

induction: if only link heaps of same degree, than any degree-$d$
  heap has $2^{d}$ nodes.
\begin{itemize}
\item creates ``binomial trees'' saw above
\item ``Binomial heaps'' do this aggressively---when delete items,
  split up trees to preserve exact tree shapes.
\item but we'll be sloppier/faster
\end{itemize}

Consolidation algorithm for list of arbitrary HOTS
\begin{itemize}
\item inserts/deletes may create arbitrary set of them
\item create buckets
\item bucket HOTs by degree (maintain degree in node)
\item start at smallest bucket
\item link pairs till $<2$ left.  
\item next bucket.
\item at end, collect items from buckets and discard buckets
\end{itemize}

For delete min
\begin{itemize}
\item remove min
\item add children HOTs
\item consolidate to identify new min
\end{itemize}

Combine with lazy insert
\begin{itemize}
\item keep pointer to min root (or just scan on delete min)
\item on insert, just add new (degree 0) HOT to set of HOTs
\item Formalize notion that scan through inserted items is ``paid
  for'' by consolidation
\end{itemize}

Heap ordered trees implementation
\begin{itemize}
\item definition
\item represent using left-child, parent, and sibling pointers (both directions)
\item keep double linked list of HOTs
\item and a pointer to min HOT root.
\item so in constant time, can find min
\item in constant time, can add item
\item in contant time, can link two HOT lists (Fibonacci heaps are
  \emph{mergeable} in constant time)
\item time to delete-min equal to number of roots, and simplifies struct.
\end{itemize}


Insert solution idea: adversary has to do many insertions to make consolidation expensive.
\begin{itemize}
\item analysis: \textbf{potential function} $\phi$ equal to number of roots.
\begin{itemize}
\item amortized$_i=$ real$_i+\phi_i-\phi_{i-1}$
\item then $\sum a_i=\sum r_i+\phi_n-\phi_0$
\item upper bounds real cost if $\phi_n \ge \phi_0$.
\item sufficient that $\phi_n \ge 0$ and $\phi_0$ fixed
\end{itemize}
\item insertion real cost 1, potential cost 1.  total 2.
\item deletion: take $r$ roots and add $c$ children, then do $r+c$ scan work.
\item $r$ roots at start, $\log n$ roots at end.  So, $r-\log n$ potential decrease
\item so, total work $O(c+\log n)=O(\log n)$
\end{itemize}
Result: $O(1)$ amortized insert, $O(\log n)$ amortized delete

What about decrease-key?
\begin{itemize}
\item basically easy: cut off node from parent, make root.
\item constant time decrease-key
\item what goes wrong?
\item may violate exponential-in-degree property
\end{itemize}

``saving private ryan''
\begin{itemize}
\item fix: if a node loses more than one child, cut it from parent,
  make it a root (adds 1 to root potential---ok). 
\item implement using ``mark bit'' in node if has lost 1 child (clear
  when becomes root)
\item may cause ``cascading cut'' until reach unmarked node
\item why 2 children?  We'll see.
\end{itemize}

Analysis: must show
\begin{itemize}
\item  cascading cuts ``free'' 
\item tree size is exponential in degree
\end{itemize}

Second potential function: number of mark bits.
\begin{itemize}
\item if cascading cut hits $r$ nodes, clears $r$ mark bits
\item adds 1 mark bit where stops
\item amortized cost: $O(1)$ per decrease key
\item note: if cut without marking, couldn't pay for cascade!
\begin{itemize}
\item this is binomial heaps approach.  may do same $O(\log n)$
  consolidation and cutting over and over.
\end{itemize}
\item Wait, problem
  \begin{itemize}
    \item can't have 2 incompatible potential functions
\item new root per cut 
\item adds to first potential function
\item and thus to amortized cost
\item fix: double new potential function.
\item use one unit to pay for cut, one to pay for increase in 1st potential
\item so, doesn't harm first potential function analysis
\end{itemize}
%\item idea: cascading cuts for tree of min-degree 2.  number nodes
%  less than twice number of leaves, which is number of decrease keys.
\end{itemize}

Analysis of tree size:
\begin{itemize}
\item node $x$.  consider \emph{current} children in order were added
  (indexed from 1, not 0).
\item claim: $i^{th}$ remaining child (in addition order) has degree at
least $i-2$
\item proof: 
\begin{itemize}
\item Let $y$ be $i^{th}$ added child
\item When added, the $i-1$ items preceding it in the add-order were
  already there
\item i.e., $x$ had degree  $\ge i-1$
\item So $i^{th}$ child $y$ had (same) degree $\ge  i-1$
\item $y$ could lose only 1 child before getting cut
\end{itemize}
\item let $S_k$ be minimum number of descendants (inc self) of degree
  $k$ node.  Deduce $S_0 =1$, $S_1=2$, and
\begin{align*}
S_k &\ge 2+ \sum_{i=0}^{k-2} S_i
\end{align*}
\item to upper bound, solve equality
\begin{align*}
S_k &= 2+\sum_{i=0}^{k-2} S_i\\
S_k - S_{k-1} &= S_{k-2}
\end{align*}
\item deduce $S_k \ge F_{k+2}$ fibonacci numbers
\item reason for name
\item we know $F_k \ge \phi^k$
\item so biggest possible $k=\log_\phi n$
\end{itemize}

Practical?
\begin{itemize}
\item non-amortized versions with same bounds exist (good for realtime
  apps).
\item Constants not that bad
\item ie fib heaps reduces comparisons on moderate sized problems
\item we've done experiments with graph problems where fib wins
\item but, regular heaps are in an array
\item fib heaps use lots of pointer manipulations
\item lose locality of reference, mess up cache.
\item work on ``implicit heaps'' that don't use pointers
\item Pettie developed an optimal one
\end{itemize}

Cannot beat these bounds, since can use to sort.

\subsection{Minimum Spanning Tree}


minimum spanning tree (and shortest path) easy in $O(m+n\log n)$.

More sophisticated MST:
\begin{itemize}
\item why $n\log n$?  Because deleting from size-$n$ heap
\item idea: keep heap small to reduce cost.
\begin{itemize}
\item choose a parameter $k$
\item run prim till region has $k$ neighbors (i.e. heap members)
\item set aside and start over elsewhere.
\item heap size bounded by $k$, delete by $\log k$
\item ``contract'' regions (a la Kruskal) and start over.
\end{itemize}
\end{itemize}

Formal:
\begin{itemize}
\item phase starts with $t$ vertices.  
\item pick max region size $k$ (TBD)
\item unmark all vertices and repeat following
\begin{itemize}
\item choose unmarked vertex
\item Prim until attach to marked vertex or heap reaches size $k$
\item ie $k$ edges incident on current region
\item (maybe both ends inside; that's fine too)
\item mark all vertices in region
\end{itemize}
\item contract graph in $O(m)$ time and repeat
\end{itemize}

Analysis:
\begin{itemize}
\item time for phase: $m$ decrease keys, $t$ delete-mins from size-$k$
  heaps, so $O(m+t\log k)$.
\item balance: 
\begin{itemize}
\item have to spend $m$
\item so can make $t\log k = m$ ``for free'' (no asymptotic change)
\item set $k=2^{m/t}$.
\item runtime of phase $O(m)$
\end{itemize}
\item number of phases:
\begin{itemize}
\item At end of phase, each compressed vertex ``owns'' $k$ edges (one
  or both endpoints)
\item so next number of vertices $t' \le m/k$
\item so $k' = 2^{m/t'} \ge 2^k$
\item what is starting $k$?  
\begin{itemize}
\item Initially $t=n$ 
\item so need $n\log k = O(m)$ for $O(m)$-time phase
\item so can take $k=2^{m/n}\ge 2$
\item (note if $m>n\log n$ then $k>n$ and one phase suffices to finish)
\end{itemize}
\item when reach $k=n$, Prim step solves whole graph, done.
\item number of phases: $\beta(m,n) = \min\set{i \mid \log^{(i)} n \le
    m/n} \le \log^* n$.
\end{itemize}
\end{itemize}

Remarks: 
\begin{itemize}
\item subsequently improved to $O(m\log\beta(m,n))$ using edge packets
\item chazelle improved to $O(m\alpha(n)\log\alpha(n))$ using
  ``error-prone heaps''
\item ramachandran gave optimal algorithm (runtime not clear)
\item randomization gives linear.
\item fails for Dijkstra.  
\item Why?  Because cannot contract.
\end{itemize}



\end{document}
%%% Local Variables: 
%%% mode: latex
%%% TeX-master: t
%%% End: 
