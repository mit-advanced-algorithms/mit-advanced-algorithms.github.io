\documentclass{article}
\usepackage{me}
%\input{abbrevs}
\setlength{\parindent}{0pt}

\title{Min Cost Flow}
\author{David Karger}

\begin{document}

  Gross flow formulation (otherwise double-count).

  Many different max-flows in a graph.  How compare?
  \begin{itemize}
  \item {\em cost} $c(e)$ to send a unit of flow on edge $e$
  \item (that's why no $c$ for max-flow capacity)
  \item find max-flow minimizing $\sum c(e)f(e)$
  \item costs may be positive or negative!
  \item note: pushing flow on cost $c$ edge create residual cost $-c$
    edge.
  \item also easy to find min-cost flow of given value $v$ less than max
    (add bottleneck source edge of capacity $v$)
  \end{itemize}

  Clearly, generalizes max-flow.  Also shortest path:
  \begin{itemize}
  \item How send flow 1 unit of flow?
  \item just use shortest path
  \item more generally, flow decompose into paths and cycles
  \item cost of flow is sum of costs of paths and cycles.
  \end{itemize}

\marnote{2016 L10 start}

  Min-cost circulation:
  \begin{itemize}
  \item no source or sink
  \item just find flow satisfying balance everywhere, min-cost
  \item if satisfy balance everywhere, all flow must be going in
    circles!
  \item more formally: circulation can be decomposed into just cycles.
  \item hard to define in max-flow perspective, but makes sense once allow
    negative cost arcs.
  \item reduction to min-cost flow: add disconnected $s$, $t$.
  \item reduction from min-cost flow:
    \begin{itemize}
    \item add $t$-$s$ arc of ``infinite'' capacity, ``infinite'' negative
      cost
    \item of course, circulation will push max possible through this edge
    \item how much can it?  max $s$-$t$ flow
    \item so of course, suff to assign capacity equal to max-flow value
    \item see later, sufficient to assign cost $-nU$ (good for scaling)
    \end{itemize}
  \item another reduction from min-cost flow:
    \begin{itemize}
    \item find any old max-flow $f$
    \item consider min-cost flow $f^*$
    \item difference $f^*-f$ is a circulation
      \begin{itemize}
        \item $-f$ is sending flow backwards from $t$ to $s$
        \item diff of two equal    capacity flows is a circulation
      \end{itemize}
\begin{itemize}
    \item claim feasible in $G_f = G-f$.  
    \item if $f^*_e-f_e$ positive, have flow $f^*_e-f_e$ on $e$
    \item meanwhile capacity in $G_f$ is $u_e-f_e$
    \item but $f^*_e<u_e$, done
    \item if $f^*_e-f_e$ negative, have flow $f_e-f^*_e$ on $e'$ (reverse)
    \item meanwhile residual capacity is $f_e$, greater!
    \item (note $e'$ is reverse residual arc for $e$, not the original
      $G$ edge reverse to $e$ (may have different cost, and must
      be considered entirely separately)
\end{itemize}
    \item so find circulation $q$ in $G_f$.
    \item $q+f$ is a feasible flow in $G$ (note: flow+circulation$=$flow with same demand)
    \item cost is $c(q)+c(f)$
    \item so adding min-cost $q$ in $G_f$ yields min-cost flow
    \end{itemize}
  \end{itemize}

\marnote{2011 Lecture 8 End}

\subsection*{Deciding optimality:}
Given a max-flow.  How decide optimal?
  \begin{itemize}
  \item by above, optimal if the cost of the min-cost residual circulation is 0
  \item suppose not.  so have negative cost circulation
  \item decomposes into cycles of flow
  \item one must have negative cost.
  \item so, if $f$ nonoptimal, negative cost cycles in $G_f$
  \item converse too: if negative cost cycle, have negative cost
    circulation.  So min-cost$<0$.
  \end{itemize}

\subsection*{Cycle Canceling Algorithm}

(Klein)

Cycle canceling algorithm:
\begin{itemize}
\item start with any max-flow (or min-cost circulation problem)
\item find a negative cost cycle
\item push flow around it
\item analogue of generic augmenting paths under circulation reduction
\item how many times?
\begin{itemize}
\item cost decreases by 2 each push
\item initial cost (in residual graph) 0
\item final cost at least $-mCU$ (why?)
\item so $mCU$ iterations.
\end{itemize}
\end{itemize}

How find negative-cost cycle?
\begin{itemize}
\item think back to shortest paths
\item dijkstra only works for positive edges
\item but Floyd, Bellman-Ford work for negative edges too
\item {\bf Unless} have negative cost cycle
\item then, of course, shortest paths undefined
\item however, Floyd/BF will indentify one negative cycle that is
  wrecking things.
\item Floyd $O(n^3)$, BF $O(nm)$.
\item fancy scaling algorithm running in $O(m\sqrt{n}\log C)$ also
  known.
\end{itemize}

So: time bound of $O(m^2\sqrt{n}CU\log C)$ or $O(nm^2CU)$ time.

Slow, and not even weakly polynomial!  Let's do better.

Later (homework), you'll see that cycle canceling is a good idea
combined with scaling.  But for now,
lets take a different approach.

\marnote{2013 Lecture 11 end}

\section*{Prices and Reduced Costs}


Recall: flow/circulation optimal iff no residual negative cost cycle.

Reduced-Cost optimality.
\begin{itemize}
\item another way to decide optimal
\item based on LP duality (see next week)
\item given min-cost flow problem
\item one way to solve: compute opt flow (command economy)
\item alternative: market economy!
\item infinite boba tea {\em supply} at source $s$, infinite {\em
    demand} for boba at MIT source $t$
\item give boba away at $s$
\item at $t$  will pay for any boba they can get
\item creates ``market'' where prices get set at all vertices
\item {\em price} $p(v)$ for widget at vertex $v$
\item costs reflect shipping charges
\item circulation version: boba free everywhere, but possible profit
  for shipping in circles (government subsidies?)
\end{itemize}

When is a set of prices ``stable''?
\begin{itemize}
\item suppose have $p(v),p(w),c(vw)$ such that $p(w) \ge p(v)+c(vw)$
\item then merchant can make money by buying at $v$, shipping to $w$,
  and selling.
\item {\em reduced cost} $c_p(vw) = c(vw)+p(v)-p(w)$
\item reduced costs reflect ``net cost'' of moving boba from $v$
  to $w$
\item merchant will want to ship if reduced cost negative.
\item this will increase demand at $v$, raise price: so current prices wrong.
\item what stops him?  no residual capacity!
\item Also, if there is any residual $s$-$t$ capacity, more will get
  shipped
\begin{itemize} 
\item $t$ will pay whatever it costs
\item so prices will rise on $s$-$t$ path until shipping happens
\item so can only be stable if shipping max-flow
\end{itemize}
\item {\bf Definition:} a price function is {\em feasible} for a (residual) graph if no (residual) arc has negative reduced cost
\end{itemize}

Important observation: prices don't affect overall cost:
\begin{itemize}
\item reduced cycle cost same as original cycle cost
\item cost of every $v,w$ path changes by $p(v)-p(w)$
\item negative cost cycles still negative
\item min-cost paths still min-cost
\item every flow of same value sees same cost change (path decomposition)
\item also, all prices can be shifted by a constant without change
\end{itemize}

{\bf Claim:} A circulation/flow is optimal iff there exists a feasible
price function on its residual graph.
\begin{itemize}
\item first: price function implies optimal
\begin{itemize}
\item recall: optimal iff no negative cost cycles in residual graph
\item suppose have feasible price function
\item so no negative reduced-cost edges in residual 
\item so no negative cost cycles under reduced costs
\item thus no negative cost cycles under original costs!
\item key: cost of cycle unchanged under reduced costs (prices
  telescope)
\end{itemize}
\item converse: suppose have optimal flow/circulation
\begin{itemize}
\item then residual graph has no negative cycles.  How find prices?
\begin{itemize}
\item what does no negative cycles allow?  Shortest paths from source!
\item i.e., find price of boba if give away at $s$
\item formalizes our market idea
\item problem: only defines prices on vertices reachable from $s$
\item attach vertex $s'$ with 0-cost edges to everywhere
  \item (or just add huge-cost arcs from $s$)
\item  compute shortest residual paths from $s'$
\item well defined because no negative cycles
\item define $p(v)=d(s',v)$
\item (note: yields reduced cost of shipping a unit from $s'$)
\end{itemize}
\item note $p(w) \le p(v)+c(vw)$
\item thus $c_p(vw) \ge 0$ everywhere, as desired.
\item 0-edges guaranteed finite price at all vertices (this why didn't
  do paths from $s$)
\end{itemize}
\item {\bf note:} given min-cost flow, can verify optimality via one
  shortest path computation!
\item {\bf note:} Using this computation, all edges on shortest paths
  from $s'$ have reduced cost exactly 0 (useful for later).
\end{itemize}

Feasible prices \emph{certify} a min-cost flow
\begin{itemize}
  \item given prices, checking optimality takes linear time
\end{itemize}

Price function is a general concept
\begin{itemize}
\item like min-cut for max-flow
\item shifts difficult test for global optimum to easier local test
\item by creating a ``bound'' that optimum cannot beat
\item so if you meet it you must be optimum
\item (later, we'll see there's actually a \emph{number} on the
  feasible price function that is \emph{equal} to the value of the min
  cost flow.
\end{itemize}



\section*{Algorithms}


\subsection*{Shortest Augmenting Path}

Idea:
\begin{itemize}
\item Previously, started with max-flow, made min-cost by cycle
  canceling
\item Now, start with empty ``min-cost'' flow, make optimal by augmenting to
  maximum
\item For min-cost, assume starting with no negative cycles
\item Augment in ways that don't create them
\item When reach max-flow, know optimal
\end{itemize}

Idea: augmenting paths
\begin{itemize}
\item natural greedy strategy: what augmenting path to use?
\item Repeatedly augment shortest (min-cost) path (just like max-flow SAP)
\end{itemize}

Claim: shortest augmenting path doesn't create negative cycles
\begin{itemize}
\item Suppose none at start
\item Suppose shortest augmenting path creates one
\item only new edges are residual from augmenting path
\item So cycle includes a residual edge
\item So is connected to the shortest augmenting path
\item So insert cycle into SAP at some vertex where they intersect
\item ie, add ``send flow around residual cycle'' to augmenting path
\item Yields a shorter augmenting path
\end{itemize}


Alternative correctness argument: use price function.
\begin{itemize}
\item key: price function changes {\em value} of paths, but not which
  is shortest (proof: telescoping)
\item claim: at no time does residual graph have negative cost cycle
\item proof by induction
\item if currently true, can compute shortest paths from $s$ to define
  (new) prices
\item result: all edges on shortest $s$-$t$ paths have cost 0
\item so augmentation is along cost 0 edges
\item create residual arcs, but only cost 0
\item so, no negative reduced cost arcs
\item so, no negative cost cycles.  induction proved.
\item so after $f$ iterations, have flow value $f$ of minimum cost.
\item Wait, why no $s'$ supersource?
\begin{itemize}
\item shortest paths from $s$ assigns a new price to every vertex
  reachable from $s$
\item new prices don't affect cycle costs anywhere
\item some vertices/edges don't get repriced because not reachable
\item but then we won't augment through them
\item So: if cannot reach, then (by induction) can't ever reach.
\item Discarding doesn't change max-flow or future shortest paths
\item Alternatively, add a large value to every vertex $v$ not reachable from $s$.
\item Makes edges \emph{from} $v$ to anything reachable from $s$ very positive
  \item And nothing goes from $s$-reachable to $v$, so no negative.
\end{itemize}
\end{itemize}

Is SAP strongly polynomial for MCF?
\begin{itemize}
\item no
\item we had length-1 edges when we did SAP for max-flow
\item and SAP increased distance to $n$ to finish, no dependence on numbers
\item but now our lengths are edge costs---numeric, maybe large
\item so progress in distance is no longer a number-independent measure
\end{itemize}

Special case algorithm:
\begin{itemize}
\item integer capacity edges (or at least small flow),
\item no negative cost edges.
\item so mincost circ 0, but min-cost flow maybe not
\item one unit of flow per (integer) augmentation
\item Time analysis: $O(mf\log n)$ for flow of value $f$
\item so $O(mn\log n)$ for typical unit capacity graph
\item since {\em capacity} of flow at most $n$.
\end{itemize}

Note: don't need explicit prices to run
\begin{itemize}
\item prices don't change which paths are shortest
\item but with good prices, can use Dijkstra's fast algorithm.
\end{itemize}

Limitations:
\begin{itemize}
\item unit capacity
\item no negative edges
\end{itemize}

Lets handle negative arcs.
\begin{itemize}
\item Min-cost circulation
\item Still unit-cap edges (or at least small flow)
\item Can't do shortest paths to eliminate
\item So, try brute force
\item saturate all negative arcs
\item creates excesses, deficits
\item now, need to send excess back to deficits
\item find min-cost flow in residual graph
\item (know one exists, namely saturations)
\item sum of saturations and flow back is circulation
\item and, since min cost flow back, min-cost circulation
\item now, no negative arcs
\item so shortest augmenting path works
\item total excess is $m$
\item so $m$ augmentations send it back
\item time $O(m^2\log m)$
\end{itemize}

Handle min-cost flow via max-flow plus above min-cost circulation


\section*{Scaling Min-Cost flow}


\subsection*{Capacity Scaling}

Scale in capacity bits one at a time. (Edmonds Karp 1972)
\begin{itemize}
\item min-cost circulation problem
\item in scaling step, double capacities, possibly add 1
\item double flow values
\item changes residual capacities by at most 1
\item so some negative reduced cost arcs might get capacity 1
\item how fix? just like unit case
\item saturate all negative cost arcs (creates excess, deficits)
\item find min-cost flow to ship excesses to deficits (know one exists)
\item total excess $O(m)$
\item so $O(m)$ shortest augmenting paths solve
\item time $O(m^2 \log n)$
\item $O(\log U)$ phases
\item total $O(m^2 \log n \log U)$.
\end{itemize}

\marnote{2012 Lecture 11 End}
\marnote{2013 Lecture 12 End}

\subsection*{Cost Scaling}

HW

\iffalse
{\bf skip this.  takes 15 minutes to survey.  just mention exists.}

Idea:
\begin{itemize}
\item Recall goal: flow/price function such that all residual arcs
  have nonnegative cost
\item On capacity scaling, allow negative arcs that are almost nonresidual
\item now, allow residual arcs that are almost nonnegative.
\item a flow/price-function is {\em $\epsilon$-optimal} if every
  residual arc has cost at least $-\epsilon$
\end{itemize}

Outer loop:
\begin{itemize}
\item scaling phase reduces $\epsilon$ to $\epsilon/2$
\item Any flow is $C$-optimal (gives starting point)
\item problem: may get nonintegral prices.  when can we stop?
\item (note with flow, knew stayed integral so $1/2$-optimal was optimal.
\item claim: an $1/n$-optimal flow is optimal
\item proof: no negative cycle.  cycle cost unchanged, was originally
  an integer.
\end{itemize}

Scaling phase.  Many different approaches, we'll start with one based
on push relabel.
\begin{itemize}
\item start by saturating all negative cost arcs.  Creates excesses,
  deficits.
\item preserve $\epsilon/2$-optimality (by juggling flow, prices)
\item uses pushes to send excesses to deficits
\item need notion of ``downhill''
\item An arc is {\em admissible} if negative reduced cost
\item pushes are only on admissible arcs
\item if an active vertex has no admissible arc, {\em relabel} it.
\item reduce its price by $\epsilon/2$
\item decreases cost of outgoing arcs, increases incoming
\item since no outgoing arc started negative, all outgoing arcs
  $>\epsilon/2$, so still $\epsilon/2$-optimal.
\item no incoming admissible arcs!
\item clearly, so long as have excess, can push or relabel
\end{itemize}

Claim: no vertex relabeled more than $3n$ times.
\begin{itemize}
\item Initially, had pushed flow from supersink $t$ to supersource $s$
\item then did some push of active flow
\item so can think of current situation as a preflow with source $t$
\item decompose into paths from $t$ ending at current excesses.
\item consider path $P$ from $t$ to $v$
\item since started $\epsilon$-optimal, cost of path at least
  $-\epsilon n$
\item so reverse path originally had cost at most $\epsilon n$
\item this is a residual path in current graph
\item so (by $\epsilon/2$-optimality) has cost at least $-\epsilon/2
  n$
\item each relabel of $v$ decreased path cost by $\epsilon/2$
\item so only $3n$ relabels.
\end{itemize}

Claim: $O(nm)$ saturating pushes

Claim: admissible network acyclic

Claim: $O(n^2 m)$ nonsaturating
\begin{itemize}
\item let $g(v)$ be number of nodes admissibly reachable from
  $v$
\item $\phi=\sum g(v)$ over active $v$
\item saturating push increases by $n$ (makes dest active)
\item relabel increases by $n$ ($v$ can reach more stuff, but noone
  can reach $v$)
\item nonsaturating push $(v,w)$ decreases by 1 ($g(w)<g(v)$)
\end{itemize}

Cleverer methods give $O(n^3)$.

Other approaches (eg double-scaling) gives $O(nm)$ per scaling phase.
\fi

\marnote{2011 Lecture 9 end}
\marnote{2013 Lecture 13 start}
\section*{Complementary Slackness}
Looking ahead to linear programming.
\begin{itemize}
\item what do merchants do?  think about reduced costs.
\item if reduced cost positive, noone will ship: flow 0 on edge
\item if negative, will ship: flow equals capacity on edge
\item if 0, don't care: flow arbitrary on edge.
\item flow with these properties has {\em complementary slackness:},
  another optimality condition.
\item complementary slackness implies optimal, since no residual
negative  reduced cost arcs
\item suppose optimal.  assign prices so no residual negative cost
  arcs.  implies any negative reduced cost original arc is saturated,
  any positive reduced cost arc empty (since reverse would have neg cost)
\end{itemize}

Complementary slackness is on {\bf original graph}, corresponds to
feasible pricing on {\bf residual graph}

Given feasible price function, can find opt flow easy:
\begin{itemize}
\item delete positive redued cost arcs (no flow in optimum)
\item saturate negative arcs
\item creates excesses/deficits at nodes
\item ship excesses to deficits on 0-cost arcs
\item know this can be done, since optimum does
\item do by creating supersource for excesses, supersink for deficits,
  finding max-flow on 0 arcs
\end{itemize}

saw converse: given flow, need to compute optimum distances.  So
  min-cost flow really is max-flow plus shortest paths!
\begin{itemize}
\item some flow algs use prices implicitly, to prove correctness
\item others use explicitly, to guide solution.
\end{itemize}

\section*{Conclusion}

Finish remarks on min-cost flow.
\begin{itemize}
\item Strongly polynomial algorithms exist.
  \begin{itemize}
  \item Tardos 1985
  \item minimum mean-cost cycle
  \item reducing $\epsilon$-optimality
  \item ``fixing'' arcs of very high reduced cost
  \item best running running time roughly $O(m^2)$
  \item best scaling time (double scaling) $O(mn \log\log U \log C)$.
  \end{itemize}
\end{itemize}

\end{document}
